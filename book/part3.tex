%\chapter{Introduction}
%\label{introduction}
%
%
%
%%bibliography
%\usepackage{natbib}
%\bibpunct[:]{(}{)}{,}{a}{}{,}
%
%% phonological examples
%%\usepackage{simplex}
%\usepackage{amsmath}
%
%% fonts
%%\usepackage{mathspec}
%%\setmainfont[Mapping=tex-text]{Linux Libertine}
%%\setmathfont(Digits,Greek,Latin){Linux Libertine}
%%\usepackage{microtype}
%%\usepackage{coptic}
%
%
%% tables and figures
%\usepackage{booktabs}
%\usepackage{graphicx}
%\usepackage{floatrow}
%\usepackage{multirow}
%\usepackage{enumitem}
%\newfloatcommand{capbtabbox}{table}[][\FBwidth]
%\setlist{noitemsep}
%
%% Add packages and definitions you want to use here:
%\usepackage{times}
%\usepackage{multirow,sectsty}
%\usepackage{setspace}
%\usepackage{subfigure,graphicx}
%\usepackage{amsmath,amsthm,amsfonts, amssymb}
%\theoremstyle{definition} \newtheorem{definition}{Definition} 
%\usepackage{linguex}
%% \usepackage{betababel}
%\usepackage[english,greek]{betababel}
%\usepackage{tikz-qtree}
%\usepackage{tikz}
%\usetikzlibrary{arrows,automata,chains,matrix,positioning,scopes}
%
%\usepackage[normalem]{ulem}
%
%\usepackage{pdfpages}
%
%\usepackage{natbib}
%
%\usepackage{epigraph}
%\usepackage{hyperref}
%
% \usepackage[only, llbracket,rrbracket]{stmaryrd}
% \newcommand{\sem}[1]{\ensuremath{\{ #1 \} }}
% \newcommand{\pair}[1]{\ensuremath{\langle #1 \rangle}}
% \newcommand{\la}{\ensuremath{\lambda}}
% \newcommand{\inter}[1]{\ensuremath{\llbracket#1\rrbracket}}
%
%\newcommand*\circled[1]{\tikz[baseline=(char.base)]{
%            \node[shape=circle,draw,inner sep=2pt] (char) {#1};}}
%
%
%\newcommand{\comm}[1]{}
%\long\def\symbolfootnote[#1]#2{\begingroup%
%\def\thefootnote{\fnsymbol{footnote}}\footnote[#1]{#2}\endgroup}
%
%\begin{document}

%\setcounter{chapter}{0}

In this document, I will argue that violations of strict locality in ditransitive passives can be caused by case sensitivity in subject positions. The focus of this paper will be on recipient ditransitives of the type "The \textbf{agent} gave the \textbf{recipient} the \textbf{theme}". Since at least \cite{Oehrle.1976}, this class has been identified as involving the causation of the (potential) possession of the \textbf{theme} by the \textbf{recipient}. Note that this is distinct from verbs of motion that introduce \textbf{goals} (i.e. the endpoint of an arc of motion). Some verbs are ambiguous between a \textbf{goal} and \textbf{recipient} interpretation \citep{Hovav.2008}. In this paper, we will focus on data from a variety of German languages, drawing on verbs that exclusively introduce recipients (e.g. "give").

The structure of the document is as follows. First, I will start by demonstrating what a naive version of locality would predict that only recipients should become subjects in passives (i.e. "John was given the book" vs "*The book was given (to) John"). I will then give evidence that this naive prediction does not hold (i.e. theme passivization exists). I will then show that Germanic languages show case sensitivity in subject positions. Finally, I will show that other solutions proposed in the literature provide a poor explanation of these cases and that the case sensitivity properties perform better. I will end with some implications for this work outside of ditransitives (specifically in the morphosyntax of ergativity).

\section{Naive Locality}
This argument relies on there being a locality violation in ditransitive passives. The naive version of locality, which we will initially assume, holds that "in considering possible objects for movement only the base generated structurally highest argument is able to be targeted". This allows us to identify all possible locality violations before seeking possible solutions. In this section, I will provide evidence from English, German and Icelandic that suggest that recipients are base generated in a structurally higher position than themes.

For English, we focus on demonstrating the asymmetry in the double object construction (e.g. "The man gave the boy the book"). Prepositional object constructions (e.g. "The man gave the book to the boy") will be dealt with in a later section, where it will be considered as a solution to the locality problem. 

\cite{Barss.1986} show that the recipient has a higher structural position than the theme in English using a number of diagnostics for structural asymmetries. The data for English is given here, but the same judgements have been replicated in a number of other Germanic lanuages \citep[AND OTHER CITATIONS TO BE ADDED]{Falk.1990}.
\bigskip

INSERT BARSS AND LASNIK EXAMPLES HERE
\bigskip

This same asymmetry in base word order is seen in German. All things being equal, both of the following structures are grammatical in German:
\bigskip

EXAMPLES OF GERMAN DATIVE--ACCUSATIVE and ACCUSATIVE--DATIVE WORD ORDERS
\bigskip

However, \cite{Lenerz.1977} showed that it is possible to determine which of the word orders is underlying. The best method to do this in German relies on a German restriction on scrambling, namely that non-contrastive prosodically focused phrases cannot scramble. The most common source of non-contrastive prosodic focus is when the targeted phrase is the answer to a wh-question (i.e. information focus). Using this diagnostic, \cite{Lenerz.1977} showed that the recipient--theme word order is basic and the theme--recipient order is derived:
\bigskip

LENERZ'S QUESTION-ANSWER PAIRS
\bigskip

Finally, Icelandic (in combination with Germanic typology) provides additional support for the notion that the recipient--theme order is base generated and that theme--recipient orders are derived. \cite{Dehe.2004} has shown that the recipient--theme is the only grammatical order in Modern Icelandic (modulo rightward heavy-NP shift). 
\bigskip

DEHE'S EXAMPLES
\bigskip

This also shows up in the Icelandic Parsed Historical Corpus. Old Norse shows variation between theme--recipient and recipient--theme word orders. Over time, the theme--recipient order is lost.
\bigskip

ICEPAHC GRAPHS
\bigskip

While languages like Icelandic exist that only allow for recipient--theme word orders, no corresponding Germanic language exists that only allows for theme--recipient word orders. Taken together with the evidence of object asymmetries and the German evidence for base generation, we conclude that the recipient--theme order is base generated with a structurally higher recipient. Under the naive approach to locality we are initially assuming, this means that only the recipient should be able to raise in the passive.

\section{Existence of Strict Locality Violations}
Remember that under the naive version of locality that we are currently considering, any evidence of theme passivization (i.e. where the theme becomes a subject in the passive) counts as a locality violation. I will show that \textbf{all} of the Germanic languages allow for theme passivization.

In Icelandic and Faroese, there is a unique subject position. Only subjects are able to occur immediately after the verb in cases of topicalization and yes-no questions:
 \bigskip

EVIDENCE FOR SUBJECT POSITION IN ICELANDIC (TOPICALIZATION;YES-NO QUESTIONS)
\bigskip

In Icelandic, the theme is able to fill this position in passives of ditransitives:
\bigskip

EXAMPLE OF THEME PASSIVE IN ICELANDIC
\bigskip

In English, theme passives are also grammatical. For most of the history of English this has been possible both with and without `to' occurring before the recipient. In modern American English, `to' is required, which will be discussed later as part of the argument for what causes the locality violation.

\begin{exe}
\ex The books were given (to) John.
\end{exe}

This same situation obtains in the mainland Scandinavian languages where there is variation as to whether or not the equivalent of `to' is obligatory \citep{Falk.1990,Sprouse.1995}.
\bigskip

EXAMPLES FROM DANISH, NORWEGIAN, and SWEDISH
\bigskip

In German and Dutch, there is no obligatory subject movement, however, finite verbs show agreement with certain arguments that receive nominative case (overtly marked in German; usually covert in Dutch). In the standard passive of ditransitives, only the theme is able to receive nominative case and trigger verbal agreement.
\bigskip

EXAMPLE OF THEME PASSIVE WITH RAISING
\bigskip

The locality violation can be even more clearly seen here, since the arguments can stay in their base generated positions with the theme still receiving nominative case. The following example relies on the assumption that \textit{nicht} `not' is a marker of the left edge of the VP and uses a second clause to insure sentence and not constituent negation.

\begin{exe}
\ex \gll Gestern wurde nicht einem Mann ein Buch gegeben, SONDERN EINER FRAU EIN FILM GESCHENKT \\
yesterday was.3sg not a.DAT man a.NOM book given, but a.DAT woman a.NOM film sent\\
\trans `Yesterday, a book was not given to a man, but a film was sent to a woman.'
\end{exe}


\section{Existence of Case Sensitivity}
Before demonstrating that theme passives exist, I discuss what is meant by the subject of a passive and demonstrate that the notion is sensitive to cases properties. One of the major discoveries of the generative grammar program was the epiphenominal nature of many traditional grammatical primitives. One of these primitives was the notion of subject. Starting in the 1980s, the notion of VP-internal subjects (CITATIONS) began to break down the notion of subject into dissociable properties. The two main properties that I will focus on are: (1) unique subject position and (2) case/subject agreement. 

\cite{Zaenen.1985} show that in Icelandic these two subject properties are dissociated. Namely, Icelandic allows for oblique subjects with nominative objects, which can trigger (partial) subject agreement on the verb. The Icelandic subject position can be filled by oblique case marked elements. This can happen with (1) certain psych-verbs (e.g. like) and (2) passives of verbs with dative (or genitive) objects. The following examples show how dative marked recipients in ditransitive passives can exist in the unique subject positions:
\bigskip

EXAMPLES OF OBLIQUE SUBJECTS IN ICELANDIC
\bigskip

However, these oblique subjects do not trigger subject agreement on the verb nor do they receive nominative case. Instead, the theme, which receives accusative case in the active, receives nominative case in the passive. As for subject agreement, either default agreement (i.e. 3sg) occurs or the verb agrees in number (but not person) with the nominative object. In other words, the nominative case assignment/agreement properties of the verb show a sensitivity to the case that potential subjects receive in the active.

Further evidence for this sensitivity comes from German. In German, generally only arguments that receive accusative case in the active can receive nominative case in the active.
\bigskip

EXAMPLE OF GRAMMATICAL VS UNGRAMMATICAL WERDEN PASSIVES IN GERMAN
\bigskip

This sensitivity to active case seems to be located on some higher head in the projection, given that different auxiliaries show different sensitivities. In German, passives are typically formed with the auxiliary \textit{werden} `become'. However, in ditransitive passives, two other auxiliaries are possible \textit{kreigen} `get' and \textit{bekommen} `receive'. These auxiliaries allow for recipients to receive nominative case when used as passive auxiliaries. The mechanics of this will be discussed later, but the point here is that the sensitivity to active case seems to be modulated by the same argument that receives subject agreement (namely the auxiliary).
\bigskip

EXAMPLE OF KREIGEN PASSIVES
\bigskip

In Swedish, monomorphemic verbs (i.e. those without a particle prefix) have been reported to \textbf{only} be grammatical in the theme passive. Given the evidence given above, it seems probable that the case properties of the recipient and theme in the active (even though they are not morphologically marked) drive this restriction in Swedish as well.

A final piece of evidence for case sensitivity in passives comes not from patterns of grammaticality, but from patterns of use. In both, Modern British English (esp. 1600-1800) and Modern Faroese (CITATION ABOUT FAROESE PASSIVES) both recipient and theme passives are grammatical. However, these two constructions are not used at the same rate. Instead, the theme passive is much more common than the recipient passive, again suggesting that 
\bigskip

GRAPH OF RELATIVE PASSIVIZATION RATES IN BRITISH ENGLISH
\bigskip

In all of these cases, the theme is being preferred to the recipient for receiving nominative case and triggering agreement. In at least some of these languages (e.g. Swedish, Faroese, and British English), the agreement preference also influences movement to subject position. A final piece of evidence that this preference is triggered by the active case of the objects comes from the verb \textit{lehren} `teach' in German, which takes two accusative arguments in the active.
\bigskip

EXAMPLE OF ACTIVE LEHREN; UNGRAMMATICALITY OF DATIVE RECIPIENT
\bigskip

In the passive, unlike other ditransitive verbs in German, the recipient receives nominative case, even when the auxiliary \textit{werden} is used. 
\bigskip

EXAMPLE OF PASSIVE LEHREN WITH NOMINATIVE RECIPIENT
\bigskip

Finally, the theme can receive nominative case only if the recipient receives dative case (contrary to the active case).
\bigskip

EXAMPLE OF PASSIVE LEHREN; GRAMMATICAL WITH NOMINATIVE THEME ONLY WITH DATIVE RECIPIENT
\bigskip

Thus, it is the relative case marking of the theme and the recipient that drives the choice of subject in German. The generalization seems to be that movement of dative elements to subject position is generally dispreferred and that assignment of nominative case to them and having them trigger subject agreement is even rarer.

\section{Causal Relationship}


Show that locality is not obviated in previously suggested ways:
1) A-scrambling of the theme
	-No A-scrambling in Icelandic
	-American English (Loss of active bare recipients before passive)
2) Clitic Doubling/"Movement"
	-May play a role (higher rates of bare recipient theme passives with pronouns in English)
	-However, robust existence of non-pronominal recipients argues that this isn't the full story

Show examples of how the system works (different types of subject position lead to different passivisation patterns)
 German double accusatives -> dative--accusative in passive
	

\section{Further Implications}
One further question that can be asked is what implications does the claim about the role of subject case sensitivity on locality have outside of ditransitive passives. I suggest that this claim also has implication for the syntax of ergative clauses. While I do not have the space to full explore these implcations, I will provide an outline of this line of argumentation.

There has been a long literature discussing the difference between morphological and syntactic ergativity (CITATIONS). Morphological ergativity refers to the pattern of case marking on arguments. The normal case of morphological ergativity has the same case marker on the logical subjects of intransitives and the logical objects of transitives. Syntactic ergativity is when this shared case marked object moves to subject position in both transitives and intransitives.

This split between case marking and subject movement echoes the data from ditransitive passives discussed above. Also the split between monotransitives and ditransitives is reflected in the passivization of clauses with oblique objects in Faroerese. In Faroese, monotransitive clauses with obdative objects show 


\section{Relativised Locality and T-Argument Interactions}\label{sec:RelLoc}
\subsection{Theoretical Background}
\subsubsection{Subject Positions}\label{sec:subjpos}
\cite{McCloskey.1997} describes how one of the major innovations of the generative program was to remove subjecthood as a primitive notion, instead associating the properties of subjecthood with a structural position. After the late 80s, and the development of the VP-internal subject hypothesis, properties of subjecthood were split, some being derived from the subjects base position within the lower part of the clause, and some being derived by movement to the higher subject position. For the purposes of dealing with passivisation, only this higher position is relevant, since the objects are presumably base generated in their object positions. In V2 clauses, this position is either immediately before or after the finite verb depending on whether there is another element in the preverbal position.
\begin{exe}
\ex Danish Examples:
\begin{xlist}
\ex Initial Subject:
\gll Jens købte en bil {i går}\\
Jens bought a car yesterday\\
\trans `Jens bought a car yesterday \citep[575--576]{Allan.1995}.'
\ex Postverbal subject:
\gll {I går} købte Jens en bil\\
Yesterday bought Jens a car\\
\trans `Yesterday, Jens bought a car \citep[575--576]{Allan.1995}.' 
\end{xlist}
\end{exe}

Note that there is no principled reason within this framework for assuming that all sentences (or all languages) have this higher subject position filled. In such languages, movement to the higher subject position (in the specifier of T) is not obligatory (or even impossible). \cite{Besten.1990} argues for Standard German and Dutch that they have no subject position. All arguments occur in situ, unless they scramble out of the verb phrase to adjoin to something in the T or undergo A-bar movement to the C domain. There would be no subject raising in passives, since there is no subject position to move to. The strong version of this hypothesis (i.e. that there are no subject positions at all in these languages) is not essential to my argument. For some of the arguments below, however, it will be crucial that the higher subject position is not obligatorily filled (i.e. that object DPs \textbf{can} be licensed in their theta positions in the passive).

\subsubsection{Passive Asymmetries}\label{sec:PassiveAsym}
This dissertation will propose an analysis that combines features both from case-based approaches \citep{Larson.1988,Baker.1988,Pesetsky.1996,Holmberg.2001} and from locality-based approaches to passive asymmetry \citep{Falk.1990,Holmberg.1995,McGinnis.1998,Anagnostopoulou.2003}, a property it shares with the analysis in \cite{Platzack.2005}. Case based theories assume that only non-inherently case marked elements (or direct objects instead of indirect objects) are able to enter into a relationship with T. The strongest version of case-based theories is impossible given the possibility of oblique subjects, which has led to a general rise in prominence of locality-based theories.

Locality-based theories state that only the highest DP can enter into a relationship with T, assuming that only the highest argument (i.e. usually the recipient) should be visible to the finite verb. There are three ways that this locality constraint could be circumvented. First, the theme could move across the recipient before the finite verb begins its search (A-scrambling). Second, the recipient could move out of the way of the theme via $\bar{\text{A}}$ movement (see \cite{Anagnostopoulou.2003} on clitic doubling and A-bar scrambling), assuming that DPs that have undergone A-bar movement are invisible for A-movement probes. Finally, this dissertation will argue that the locality condition can be relativized to only search for DPs that have not received inherent case. Assuming that dative case is inherent, the dative-case marked recipient becomes invisible to the search mechanism (see \cite{Rizzi.1990} for the idea of relativized locality, and \cite{Chomsky.2001} for an implementation using a Probe/Goal mechanism). Empirically, the target of the search is either certain active external arguments (e.g. agents and causers) or themes. In \autoref{sec:Case}, I explore analyses of case which allow these elements to be unified, by either having received no case at all, or having received structural accusative case.

Relativized locality is a property of the operations that establish the relationship between T and nominal arguments. Since the relativization is sensitive to case marking, only nominative case assignment/agreement can be independently relativized. Often, however, T does not split the relativizing properties of its nominative case assignment and subject movement operations, which means that subject movement is \emph{also} relativized. Having the case assignment/agreement operation relativized, but the subject raising operation unrelativized would license oblique subjects with nominative objects. Having both relationships relativized creates a situation where only themes are able to passivize. In Modern American English neither operation is relativized, and thus the highest argument always raises, triggers agreement, and for pronouns shows nominative case morphology. As predicted, I know of no language where case assignment/agreement is not relativized, but subject raising is. Such a language would have non-nominative theme in subject position (spec-TP) and agreement with a (nominative) recipient object.

\subsection{Oblique Subjects}\label{sec:ObliqueSubj}
Oblique subjects in ditransitive passives arise from having unrelativized subject raising. \cite{Zaenen.1985} gives the classic presentation of the evidence in Modern Icelandic for oblique subjects. In Icelandic, only subjects can occupy the post-finite verb position.

\begin{exe}
\ex Topicalization
\begin{xlist}
\ex \gll Refinn skaut Ólafur með  þessari byssu.\\
fox.DEF.ACC shot Olaf.NOM with this shotgun\\
\trans `The fox, Olaf shot with this shotgun \citep[ex. 19a]{Zaenen.1985}.'
\ex[*]{\gll Með  þessari byssu skaut refinn Ólafur.\\
with this shotgun shot fox.DEF.ACC Olaf.NOM\\
\trans `The fox, Olaf shot with this shotgun \citep[ex. 19b]{Zaenen.1985}.'}
\end{xlist}
\ex Direct Question
\begin{xlist}
\ex \gll Hafði Sigga aldrei hjálpað Haraldi?\\
had Sigga.NOM never helped Harald.DAT\\
\trans `Had Sigga never helped Harald \citep[ex. 20b]{Zaenen.1985}?'
\ex[*]{\gll Hafði Haraldi Sigga aldrei hjálpað?\\
had Harald.DAT Sigga.NOM never helped\\
\trans `Had Sigga never helped Harald \citep[ex. 20c]{Zaenen.1985}?'}
\end{xlist}
\end{exe}

In cases of ditransitive passives, the dative phrase is capable of filling this position patterning with undisputed subjects.

\begin{exe}
\ex Topicalization
\begin{xlist}
\ex \gll Um veturinn voru konunginum gefnar amb\'{a}ttir.\\
In winter.the were king.the.DAT given slaves.NOM\\
\trans `In the winter the king was given slaves \citep[ex. 47a]{Zaenen.1985}.'
\ex \gll Um veturinn voru amb\'{a}ttin gefin konunginum.\\
In winter.the were slave.NOM given king.the.DAT\\
\trans `In the winter slaves were given to the king \citep[ex. 47b]{Zaenen.1985}.'
\end{xlist}
\ex Direct Question
\begin{xlist}
\ex \gll Voru konunginum gefnar amb\'{a}ttir?\\
were king.the.DAT given slaves.NOM\\
\trans `Was the king given slaves \citep[ex. 48a]{Zaenen.1985}?'
\ex \gll Var amb\'{a}ttin gefnar konunginum?\\
were slave.NOM given king.the.DAT\\
\trans `Was a slave given to the king \citep[ex. 48b]{Zaenen.1985}?'
\end{xlist}
\end{exe}

Note that in all cases with dative subjects, the verb obligatorily agrees with the nominative object in number only \citep{Arnadottir.2013}. Thus, the finite verb must enter into a relationship with both of the object DPs: the recipient in order to trigger movement to the subject position, and the object in order to assign nominative case and to trigger verbal agreement. The weaker nature of agreement without movement is seen in other languages with both preverbal and postverbal positions for the agreement triggering phrase (e.g. Arabic subject agreement).

\subsection{Theme Passivization Across the Recipient}
\subsubsection{Languages without A-scrambling}
As discussed above, three of the modern Germanic languages (Modern Icelandic, Faroese, and Yiddish) lack A-scrambling. This was seen by their fixed word order in the active, where the recipient obligatorily precedes the theme within the verb phrase. Outside of the verb phrase, A-bar movement operations (e.g. wh-movement, heavy DP shift, topicalisation, weak pronoun movement) are able to change the order. Yet, each language still shows evidence for theme passivisation. 

\begin{exe}
\ex Icelandic
\begin{xlist}
\ex \gll J\'{o}ni var gefin b\'{o}kin.\\
John.DAT was given book.the.NOM\\
\trans `John was given the book \citep{Holmberg.1995,Bardal.2001}.'
\ex \gll B\'{o}kin var gefin J\'{o}ni.\\
book.the.NOM was given John.DAT\\
\trans `The book was given to John \citep{Holmberg.1995,Bardal.2001}.'
\end{xlist}
\ex Faroese 
\begin{xlist}
\ex \gll Kúgvin varð seld bóndanum\\
cow.DEF.NOM was sold.NOM farmer.DEF.DAT\\
\trans `The cow was sold to the farmer \citep[ex. 103]{Barnes.1986}.'
\ex \gll Ein blýantur varð givin henni\\
a pencil.NOM was given.NOM her.DAT\\
\trans `A pencil was given to her \citep[ex. 104]{Barnes.1986}.' 
\end{xlist}
\ex Yiddish
\begin{xlist}
\ex \gll afilu biz a halber meluka zal dir gegebn wern\\
even {up to} a half country.NOM should you.DAT given be\\
\trans `Even up to half a country should be given to you.' Translation of Ester 5:3 from Corpus of Modern Yiddish \citep{Bezrukov.2014}
\end{xlist}
\end{exe}

Since the recipient occurs in its base generated position, and there is no evidence of clitic doubling, only two possibilities remain (out of the three possibilities outlined in \autoref{sec:PassiveAsym}). One possibility is that A-scrambling is restricted to the passive, with the scrambled position available only as an intermediate landing site that cannot be filled on the surface (see \cite[119ff]{Richards.2001} for arguments in favor of obligatorily covert movement). The possibility argued for here is that theme passivization occurs across the intervening recipient because of relativized locality.

\subsubsection{Agreement Across the Recipient}
Both High German and Dutch\footnote{See \autoref{sec:A-scramDativShift} for a discussion of A-scrambling and prepositional dative marking.} have A-scrambling, as seen by the possibility of having both the recipient--theme and theme--recipient word orders within the verb phrase \citep{McGinnis.1998}. 
\begin{exe}
\ex High German
\begin{xlist}
\ex \gll  dann hat die Frau dem Jungen das Buch gegeben\\
then has the woman.NOM the boy.DAT the book.ACC given \\
`then the woman gave the boy the book \citep[ex 1a]{Czepluch.1990}, \citep[20a]{Choi.1996}'
\ex \gll dann hat die Frau das Buch dem Jungen gegeben\\
then has the woman.NOM the book.ACC the boy.DAT given\\
`then the woman gave the book to the boy \citep[ex 1b]{Czepluch.1990}, \citep[20b]{Choi.1996}'
\end{xlist}
\ex Dutch
\begin{xlist}
\ex \gll Ik heb (aan) Jan een boek gegeven.\\
I have (to) John a book given\\
\trans `I gave John a book \citep[ex. 1a]{vanBelle.1996b}.
\ex \gll Ik heb een boek $^{*}$(aan) Jan gegeven.\\
I have a book $^{*}$(to) John given\\
\trans `I gave a book to John \citep[ex. 1a]{vanBelle.1996b}.
\end{xlist}
\end{exe}

As a result, theoretically any movement of the theme to subject position could be fed by A-scrambling. However, both German and Dutch lack obligatory subject raising and allow nominals to be licensed in their theta positions \citep{DenDikken.1995}. In passive ditransitive clauses T is able to see past the unmoved recipient in order to agree with the theme and assign nominative case. This can be seen in the following clauses, where the verb shows agreement with the theme and not with the recipient, even in the recipient--theme order.\footnote{Ideally, I would use examples with vP-adverbs, so that we can be sure that neither of the objects have scrambled. While the discussion of such data in the literature implies that examples with adverbs would be grammatical, no such examples are given. In the dissertation, I will make sure to obtain judgments from High German and Dutch speakers confirming the expected judgments.}

\begin{exe}
\ex High German
\begin{xlist}
\ex \gll Ich glaube, dass den Kindern das Fahrrad geschenkt worden ist.\\
I beleive that the.DAT.PL children the.NOM bicycle granted become be.3sg\\
\trans `I believe that the child was granted the bicycle.'
\end{xlist}
\ex Dutch
\begin{xlist}
\ex \gll Er werd mij een boek gegeven.\\
There became.3sg me a book given\\
\trans `A book was given to me. \cite[pg 245]{Donaldson.2008}'
\end{xlist}
\end{exe}



\part{Introduction}
\chapter{Introduction}

The goal of this work is to provide a summary of the possible grammars in the Northwest Germanic languages (not Gothic) vis-a-vis recipient arguments. There are three major points of variation amongst the languages: (1) type and distribution of recipient marking, (2) linearisation of the recipient with the theme in active sentences, and (3) passivisation. The goal is to provide both a syncronic discription of the range of possible grammars, of both the modern standard Germanic languages (as well as some relevant dialect variation) as well as relevant distinct historical stages. When earlier stages of the languages differ from the modern forms, I will show what different pathways of change have occured. 

Previous work has compared different sub-groups of the modern Germanic languages and between Germanic languages and languages from other families \citep[and others]{Falk.1990, Holmberg.1995, Sprouse.1995, Weerman.1997, Holmberg.1998, Primus.1998, Anagnostopoulou.2003, McFadden.2004, Platzack.2005, Bardal.2006, Heine.2010, Alexiadou.2013, Johannessen.2013, Haddican.2014}; studied the history of a particular Germanic language or language sub-family \citep[and others]{Burridge.1993, Kristoffersen.1994, Kiparsky.1997, Allen.1999, Bardal.2001b, McFadden.2002, Hoskuldurrainsson.2004, Sigursson.2012}; studied the ditransitive construction within a particular language or comparison between dialects (see citations in the individual language sections below). 

This work, as far as I know, is the first to bring all of these pieces of research together, in order to see what generalisations hold over the language family as a whole. By focusing on a particular language family, it is possible to rely on their large set of common features to see how small changes in one area of the grammar effect change in another aspect. This seems to me to be the closest that we can come to the experimental paradigm within historical-comparative linguistics. In order to facilitate comparison, and to avoid prejudicing the study one way or the other, I have focused on a semantic object as the domain of interest, namely recipients. This is under the assumption that while the syntactic and morphological realisation of this role may vary from language to language, the concept is one that all of the languages studied need to have some means to express.  This also gives a construction and morphology independent way to refer to the different arguments that will be used throughout the text: \textbf{agent}, \textbf{recipient} and \textbf{theme}.

For my purposes, I will be focusing on recipients narrowly defined, i.e. recipient arguments of verbs for which the argument is unambiguously a recipient, i.e. the new possessor at the end of a transfer action excluding verbs of motion. This means that out of the list of seven types of recipient introducing verbs discussed in \cite{Hovav.2008}, only (\ref{give}) and (\ref{promise}) will be focused on here. These verbs have an agent, who was the previous possessor of the theme, the theme, which is the object to be transferred, and the recipient, who is the new possessor after the event (e.g. ``The agent gave the recipient the theme''). However, with less studied languages/dialects, where the only data about recipients occurs with some other verbs, data from other categories will be included. 

\begin{exe}
\ex
\begin{xlist}
\ex\label{give} Verbs that inherently signify acts of giving: give, hand, lend, loan, pass, rent sell, ...
\ex \label{promise} Verbs of future having: allocate, allow bequeath, grant, offer, owe, promise, ...
\ex \label{tell} Verbs of communication: tell, show, ask, teach, read, write, quote, cite, ...
\ex \label{send} Verbs of sending (\emph{send}-type verbs): forward, mail, send, ship, ...
\ex \label{throw} Verbs of instantaneous causation of ballistic motion (\emph{throw}-type verbs): fling, flip, kick, lob, slap, shoot, throw, toss, ...
\ex \label{bring} Verbs of causation of accompanied motion in a deictically specified direction: bring, take
\ex \label{fax} Verbs of instrument of communication: e-mail, fax, radio, wire, telegraph, telephone, ...
\end{xlist}
\end{exe}

I give a summary of the types of parameters that I have found to vary among the Germanic languages, starting with types of recipient marking (\autoref{sec:marking}), followed by active word order (\autoref{sec:actwo}), and finally a discussion of passivisation (\autoref{sec:paswo}). This is followed by an introduction to Northwest Germanic languages,  and an in-depth discussion of the data from each language. When there has been a major change in the use of recipients in the recorded history of the language, each distinct historical stage gets its own treatment. Dialect variation on the other hand is treated along side the description of the standard languages (\autoref{part:langg}). The goal is for each language section to stand independently, so that someone interested in a particular language can find all the relevant information in that language's section. This will necessitate repeating information for closely related languages. In \autoref{Conc}, I conclude by drawing together generalisation about the range of possible grammars (\autoref{chap:synchgen}), followed by a discussion of diachronic connections between the different grammar types (\autoref{chap:diagen}). Next, I discuss the theoretical implications of these generalisations. Finally, I collate the remaining questions, whose answers are needed to fill in holes in various language/dialect parameter settings and open theoretical questions (\autoref{chap:furtherquest}). 

\chapter{Parameters}

\section{Recipient Marking}\label{sec:marking}
All of the Germanic languages have at least a subjective (nominative) vs. objective (accusative and dative) system among pronouns. Among some of the languages, there is a distinct dative case that recipients can be marked with. Other of the languages have a prepositional marking system (e.g. English ``to''), which can be used to differentiate between the two objects. Most of the languages, thus, have three different possible markings that a recipient can take in any construction: subjective marking (e.g. nominative case), objective marking (e.g. accusative case), and recipient marking (e.g. dative case or prepositional marking). For each language, I lay out a brief description of the case system for that language, and whether or not it has prepositional marking. In the next two sections, I provide a number of different configurations (which consist of combinations of particular properties, such as the pronominal vs nominal status of the arguments). For each of these configurations, it is possible for recipient marking to vary, which will be discussed in the data sections for each particular language.

\section{Active Word Order}\label{sec:actwo}
The main parameter here is the relative order of the theme and recipient. The standard word order for all the Germanic languages is recipient--theme (e.g. ``I gave John the book.'') The parameter is whether or not the language allows the theme--recipient word order (e.g. ``I gave the book to John.'') This parameter intersects with two other properties: (1) the nominal vs. pronominal status of the recipient and theme and (2) whether or not scrambling or object shift has taken place (i.e. whether or not the objects are still within the verb phrase). This leads to 16 possible post-subject configurations (\ref{tab:activeorders}, in which this parameter could potentially vary.

\begin{table}[h!]
\begin{tabular}{ccccccc}
&&& \multicolumn{4}{c}{Recipient}\\
&&& \multicolumn{2}{c}{Noun} & \multicolumn{2}{c}{Pronoun}\\
&&& Scrambled & Not Scrambled & Scrambled & Not Scrambled\\
{\multirow{4}{*}{Theme}}&{\multirow{2}{*}{Noun}}&Scrambled&RT or TR&RT or TR&RT or TR&RT or TR\\
&&Not Scrambled&RT or TR&RT or TR&RT or TR&RT or TR\\
&{\multirow{2}{*}{Pronoun}}&Scrambled&RT or TR&RT or TR&RT or TR&RT or TR\\
&&Not Scrambled&RT or TR&RT or TR&RT or TR&RT or TR\\
\end{tabular}
\caption{Possible active configurations (RT = `Recipient--Theme, TR = `Theme--Recipient'}
\label{tab:activeorders}
\end{table}

 I am ignoring pre-subject configurations (e.g. topicalisation, ``Books, I gave John'' and wh-movement ``Who did I give the books to?''), as well as rightward dislocation, which is an A' process that moves objects to the right of VP material (e.g. heavy NP shift ``I gave the books yesterday to the man that I met last summer''), because all of these constructions involve further movement of the objects whose pre-movement order is difficult to reconstruct.

\section{Passivisation and Recipient Subjects}\label{sec:paswo}
For the purposes of this work, I will be adopting an inclusive definition of passivisation, namely: a passive sentence is a sentence with a marked form of the verb, in which the agent is optional, but it still implicit if not expressed. With passivisation, the major parameter is usually given as which object raises to subject position, which \cite{Allen.1999} gave the titles of theme and recipient passives. I suggest that there are two other possible passivisation configurations. The first is an expletive passive, where neither argument raises to subject position, but instead an expletive pronoun fills the subject position. Finally, \cite{McCloskey.1997} describes how one of the major innovations of the generative program was to remove subjecthood as a primative notion, instead associating the properties of subjecthood with a structural position. After the late 80s, and the development of the VP-internal subject hypothesis, properties of subjecthood were split, some being derived from the subjects base position within the lower part of the clause, and some being derived by movement to the higher subject position. For the purposes of dealing with passivisation, only this higher position is relevant, since the objects are presumably base generated in their object positions. Note that there is no principled reason within such a framework for assuming that all sentences have subjects, or that all languages have the higher subject position. In such a language, there is no notion of subject, which would thus create a subjectless passive. For any language (or historical stage of a language), there needs to be some evidence that subjects exist; only then can it be sensibly asked which argument can become the subject in the passive. This means that there are four possible settings for this parameter. In languages without subjects, only the subjectless passive is possible. In the other languages, a particular configuration can either allow recipient, theme, expletive passive, or some combination of them all.

When discussing the individual languages, I will describe what properties are being associated with a subject position. Note that this generative notion of subject requires that these properties be correlated with a particular position for subjects. In V2 clauses, this position is either immediately before or after the finite verb depending on whether there is another element in the preverbal position. Two major properties that have been associated with subjects that are certainly not subject properties according to this restricted definition: nominative case and verb agreement. \cite{Bardal.2000} makes this methodological point, namely that after the discovery of oblique subjects and nominative objects in Icelandic (see \autoref{sec:Icelandic} below for details), it was no longer possible to assume that neither nominative case nor the associated verbal agreement can be viewed as properties of subjects alone. Thus, while case marking patterns with recipients and themes in ditransitive passives will be discussed, they will not be viewed as giving evidence to the subject vs. non-subject status of arguments.

As with the active orders, the nominal vs. pronominal status of both the theme and recipient are relevant for this parameter as well. Also relevant for some of the languages the type of passive verbal marking is relelvant, either by having different passive auxilaries (be/become vs. get), or by having synthetic passive marking instead of a passive auxiliary (Scandinavian -s- passive). Even in subjectless languages, the arguments can receive different case/prepositional marking in the different configurations, which will be addressed in the data for the individual languages. Combining all the possible variables means that there are a maximum of twelve different configurations as seen in \autoref{tab:pasorders}. 

\begin{table}[h!]
\begin{tabular}{ccccc}
& \multicolumn{2}{c}{Recipient Noun} & \multicolumn{2}{c}{Recipient Pronoun}\\
& Theme Noun & Theme Pronoun & Theme Noun & Theme Pronoun\\
Be/become auxiliary & RP, TP, EP or SP & RP, TP, EP or SP & RP, TP, EP or SP & RP, TP, EP or SP\\
Get auxiliary & RP, TP, EP or SP & RP, TP, EP or SP & RP, TP, EP or SP & RP, TP, EP or SP\\
-s- passive & RP, TP, EP or SP & RP, TP, EP or SP & RP, TP, EP or SP & RP, TP, EP or SP\\
\end{tabular}
\caption{Possible passive configurations (RP = `Recipient Passive, TP = `Theme Passive', EP = `Expletive Passive`, SP = `Subjectless Passive }
\label{tab:activeorders}
\end{table}

With respect to `get', \cite{Broekhuis.1994} argues that in at least some of the Germanic languages, `get' can act as a passive auxiliary. When that happens, `get' co-occurs with a passive participle, which heads the verb phrase and introduces the nominal arguments. This is different from what they call the ``undative'' meaning of `get'. `Get' as a verb introduces a recipient as its subject and a theme as its object. \cite{Broekhuis.1994} suggest that in both the `undative' and passive uses of `get', the recipient subject starts out low and raises to subject position. My focus in this work is on the use of `get' as a passive auxiliary, and how the construction it introduces differs from the other passive constructions.

\part{Language Data}\label{part:langg}
\chapter{Introduction to Northwest Germanic}
Northwest Germanic consists of the modern Germanic languages and their ancestors, to the exclusion of East Germanic (of which Gothic is the only daughter language for which any substantial evidence remains). The major divide in the family is between North Germanic (Scandinavian) and West Germanic. Among the North Germanic languages, historically there was a split between Norse in the west, which was the ancestor of Icelandic, Faroese and Norwegian, and East Scandinavian, which was the ancestor of Danish and Swedish. By the Middle Ages, however, the division between the Insular Scandinavian languages on the one hand (Icelandic and Faroese) and the Mainland Scandinavian languages on the other (Norwegian, Swedish and Danish) became much more relevant. Among the West Germanic group, the first major historical division is between the West Germanic group proper (the ancestor of Dutch, Afrikaans, High German and Yiddish), and the Ingvaeonic languages (the ancestors of Frisian, Low German, and Old English). Again, more recently the difference between Insular West Germanic (i.e. English) and the mainland languages has become more pronounced. Also Dutch (and its daughter Afrikaans) have begun to pattern more with the continental Ingvaeonic languages (i.e. Frisian and Low German). Both Frisian and Low German, on the other hand, have undergone strong sociolinguistic pressure from both Dutch, High German (and to a smaller extent Danish).

For each of the language family below, the structure of the description is the same. I start with older forms of the languages, and then move towards the modern period, when significantly different stages can be identified. Within each stage of the language, I start with an introduction to that stage, describing its location in both time and space, and describing the extant data, either in the form of written corpora, or current speakers. I then discuss the case marking system of the language, as well as the possibility of using prepositional marking with recipients. Then I go through the possible Active configurations and discuss what word order and case/prepositional marking patterns are permitted in that configuration, or demonstrate that the configuration itself is unattested/ungrammatical. Finally, I discuss the passive, starting with a discussion of subject properties (or lack thereof) in the language, and followed by a description of the parameter settings for all of the possible passive configurations (which will include both `'be/become'' and `'get'' passives in all the languages, but -s- passives only in Scandinavian).
\chapter{Scandinavian}
\section{Old Scandinavian}\label{sec:OldScand}

\subsection{Introduction}
Old Scandinavian is the set of dialects spoken in the Scandinavian nations before 1500. There are four distinct dialects, although they are often described as being fairly homogenous with respect to syntax. The most studied dialect is Old Norse, which consists of documents written in Iceland. Closely related to Old Norse is Old Norwegian. More divergent are the Eastern languages Old Swedish and Old Danish.
\subsection{Case and Preposition Marking}
Most of the Old Scandinavian dialects still maintained the distinction between nominative, accusative and dative case in both full noun phrases and pronouns. On the other hand, most Old Danish dialects had already lost synthetic case marking in noun phrases by the earliest texts (circa 1350). CHECK ON OLD DANISH PRONOUN CASE SYSTEM. There is no evidence that any of the Old Scandinavian languages had prepositional marking of recipients.

\subsection{Active Word Orders}
In the active, both word orders are found, although in Old Norse (as represented by ICEPAHC), there is a change in word order throughout the Old Norse period, where the theme-recipient word order becomes less available. The only configuration in which the recipient-theme word order does not become more common is when the recipient is a noun and the theme is a pronoun, where the recipient-theme order occurs on average  50\% of the time throughout the Old Scandinavian period. For the remaining configurations, during the last century of the Old Scandinavian period, the recipient-theme order occurs 92.8571\% of the time.

\begin{knitrout}
\definecolor{shadecolor}{rgb}{0.969, 0.969, 0.969}\color{fgcolor}\begin{figure}[p!]


{\centering \includegraphics[width=\linewidth]{figure/lang-oiact-graph} 

}

\caption[LOESS lines for Old Scandinavian active sentence from IcePaHC]{LOESS lines for Old Scandinavian active sentence from IcePaHC.\label{fig:oldscanactgraph}\label{fig:oiact-graph}}
\end{figure}


\end{knitrout}

\subsection{Passive}
\subsubsection{Subjecthood}
\cite{Bardal.1997,Bardal.1998,Bardal.2000,Bardal.2001} argues that Old Scandinavian shows the same constructions as Modern Icelandic, and therefore should be considered to have oblique subjects (Her focus is on experiencer verbs, insead of recipient verbs). \cite{Rognvaldsson.1991}, again about dative experiencers, states: ``I have not found a signle case where inverted oblique subject-like NPs follow the main verb, they always immediately follow the finite verb.'' This seems to indicate that in Old Scandinavian, as in Modern Icelandic, the standard V2 subject position exists (i.e. immediately before the finite verb, if there is no fronted element, and immediately after the finite verb if some element has been fronted). However, I have found at least one example in IcePaHC where there is nothing in the subject position. The following (passive) example has a fronted temporal adverb, and then nothing in between the finite verb and the main verb (i.e. in the subject position). Both of the objects are presumably still in-situ in the verb phrase.
\begin{exe}
\ex \gll Nú er sagt konungi dráp ármannsins\\
now was said king.DAT killing.NOM/ACC Herman.GEN.his.GEN\\
\trans `Now the king was told of the killing of Herman. (IcePaHC, 1275.MORKIN.NAR-HIS,.667)'
\end{exe}
Far more examples, however, have a nominal element which does occur in one of the subject positions. 
\begin{exe}
\ex
\begin{xlist}
\ex \gll Málróf er gefið mörgum\\
Málróf is given much\\
\trans `Málróf is given much.'
\ex \gll en annan dag jóla var hann í jörð lagður\\
and the day Christmas was he in earth laid\\
\trans `and on the day after Christmas he was laid in the earth. (IcePaHC, 1210.THORLAKUR.REL-SAG,.489)'  
\end{xlist}
\end{exe}
 \cite{Kinn.2010} suggests, discussing Old Norwegian, that in the early period, the subject position existed, but was not obligatorily filled. This seems to correspond to the situation in IcePaHC, where most clauses have subjects, but some small number do not.
\subsubsection{Be/Become Passive}
There are not enough passive tokens in IcePaHC to see any historical trajectory. However, when collating the data from the entire Old Icelandic period, it can be seen that recipient passivisation seems to have been the preferred strategy across the board. This is especially true when the recipient is a pronoun, but there seems to be strong tendencies towards recipient passivisation even with full noun phrase recipients. The theme empty category includes both wh-moved elements and pro-dropped elements, either under pragmatic conditions or under conjunction.

% latex table generated in R 3.1.0 by xtable 1.7-3 package
% Mon Sep 01 13:03:19 2014
\begin{table}[ht]
\centering
\begin{tabular}{rlll}
  \hline
 & Theme Empty & Theme Noun & Theme Pronoun \\ 
  \hline
Recipient Noun & 0\% (3) & 76.2\% (21) & 66.7\% (3) \\ 
  Recipient Pronoun & 100\% (13) & 90.5\% (21) & 100\% (4) \\ 
   \hline
\end{tabular}
\caption{Passive Recipient Passivisation Rates in Old Icelandic (Total Token Numbers in Parentheses)} 
\end{table}



\subsubsection{Get Passive}
I have no information about `get'-passives in Old Scandinavian, however, based on Modern Icelandic, it is likely that the languaged lacked `get' as a passive auxiliary.
\subsubsection{-s- Passive}
The -s- morpheme, which is historically a reduced form of the anaphoric pronoun (e.g. German \emph{sich} `himself') had a middle function in Old Icelandic.

\section{Modern Icelandic}\label{sec:Icelandic}
\subsection{Introduction}
Spoken in Iceland from 1500.
\subsection{Case and Preposition Marking}
Icelandic has distinct nominative, dative and accusative cases for both nouns and pronouns:
\begin{exe}
\ex Noun phrases:
 \gll Pétur gaf konunginum ambáttina.\\
Peter.NOM gave king.DEF.DAT maid-servant.DEF.ACC\\
\trans `Peter gave the king the maid-servant.'
\ex Pronouns:
\gll Hann sýndi henní hana oft\\
he.NOM showed her.DAT it.ACC often\\
\trans `He often showed her it.'
\end{exe}
Preposition marking with \emph{til} `to' is restricted to a small set of recipients, for which animacy seems to be relevant. (\ref{ex:icetil}).
\begin{exe}
\ex \label{ex:icetil} \cite{Ottosson.1993} 
\begin{xlist} 
\ex[*] {\gll Jón gaf bókina til Maríu\\
John gave book.DEF to Mary\\
\trans `John gave the book to Mary.'}
\ex \gll Jón gaf bókasafn sitt til háskólans\\
John gave library {his own} to university.DEF\\
\trans `John donated his own library to the university.'
\end{xlist}
\end{exe}
\subsection{Active Word Order}
\subsubsection{Both Nouns}
Looking at cases where both the recipient and the theme are full noun phrases, there is some disagreement about the possible word orders. Some scholars have reported that both word orders are allowed, in cases where both the theme and recipient are animate (\ref{ice:bothanimateact}). \cite{Dehe.2004} found that her naive Icelandic informants reported the theme-recipient word order degraded/ungrammatical with almost all variations (including IO focus and shared animacy). 
\begin{exe}
\ex\label{ice:bothanimateact}
\begin{xlist}
\ex \gll Hann gaf konunginum ambáttina.\\
He.NOM gave king.DEF.DAT maid-servant.DEF.ACC\\
\trans `He gave the king the maid-servant. \citep[ex 14a]{Dehe.2004}
\ex[?*] {\gll Hann gaf ambáttinakonunginum.\\
He.NOM gave maid-servant.DEF.ACC king.DEF.DAT \\
\trans `He gave the king the maid-servant. \citep[ex 14b]{Dehe.2004}}
\end{xlist}
\end{exe}
For some speakers the theme-recipient order was improved by making the recipient heavy:
\begin{exe}
\ex
\begin{xlist}
\ex \gll Stefán gaf Hildi sem býr á Akureyri sítrónuna. \\
Stefan gave H.DAT Rel-part(who) is from Akureyri lemon.DEF.ACC\\
\trans `Stefan gave Hildi who is from Akureyri the lemon. \citep[ex 16b]{Dehe.2004}'
\ex[?] {\gll Stefán gaf sítrónuna Hildi sem býr á Akureyri \\
Stefan gave lemon.DEF.ACC H.-DAT Rel-part(who) is from Akureyri \\
\trans `Stefan gave the lemon to Hildi who is from Akureyri. \citep[ex 16a]{Dehe.2004}'}
\end{xlist}
\end{exe}
When it comes to object shift, an operation that moves some objects to the left of VP level adverbs as well as negation, either object can move, but only the recipient--theme word order is permitted.
\begin{exe}
\ex \cite[ex. 47]{Vikner.1989}
\begin{xlist}
\ex \gll Pétur sýndi oft Mariú bokína\\
Pétur showed often Mary.DAT book.DEF.ACC\\
\trans `Pétur often showed Mary the book.'
\ex[]{\gll Pétur sýndi Mariú oft bokína\\
Pétur showed Mary.DAT often book.DEF.ACC\\
\trans `Pétur often showed Mary the book.'}
\ex[]{\gll Pétur sýndi Mariú bokína oft\\
Pétur showed Mary.DAT book.DEF.ACC often\\
\trans `Pétur often showed Mary the book.'}
\ex[*]{\gll Pétur sýndi oft bokína Mariú\\
Pétur showed often book.DEF.ACC Mary.DAT\\
\trans `Pétur often showed the book to Mary.'}
\ex[*]{\gll Pétur sýndi bokína oft Mariú\\
Pétur showed book.DEF.ACC often Mary.DAT\\
\trans `Pétur often showed the book to Mary.'}
\ex[*]{\gll Pétur sýndi bokína Mariú oft\\
Pétur showed book.DEF.ACC Mary.DAT often\\
\trans `Pétur often showed the book to Mary.'}
\end{xlist}
\end{exe}
\subsubsection{Recipient Pronoun}
When the recipient is a pronoun, but the theme is a noun, the same general word order trend remains, except that it is odd for the pronoun not to undergo object shift. The theme still cannot precede the recipient.
\begin{exe}
\ex \cite[ex. 48]{Vikner.1989}
\begin{xlist}
\ex[??]{\gll Pétur sýndi oft henní bokína\\
Pétur showed often her.DAT book.DEF.ACC\\
\trans `Pétur often showed her the book.'}
\ex[]{\gll Pétur sýndi henní oft bokína\\
Pétur showed her.DAT often book.DEF.ACC\\
\trans `Pétur often showed her the book.'}
\ex[]{\gll Pétur sýndi henní bokína oft\\
Pétur showed her.DAT book.DEF.ACC often\\
\trans `Pétur often showed her the book.'}
\ex[*]{\gll Pétur sýndi oft bokína henní\\
Pétur showed often book.DEF.ACC her.DAT\\
\trans `Pétur often showed the book to her.'}
\ex[*]{\gll Pétur sýndi bokína oft henní\\
Pétur showed book.DEF.ACC often her.DAT\\
\trans `Pétur often showed the book to her.'}
\ex[*]{\gll Pétur sýndi bokína henní oft\\
Pétur showed book.DEF.ACC her.DAT often\\
\trans `Pétur often showed the book to her.'}
\end{xlist}
\end{exe}
\subsubsection{Theme Pronoun}
When the theme is a pronoun, and the recipient is a noun phrase, only one word order is possible, namely both objects must undergo object shift. The theme needs to undergo object shift since it is a pronoun, and since the theme cannot precede the recipient, the recipient must also undergo object shift so that it will still precede the theme.
\begin{exe}
\ex \cite{Vikner.1989}
\begin{xlist}
\ex[*]{\gll Pétur sýndi oft Mariú hana\\
Pétur showed often Mary.DAT it.ACC\\
\trans `Pétur often showed Mary it.'}
\ex[*]{\gll Pétur sýndi Mariú oft hana\\
Pétur showed Mary.DAT often it.ACC\\
\trans `Pétur often showed Mary it.'}
\ex[]{\gll Pétur sýndi Mariú hana oft\\
Pétur showed Mary.DAT it.ACC often\\
\trans `Pétur often showed Mary it.'}
\ex[*]{\gll Pétur sýndi oft hana Mariú\\
Pétur showed often it.ACC Mary.DAT\\
\trans `Pétur often showed it to Mary.'}
\ex[*]{\gll Pétur sýndi hana oft Mariú\\
Pétur showed it.ACC often Mary.DAT\\
\trans `Pétur often showed it to Mary.'}
\ex[*]{\gll Pétur sýndi hana Mariú oft\\
Pétur showed it.ACC Mary.DAT often\\
\trans `Pétur often showed it to Mary.'}
\end{xlist}
\end{exe}
\subsubsection{Both Pronouns}
When both objects are pronouns, the same restrictions applied as when only the theme was a pronoun, since the theme wants to object shift, and the recipient must object shift as well. In this case, the recipient independently would need to undergo object shift, since it is itself a pronoun.
\begin{exe}
\ex \cite[ex 50]{Vikner.1989}
\begin{xlist}
\ex[*]{\gll Pétur sýndi oft henní hana\\
Pétur showed often her.DAT it.ACC\\
\trans `Pétur often showed her it.'}
\ex[*]{\gll Pétur sýndi henní oft hana\\
Pétur showed her.DAT often it.ACC\\
\trans `Pétur often showed her it.'}
\ex[]{\gll Pétur sýndi henní hana oft\\
Pétur showed her.DAT it.ACC often\\
\trans `Pétur often showed her it.'}
\ex[*]{\gll Pétur sýndi oft hana henní\\
Pétur showed often it.ACC her.DAT\\
\trans `Pétur often showed it to her.'}
\ex[*]{\gll Pétur sýndi hana oft henní\\
Pétur showed it.ACC often her.DAT\\
\trans `Pétur often showed it to her.'}
\ex[*]{\gll Pétur sýndi hana henní oft\\
Pétur showed it.ACC her.DAT often\\
\trans `Pétur often showed it to her.'}
\end{xlist}
\end{exe}
\subsubsection{Summary}
The above examples show that the recipient is always marked with dative case in the active. With respect to the word order parameter, already in Early Modern Icelandic (c. 1500) the recipient-theme word order was predominant (at a rate of $\sim$90.678\%) in almost all conditions. The final environment, where the theme-recipient word order remained a viable possibility was where the recipient was a full noun phrase and the theme a pronoun. Starting around 1800 this condition began to change, such that by the present day it joined all the other conditions in requiring the recipient-theme word order (see \ref{fig:modscanactgraph}).

\begin{knitrout}
\definecolor{shadecolor}{rgb}{0.969, 0.969, 0.969}\color{fgcolor}\begin{figure}[p!]


{\centering \includegraphics[width=\linewidth]{figure/lang-miact-graph} 

}

\caption[LOESS lines for Modern Icelandic active sentence from IcePaHC]{LOESS lines for Modern Icelandic active sentence from IcePaHC.\label{fig:modscanactgraph}\label{fig:miact-graph}}
\end{figure}


\end{knitrout}

\subsection{Passive}
\subsubsection{Subjecthood}
\cite{Zaenen.1985} gives the classic presentation of the evidence in Modern Icelandic for oblique subjects. I can be seen below that there are two positions that are subject positions, which both the recipient and theme can fill in the passive. The first is directly after the raising predicate (e.g. `believe'). The second is the post-finite verb position. Both of these positions can be filled only by the subject in clauses with uncontested subjects.
\begin{exe}
\ex 
\begin{xlist}
\ex Raising
\begin{xlist}
\ex \gll \'{E}g tel konunginum hafa veri\dh gefnar amb\'{a}ttir.\\
I believe king.the.DAT have been given slaves.NOM.\\
\trans `I beleive the king to have been given slaves \citep{Zaenen.1985}.'
\ex \gll \'{E}g tel amb\'{a}ttir hafa veri\dh gefnar konunginum.\\
I believe slaves.NOM have been given king.the.DAT.\\
\trans `I beleive the king to have been given slaves \citep{Zaenen.1985}.'
\end{xlist}
\ex Subject verb inversion
\begin{xlist}
\ex \gll Um veturinn voru konunginum gefnar amb\'{a}ttir.\\
In winter.the were king.the.DAT given slaves.NOM\\
\trans `In the winter the king was given slaves `\citep{Zaenen.1985}.
\ex \gll Um veturinn voru amb\'{a}ttin gefin konunginum.\\
In winter.the were slave.NOM given king.the.DAT\\
\trans `In the winter slaves were given to the king \citep{Zaenen.1985}.'
\ex \gll Voru konunginum gefnar amb\'{a}ttir?\\
were king.the.DAT given slaves.NOM?\\
\trans `Was the king given slaves?`\citep{Zaenen.1985}.
\ex \gll Var amb\'{a}ttin gefnar konunginum?\\
were slave.NOM given king.the.DAT'?\\
\trans `Was a slave given to the king? \citep{Zaenen.1985}.'
\end{xlist}
\end{xlist}
\end{exe}
\subsubsection{Be/Become Passive}
As discussed previously, both theme and recipient passivisation are possible in Modern Icelandic, in passive clauses with `be' or `become' as passive auxiliaries. In both constuctions, the theme is in nominative case and the recipient in dative case. 

\begin{exe}
\ex 
\begin{xlist}
\ex \gll J\'{o}ni var gefin b\'{o}kin.\\
John.DAT was given book.the.NOM\\
\trans `John was given the book \citep{Holmberg.1995,Bardal.2001}.'
\ex \gll B\'{o}kin var gefin J\'{o}ni.\\
book.the.NOM was given John.DAT\\
\trans `The book was given to John \citep{Holmberg.1995,Bardal.2001}.'
\end{xlist}
The nominative argument always shows agreement with the auxiliary \citep[98]{Arnadottir.2013}.
\end{exe}
\begin{exe}
\ex
\begin{xlist}
\ex[] {\gll {Í gaer} voru bílarnir gefið mér\\
Yesterday were.3PL cars.the.M.PL.NOM given.DEF me.DAT\\
\trans `Yesterday, the cars were given to me.' }
\ex[*] {\gll {Í gaer} var mér gefið bílarnir\\
Yesterday was.3SG me.DAT given.DEF cars.the.M.PL.NOM\\
\trans `Yesterday, I was given the cars.'}
\end{xlist}
\end{exe}
Traditionally, expletive passives are not grammatical, although this has changed in recent years.This so-called ``New Passive'', in which both argument remain in their active word order position with the active case marking, has started to be accepted by the youngest generation of speakers. 
\begin{exe}
\ex[\%] {\gll {Í gaer} var mér gefið bílana \\
yesterday was.3SG me.DAT given.DEF cars.the.M.PL.ACC \\
\trans `Yesterday I was given the cars.' }
\end{exe}
Finally, there are a small number of speakers who are accepting two new case marking patterns in the recipient passive. A fairly substantial number of speakers are marking the theme with the accusative case in the recipient passive.
\begin{exe}
\ex[\%] {\gll mér var gefið bílana \\
me.DAT was.3SG given.DEF cars.the.M.PL.ACC \\
\trans `I was given the cars.' }
\end{exe}
 A much smaller number of speakers are marking the recipient with nominative case in the recipient passive.
 \begin{exe}
\ex[\%] {\gll Ég var gefið bílana \\
I.NOM was.3SG given.DEF cars.the.M.PL.ACC \\
\trans `I was given the cars.' }
\end{exe}
While IcePaHC does not have enough tokens to look at the development of the passive in Modern Icelandic, it does provide some information about average rates throughout the Modern Icelandic period. In most conditions, the recipient passive is the predominant form. While the number of tokens is small, the condition with a theme pronoun and a recipient full noun phrase seems to prefer the theme passive. This is consonant with the pattern of change in the active, where in the Early Modern Icelandic period, this condition was the main holdout for the theme-recipient. The theme empty category includes both wh-moved elements and pro-dropped elements, either under pragmatic conditions or under conjunction.

% latex table generated in R 3.1.0 by xtable 1.7-3 package
% Mon Sep 01 13:03:21 2014
\begin{table}[ht]
\centering
\begin{tabular}{rlll}
  \hline
 & Theme Empty & Theme Noun & Theme Pronoun \\ 
  \hline
Recipient Noun & 85.7\% (7) & 81.2\% (16) & 16.7\% (6) \\ 
  Recipient Pronoun & 85\% (20) & 96.4\% (28) & 66.7\% (9) \\ 
   \hline
\end{tabular}
\caption{Passive Recipient Passivisation Rates in Modern Icelandic (Total Token Numbers in Parentheses)} 
\end{table}

\subsubsection{Get Passive}
\cite{Sigursson.2012b} demonstrates that get constructions in Modern Icelandic do not pattern as passives. Unlike be/become passives only the theme object tends to precede the passive participle, which agrees with the theme object in accusative case. Also, only the recipient is capable of being the subject of `get', no matter what word order or case patterns apply. Instead, it appears that the construction is a typical undative construction, with a recipient subject, and a modified object. This leads to the translation of the `get' construction as `Mary got the sent book.' 
\begin{exe}
\ex
\begin{xlist}
\ex \gll Jón sendi Maríu bókina.\\
Jon.Nom sent Maria.DAT book.DEF.ACC\\
\trans `John sent Mary the book.'
\ex \gll Maríu var send bókin.\\
Mary.DAT was sent.PASS.F.SG.NOM book.DEF.F.NOM\\
\trans `Mary was sent the book.'
\ex \gll María fékk bókina senda.\\
Mary.NOM got book.DEF.F.ACC sent.PASS.F.SG.ACC\\
\trans `Mary got the sent book.'
\end{xlist}
\ex
\begin{xlist}
\ex \gll Nú skal konungur fá ambáttina gefna.\\
now shall king.NOM get maid.servant.the.F.ACC given.PASS.F.SG.ACC\\
‘Now the king will get given the female slave.’
\ex[*] {\gll ∗Nú skal ambáttina fá {konungur} gefna {konungur}.\\
now shall servant.the.F.ACC get king.NOM given.PASS.F.SG.ACC king.NOM\\
\trans INTENDED: ‘Now the servant will get given to the king.’}
\ex[*] {\gll ∗Nú skal amb´attin fá (konungur) gefin (konungur). \\
now shall servant.the.F.NOM get king.NOM given.PASS.F.SG.NOM king.NOM \\
\trans INTENDED: ‘Now the servant will get given to the king.’}
\ex[*] {\gll ∗Nú skal ambáttin fá (konunginum) gefin (konunginum) \\
now shall servant.the.F.NOM get king.the.DAT given.PASS.F.SG.NOM king.the.DAT\\
\trans INTENDED: ‘Now the servant will get given to the king.’}
\ex[*] {\gll ∗Nú skal ambáttina fá (konunginum) gefna (konunginum).\\
now shall servant.the.F.ACC get king.the.DAT given.PASS.F.SG.ACC king.the.DAT\\
\trans INTENDED: ‘Now the servant will get given to the king.’}
\end{xlist}
\end{exe}
\subsubsection{-s- Passive}
The -s- passive in Modern Icelandic, which is historically derived from univerbation of the verb with the reflexive pronoun, shows different patterns with respect to dative object in monotransitive and ditransitive clauses. Monotransitive datives become nominative when in the -s- passive, but the ditransitive datives (including recipients) stay dative in the -s- passive. Unlike the be/become passive, the theme remains marked in accusative instead of becoming nominative. However, unlike the be/become passive, there is no implicit agent in these clauses, and introducing an agent with a `by'-phrase is impossible, which may indicate that in Modern Icelandic this form is still a middle construction as opposed to being a true passive.
\begin{exe}
\ex Monotransitive Verbs
\begin{xlist}
\ex \gll Jón splundraði rúðunni.\\
John shattered window.DEF.DAT\\
\trans `John shattered the window.'
\ex \gll rúðan splundraðist.\\
window.DEF.NOM shattered.PASS\\
\trans `The window shattered.'
\end{xlist}
\ex Ditransitive Verbs
\begin{xlist}
\ex \gll Þeir buðu mér peninga.\\
they.NOM offered me.DAT money.ACC\\
\trans `They offered me money.'
\ex \gll Mér buðust peninga.\\
me.DAT offered.ST money.ACC\\
\trans `I got offered money.'
\end{xlist}
\end{exe}

\section{Faroese}\label{sec:Faroese}



\subsection{Introduction}
\subsection{Case and Preposition Marking}
Faroese has distinct nominative, dative and accusative case forms for both nouns and pronouns. \cite{Lundquist.2013b} notes that prepositional marking with \emph{til} `to' is becoming more common in Faroese.

\subsection{Active Word Order}
When both objects are nouns, either the recipient precedes the theme, or it is marked with the preposition \emph{til}.
\begin{exe}
\ex 
\begin{xlist}
\ex \gll Hon gav Mariu troyggiuna.\\
She gave Maria.DAT sweater.THE.ACC.\\
\trans `She gave Maria the sweater \citep{Lundquist.2013b}.'
\ex[*]{\gll Hon gav troyggiuna Mariu.\\
She gave sweater.THE.ACC Maria.DAT.\\
\trans `She gave the sweater to Maria \citep{Lundquist.2013b}.'}
\ex \gll Hon gav troyggiuna til Mariu.\\
She gave sweater.THE.ACC to Maria.DAT.\\
\trans `She gave the sweater to Maria \citep{Lundquist.2013b}.'
\end{xlist}
\end{exe}
This remains true if both of the objects are pronouns. I do not have data for this, but presumably prepositional marking of the dative could be used here as well to get the reverse word order.
\begin{exe}
\ex \gll Turið gav honum t\ae r\\
Turið gave him.DAT them.ACC\\
\trans `Turið gave him them.'
\end{exe}
Only pronouns are able to undergo object shift in Faroese. Unstressed pronouns obligatorily undergo the shift, although I do not have data on how this interacts with ditransitive cases (e.g. what if the theme is an unstressed pronoun, while the recipient is a full noun phrase).
\subsection{Passive}
\subsubsection{Subjecthood}
Faroese has the typical V2 subject position, i.e. before the finite verb, unless there is another fronted element, in which case it occurs immediately after the finite verb. This position is not tied to case, so subjects can be marked nominative, accusative or dative. This pattern is changing in recent years, with most speakers replacing traditional dative experiencer subjects with nominative experiencers. Only subjects are capable of going between the finite verb and any non-finite complement, which makes it possible to distinguish between oblique subjects and fronted oblique objects.
\begin{exe}
\ex \gll Bóndanum vað kúgvin seld\\
farmer.DEF.DAT was cow.DEF.NOM sold\\
\trans `To the farmer, the cow was sold. \cite[ft. 47]{Hoskuldurrainsson.2004}'
\end{exe}
\subsubsection{Be/Become Passive}
\cite[28]{Barnes.1986} states that: "the usual word-order in passivised sentences with an indirect object is: nominative subject + passive verb + indirect object", i.e. theme passivisation is the preferred strategy.
\begin{exe}
\ex 
\begin{xlist}
\ex \gll Kúgvin varð seld bóndanum\\
cow.DEF.NOM was sold.NOM farmer.DEF.DAT\\
\trans `The cow was sold to the farmer.' \citep[ex. 103]{Barnes.1986}
\ex \gll Ein blýantur varð givin henni\\
a pencil.NOM was given.NOM her.DAT\\
\trans `A pencil was given to her.' \citep[ex. 104]{Barnes.1986}
\end{xlist}
\end{exe}
When the theme is indefinite, recipient passivisation is marginally accepted.
\begin{exe}
\ex
\begin{xlist}
\ex[?] {\gll Bóndanum varð seld kúgvin\\
farmer.DEF.DAT was sold.NOM cow.DEF.NOM\\
\trans `The farmer was sold the cow.' \citep[ex. 98]{Barnes.1986}}
\ex[?] {\gll Henni varð givin ein blýantur\\
her.DAT was given.NOM a pencil.NOM\\
\trans `She was given a pencil.' \citep[ex. 99]{Barnes.1986}}
\end{xlist}
\end{exe}
This can be improved by making the theme relatively heavy and conveying new information, however, theme passivisation is still possible in these configurations.
\begin{exe}
\ex \gll Myndugleikunum vórðu givnar allar upplýsingarnar.\\
authorities.DEF.DAT.PL were given.NOM.PL all.NOM.PL information.DEF.NOM.PL\\
\trans `The authorities were given all the information.'
\ex \gll Allar upplýsingarnar vórðu givnar myndugleikunum.\\
all.NOM.PL information.DEF.NOM.PL were given.NOM.PL authorities.DEF.DAT.PL\\
\trans `The authorities were given all the information.'
\end{exe}
When the theme is given (usually definite), then recipient passivisation is severely degraded or ungramatical.
\begin{exe}
\ex 
\begin{xlist}
\ex[??] {\gll Bóndanum varð selt kúgv\\
farmer.DEF.DAT was sold.NOM cow.DEF.ACC\\
`The farmer was sold the cow.' \citep[ex. 100]{Barnes.1986}}
\ex[??] {\gll Henni varð givið ein blýant\\
her.DAT was given.NOM a pencil.ACC\\
\trans `She was given a pencil.' \citep[ex. 101]{Barnes.1986}}
\end{xlist}
\end{exe}
\cite{Eyorsson.2012} reports results of a grammatical survery of native Faroese speakers, looking into possible case marking patterns with recipient passives. Half of the speakers rejected even the canonical case marking pattern (dative recipient, nominative theme).
\begin{exe}
\ex 25.8\% acceptance vs. 50\% rejection\\
\gll Gentuni bleiv givin ein telda.\\
the.girl.DAT was given.NOM a.Nom computer.NOM\\
\trans `The girl was given a computer. \cite[ex 45a]{Eyorsson.2012}'
\end{exe}
More people rejected a version with an accusative theme and a dative subject. This was with an accusative passive participle. It would be interesting to see how subjects would react to a nominative participle (i.e. showing default agreement).
\begin{exe}
\ex 17.7\% acceptance vs. 61.3\% rejection\\
\gll Gentuni bleiv givið eina teldu.\\
the.girl.DAT was given a.ACC computer.ACC\\
\trans `The girl was given a computer. \cite[ex 45b]{Eyorsson.2012}'
\end{exe}
Even fewer people accepted marking the recipient with nominative case, and having it trigger agreement on both the finite verb and the participle. While this was clearly marginal, more than 1 out of 10 speakers still found this construction grammatical.  This may indicate that the change in case marking that has been seen with Faroese experiencers (dative $\rightarrow$ nominative) is also happening with passive recipient subjects. It would be hard to study this in a corpus, however, due to the overwhelming preference for theme passivisation.
\begin{exe}
\ex 14.5\% acceptance vs. 77.4\% rejection\\
\gll Gentan bleiv givin telduna\\
the.girl.NOM was given.NOM the.computer.ACC\\
\trans `The girl was given a computer. \cite[ex 48b]{Eyorsson.2012}'
\ex[*] {\gll Gentan bleiv givin teldan.\\
the.girl.NOM was given.NOM the.computer.NOM\\
\trans `The girl was given a computer. \cite[ex 48a]{Eyorsson.2012}'}
\end{exe}

\cite[ex 41]{Jonsson.2009b} reports results of a survey of Faroese speakers, with respect to expletive passivisation, which was accepted by more than half of the respondents. 
\begin{exe}
\ex 
\begin{xlist}
\ex \gll Það var sýnt þeim bæklinga áður en þau fóru\\
there was shown them.DAT brochures.ACC before they left\\
\trans `They were shown brochures before they left (52% acceptance vs 12% rejection)'
\ex \gll Var þeim ekki {einu sinni} sýnt íbúðina fyrst?\\
was them.DAT not even shown apartment.DEF.ACC first\\
\trans `Were they not even shown the apartment first? (59% acceptance vs 19% rejection)'
\end{xlist}
\end{exe}

\subsubsection{Get Passive}
I do not have any data on the get passive in Faroese.

\subsubsection{-s- Passive}
In Faroese, the -st- form, which is historically derived from the univerbation of the finite verb with the reflexive pronoun, still retains its reflexive meaning. Even in cases where it seems to have a passive function (i.e. no agent and promotion of an object to subject position), it differs from the be/become passive in not suggesting an implicit agent, and prohibiting the introduction of an explicit agent with a `by'-phrase.


\section{Norwegian}\label{sec:Norwegian}


\subsection{Introduction}
There are two different standardised Norwegian languages, bokmål and nynorsk, both of which were standardised as part of the rise of Norwegian nationalism during the 19th century. Part of this nationalism was defined by anti-Danish sentiment, since the Danes had controlled Norway for centuries. Bokmål retains mainly of the Danish features that entered the language during Danish control, while nynorsk was created by looking at the most conservative Norwegian dialects, as well as attempting to predict what Old Norwegian would have looked like without Danish influence.
\subsection{Case and Preposition Marking}
Both standard varieties of Norwegian, as well as most Norwegian dialects no longer have and case marking in full noun phrases, and only maintain a subjective/objective distinction with pronouns. The pronominal objective case is derived historically from the accusative pronominal form. Prepositional marking with \emph{til} `to' does occur. 

The dialect of Halsa will also be discussed here, since it retains distinct dative case forms in the third person singular and plural pronouns. The forms of the accusative and dative 3rd person singular masculine pronoun, which are usually cliticised, can be seen in the examples below \citep{Afarli.2012}.
\begin{exe}
\ex
\begin{xlist}
\ex \gll Ho erta'n\\
she teased.him.ACC\\
\trans `She teased him.'
\ex \gll Ho erta hainn\\
she teased him.ACC\\
\trans `She teased him.'
\end{xlist}
\ex
\begin{xlist}
\ex \gll Ho ga'nå svaret.\\
she gave.him.DAT answer.DEF\\
\trans `She gave him the answer.'
\ex \gll Ho ga hånnå svaret.\\
she gave him.DAT answer.DEF\\
\trans `She gave him the answer.'
\end{xlist}
\end{exe}
There is also a distinction in the nominal inflection system between dative nouns on the one hand, and non-dative nouns on the other (used both in nominative and accusative contexts), when the noun is definite \citep{Afarli.2012}.
\begin{exe}
\ex Definite
\begin{xlist}
\ex \gll Ho erta kattå.\\
she teased cat.DEF.ACC\\
\trans `She teased the cat.'
\ex \gll Ho ga kattåinn mat.\\
she gave cat.DEF.DAT food\\
\trans `She gave the cat food.'
\end{xlist}
\ex Indefinite
\begin{xlist}
\ex \gll Ho erta ei katt.\\
she teased a cat\\
\trans `She teased a cat.'
\ex \gll Ho ga ei katt mat.\\
she gave a cat food\\
\trans `She gave a cat food.'
\end{xlist}
\end{exe}
\subsection{Active Word Order}
\cite{Haddican.2014} ran a study on 500 native Norwegian speakers looking at their possible judgements for cases where both objects were pronouns. He found that the theme-recipient word order was judged unacceptable both in situ and when object shifted.
\begin{exe}
\ex in situ
\begin{xlist}
\ex \gll Elsa ga ikke ham den.\\
Elsa gave not him it.\\
\trans `Elsa did not give him it.'
\ex[*] {\gll Elsa ga ikke den ham.\\
Elsa gave not it him.\\
\trans `Elsa did not give it to him.'}
\end{xlist}
\ex Object Shift
\begin{xlist}
\ex \gll Elsa ga ham den ikke.\\
Elsa gave him it not.\\
\trans `Elsa did not give him it.'
\ex[*] {\gll Elsa ga den ham ikke.\\
Elsa gave it him not.\\
\trans `Elsa did not give it to him.'}\end{xlist}
\end{exe}
It seems that if the recipient is to-marked, then either order is in theory possible, although the recipient-theme order is improved if the theme is heavy (i.e. it looks like the order is derived through Heavy NP movement).
\begin{exe}
\ex
\begin{xlist}
\ex \gll Vi har lånt den interessante boken du nevnte til Petter.\\
we have lent the interesting book you mentioned to Peter\\
\trans `We have lent the interesting book you mentioned to Peter \citep{Larson.1988}'
\ex \gll Vi har lånt til Petter den interessante boken du nevnte.\\
we have lent to Peter the interesting book you mentioned\\
\trans `We have lent to Peter the interesting book you mentioned \citep{Larson.1988}'
\end{xlist}
\end{exe}
In Halsa, I only have examples where the dative marked recipient precedes the theme. I do not know if the word order can be reversed, or if this necessitates prepositional marking.
\begin{exe}
\ex \gll E ga hånnå ei skei.\\
I gave him.DAT a spoon\\
\trans `I gave him a spoon. \cite[ex 50a]{Eyorsson.2012}
\end{exe}
\subsection{Passive}
\subsubsection{Subjecthood}
Norwegian subjects go in the standard V2 subject position, i.e. before the finite verb, unless there is a topicalised phrase, in which case they go after the finite verb. \cite{Kinn.2010} discusses how the subject position became obligatory in Norwegian, as marked by an increase in the use of expletive subjects. He shows that expletive subjects begin to be used around 1450 (i.e. right before the Middle Norwegian period).
\subsubsection{Be/Become Passive}
In Standard Norwegian, all three possible subject types are attested. Either the theme or recipient can raise, or both objects can stay in situ with an expletive subject. With theme passivisation, the recipient can either be null marked, or receive prepositional marking. Both the theme and the recipient, when pronouns, are nominative in subject position, and oblique in object position.
\begin{exe}
\ex 
\begin{xlist}
\ex \gll Soldaten/Han vart gitt ein medalje/ham.\\
Soldier.DEF/he.NOM was given a medal/it.OBL.\\
\trans `The soldier/he was given a medal/it \citep{Afarli.1992,Holmberg.1995}.'
\ex \gll Ein medalje/han vart gitt soldaten/ham.\\
A medal/it.NOM was given soldier.DEF/him.OBL.\\
\trans `A medal/it was given to the soldier/to him \citep{Afarli.1992}.'
\ex \gll Ein medalje vart gitt til soldaten.\\
A medal was given to soldier.DEF.\\
\trans `A medal was given to the soldier \citep{Afarli.1992,Holmberg.1995}.'
\ex \gll Det vart gitt soldaten ein medalje.\\
it was given soldier.DEF a medal.\\
\trans `The soldier was given a medal \citep{Afarli.1992}.'
\end{xlist}
\end{exe}
The same facts hold for Halsa, even though they have overt dative case, i.e. if the recipient becomes the subject, it gets nominative case marking.
\begin{exe}
\ex
\begin{xlist}
\ex[]{\gll Hainn vart gjevinn ei skei.\\
He.NOM was given a spoon\\
\trans `He was given a spoon.' \cite[ex 50c]{Eyorsson.2012}}
\ex[*]{\gll Hånnå vart gjevinn ei skei.\\
He.DAT was given a spoon\\
\trans `He was given a spoon.' \cite[ex 50c]{Eyorsson.2012}}
\end{xlist}
\end{exe}
\subsubsection{Get Passive}
I do not have any information about the get-passive in Norwegian.


\section{Swedish}\label{sec:Swedish}

\subsection{Introduction}
The language of Sweden from 1500 onwards.
\subsection{Case and Preposition Marking}
Standard Swedish does not have any case marking on full noun phrases. Pronouns have both nominative and oblique case forms. Some recipients can also be marked with the preposition \emph{til} `to'.
\subsection{Active Word Order}
With many recipient introducing verbs, both word orders are possible (i.e. recipient-theme and theme-recipient). In the theme-recipient word order, however, the recipient must be marked with the preposition \emph{til} `to'. As far as I can tell, this also applies if either argument is a pronoun, although I have not found that to be explicitly mentioned anywhere.
\begin{exe}
\ex 
\begin{xlist}
\ex \gll Jag gav Johan en bok.\\
I gave John a book.\\
\trans `I gave John a book \citep{Holmberg.1995}.'
\ex[*] {\gll Jag gav en bok Johan.\\
I gave a book John.\\
\trans `I gave a book to John.'}
\ex \gll Jag gav en bok til Johan.\\
I gave a book to John.\\
\trans `I gave a book to John \citep{Holmberg.1995}.'
\end{xlist}
\end{exe}
With other verbs, however, the prepositionally marked variant is infelicitous. \cite{Lundquist.2006} seems to suggest that these verbs are the ones with incorporated preverbs (e.g. \emph{er-bjuda} ``offer''). 
\begin{exe}
\ex
\begin{xlist}
\ex \gll Han erbjöd oss en ny lägenhet.\\
He offered us a new apartment.\\
\trans `He offered us a new apartment \cite{Anward.1989}.'
\ex[\%] {\gll Han erbjöds en ny lägenhet till os.\\
he offered a new apartment to us.\\
\trans `He offered a new apartment to us \citep{Anward.1989}.'}
\end{xlist}
\end{exe}
\subsection{Passive}
\subsubsection{Subjecthood}
\cite{Falk.1997} discusses how expletive subjects began to be used during Early Modern Swedish (1526-1732). The obligatory use of expletive subjects in Modern Swedish is strong evidence that Swedish has an obligatory subject position.
\subsubsection{Be/Become Passive}
The be/become passive is rare in Swedish, having been replaced with the -s- passive.
\subsubsection{Get Passive}
I do not have any information about the get-passive in Swedish.
\subsubsection{-s- Passive}
With most ditransitive verbs in Swedish, only the theme passive is possible, and only when the recipient is marked with \emph{til} `to'. \cite{Anward.1989} describes this class as the class of verbs which participate in the dative alternation (i.e. which allow prepositional marking in the active). \cite{Lundquist.2006} states that these are verbs without an incorporated preverb.
\begin{exe}
\ex 
\begin{xlist}
\ex[\%] {\gll Pelle gavs ett \''{a}pple.\\
 Pelle gave.PASS an apple.\\
\trans `Pelle was given an apple \citep{Anward.1989,Lundquist.2006}.'}
\ex[\%] {\gll Ett \''{a}pple gavs Pelle.\\
 An apple gave.PASS Pelle.\\
\trans `An apple was given to Pelle \citep{Anward.1989,Lundquist.2006}.'}
\ex[\%] {\gll det gavs Pelle en vacker bok.\\
it gave.PAS Pelle a beautiful book.\\
\trans `There was given Pelle a beautiful book. \citep{Anward.1989}}
\ex \gll Ett äpple gavs till Pelle.\\
 an apple gave.PASS to Pelle.\\
\trans `An apple was given to Pelle \citep{Anward.1989,Lundquist.2006}.'
\end{xlist}
\end{exe}
When the verb does have an incorporated preverb, and thus does not license prepositional marking in the active, all three possible subject patterns are found: recipient passivisation, theme passivisation and expletive passivisation.

\begin{exe}
\ex Offer
\begin{xlist}
\ex \gll Han erbj\''{o}ds ett nytt jobb.\\
He.NOM offered.PASS a job.\\
\trans `He was offered a job \citep{Anward.1989,Falk.1990,Lundquist.2006}.'
\ex \gll Ett nytt jobb erbj\''{o}ds honom.\\
A job offered.PASS him.OBL.\\
\trans `A job was offered to him \citep{Anward.1989,Falk.1990,Lundquist.2006}.'
\ex \gll Det erbj\''{o}ds honom ett nytt jobb.\\
There offered.PASS him.OBL a job.\\
\trans `He was offered a job \citep{Anward.1989,Falk.1990,Lundquist.2006}.'
\end{xlist}

The possibility of recipient passivisation, however, is a new element in Swedish grammar. \cite{Falk.1997} shows that recipient passivisation is ``very rare before 1800s, [with] two examples that are older than 1850.'' This is centuries after the loss of morphological case, which was almost complete by the beginning of the Modern Swedish period (c. 1500).

\section{Danish}\label{sec:Danish}
\subsection{Introduction}
The language of Denmark after 1500.
\subsection{Case and Preposition Marking}
Danish has no case marking on full noun phrases, and makes on a nominative vs. oblique distinction between pronouns. This state of affairs already held in most dialects in some of the oldest Old Danish texts (c. 1350). Modern Danish does allow prepositional marking of recipients with \emph{til} `to'.
\subsection{Active Word Order}
Both word orders are permitted in Danish, however, the recipient must be marked with \emph{til} `to' in the theme-recipient order.
\begin{exe}
\ex \gll Jeg gav bogen til Anna.\\
I gave book.the to Anna\\
\trans `I gave the book to Anna. \cite{Holmberg.1998}'
\ex[*]{\gll Jeg gav bogen til Anna.\\
I gave book.the to Anna\\
\trans `I gave the book to Anna. \cite{Holmberg.1998}'}
\ex \gll Jeg gav Anna bogen.\\
I gave Anna book.the\\
\trans `I gave Anna the book. \cite{Holmberg.1998}'
\end{exe}
\subsubsection{Both Nouns}
When both objects are nouns, neither can undergo object shift \citep{Vikner.1989}.
\begin{exe}
\ex 
\begin{xlist}
\ex \gll Peter viste jo Marie bogen\\
Peter showed indeed Mary book.DEF\\
\trans `Peter indeed showed Mary the book.'
\ex[*]{\gll Peter viste Marie jo bogen\\
Peter showed Mary indeed book.DEF\\
\trans `Peter indeed showed Mary the book.'}
\ex[*]{\gll Peter viste Marie bogen jo\\
Peter showed Mary book.DEF indeed\\
\trans `Peter indeed showed Mary the book.'}
\ex[*]{\gll Peter viste jo bogen Marie\\
Peter showed indeed book.DEF Mary\\
\trans `Peter indeed showed the book to Mary.'}
\ex[*]{\gll Peter viste bogen jo Marie\\
Peter showed book.DEF indeed Mary\\
\trans `Peter indeed showed the book to Mary.'}
\ex[*]{\gll Peter viste bogen Marie jo\\
Peter showed book.DEF Mary indeed\\
\trans `Peter indeed showed the book to Mary.'}
\end{xlist}
\end{exe}
\subsubsection{Recipient Pronoun}
When the theme is a noun and the recipient is a pronoun, the recipient undergoes object shift \citep{Vikner.1989}.
\begin{exe}
\ex
\begin{xlist}
\ex[??]{\gll Peter viste jo hende bogen\\
Peter showed indeed her book.DEF\\
\trans `Peter indeed showed her the book.'}
\ex[]{\gll Peter viste hende jo bogen\\
Peter showed her indeed book.DEF\\
\trans `Peter indeed showed her the book.'}
\ex[*]{\gll Peter viste hende bogen jo\\
Peter showed her book.DEF indeed\\
\trans `Peter indeed showed her the book.'}
\ex[*]{\gll Peter viste jo bogen hende\\
Peter showed indeed book.DEF her\\
\trans `Peter indeed showed the book to her.'}
\ex[*]{\gll Peter viste bogen jo hende\\
Peter showed book.DEF indeed her\\
\trans `Peter indeed showed the book to her.'}
\ex[*]{\gll Peter viste bogen hende jo\\
Peter showed book.DEF her indeed\\
\trans `Peter indeed showed the book to her.'}
\end{xlist}
\end{exe}
\subsubsection{Theme Pronoun}
When the theme is a pronoun and the recipient is a noun, there is no acceptable word order [without \emph{til} `to', since the theme needs to object shift and occur after the recipient \citep{Vikner.1989}.
\begin{exe}
\ex
\begin{xlist}
\ex[??]{\gll Peter viste jo Marie den\\
Peter showed indeed Mary it\\
\trans `Peter indeed showed Mary it.'}
\ex[*]{\gll Peter viste Marie jo den\\
Peter showed Mary indeed it\\
\trans `Peter indeed showed Mary it.'}
\ex[??]{\gll Peter viste Marie den jo\\
Peter showed Mary it indeed\\
\trans `Peter indeed showed Mary it.'}
\ex[*]{\gll Peter viste jo den Marie\\
Peter showed indeed it Mary\\
\trans `Peter indeed showed it to Mary.'}
\ex[*]{\gll Peter viste den jo Marie\\
Peter showed it indeed Mary\\
\trans `Peter indeed showed it to Mary.'}
\ex[*]{\gll Peter viste den Marie jo\\
Peter showed it Mary indeed\\
\trans `Peter indeed showed it to Mary.'}
\end{xlist}
\end{exe}
\subsubsection{Both Pronouns}
When both objects are pronouns, both must undergo object shift \citep{Vikner.1989}.
\begin{xlist}
\ex[*]{\gll Peter viste jo hende den\\
Peter showed indeed her it\\
\trans `Peter indeed showed her it.'}
\ex[*]{\gll Peter viste hende jo den\\
Peter showed her indeed it\\
\trans `Peter indeed showed her it.'}
\ex[]{\gll Peter viste hende den jo\\
Peter showed her it indeed\\
\trans `Peter indeed showed her it.'}
\ex[*]{\gll Peter viste jo den hende\\
Peter showed indeed it her\\
\trans `Peter indeed showed it to her.'}
\ex[*]{\gll Peter viste den jo hende\\
Peter showed it indeed her\\
\trans `Peter indeed showed it to her.'}
\ex[*]{\gll Peter viste den hende jo\\
Peter showed it her indeed\\
\trans `Peter indeed showed it to her.'}
\end{xlist}
\end{exe}
\subsection{Passive}
\subsubsection{Subjecthood}
\subsubsection{Be/Become Passive}
Danish allows for all three subject types: recipient, theme and expletive. If the theme raises to subject position, however, the recipient must be marked with the preposition \emph{til} `to'.
\begin{exe}
\ex
\begin{xlist}
\ex \gll Han blev tilbudt en stilling.\\
He.NOM was offered a job.\\
\trans `He was offered a job \citep{Falk.1990}.'
\ex[]{\gll En stilling blev tilbudt til ham.\\
A job was offered to him.OBL.\\
\trans `A job was offered to him \citep{Falk.1990}.'}
\ex[*]{\gll En stilling blev tilbudt ham.\\
A job was offered him.OBL.\\
\trans `A job was offered to him \citep{Falk.1990}.'}
\ex \gll Der blev tilbudt ham en stilling.\\
There was offered him.OBL a job.\\
\trans `He was offered a job \citep{Falk.1990}.'
\end{xlist}
\end{exe}

\subsubsection{Get Passive}
I do not have any information about the get-passive in Danish.
\subsubsection{-s- Passive}
I have found no discussion of a difference between -s- and be/become passives in Danish.

\chapter{High German}
\section{High German}\label{sec:HGerman}
\subsection{Introduction}
High German consists of both Modern Standard German, and the dialects of the southern half of Germany, Austria and Switzerland.
\subsection{Case and Preposition Marking}
German has a somewhat complicated case marking system. In pronouns, there is a distinction between nominative, dative and accusative. For full noun phrases, there are three different positions in the noun phrase. The singular head noun bears some markings that are associated with gender and number, but do not seem to be sufficient on their own to establish case. Plural nouns on the other hand have distinct dative forms. If the noun is modified by an adjective, and does not have an article, then the adjective receives a set of case suffixes that distinguishes nominative, accusative and dative. If there is an article, then the article bears the case information, and the adjective only expresses gender and number agreement. Crucially, this means that in most cases a bare noun, without an article or adjective is not marked for case \citep{Draye.1996}.
\cite{Seiler.2003} describes a number of High German dialects which still maintain dative case marking on nouns (demonstratives/adjectives) and pronouns, but, with recipients, it always co-occurs with prepositional marking. As far as I can tell, these prepositional marked recipient behave in all ways like the dative marked recipients in Standard German.
\subsection{Active Word Order}
The unmarked order is agent-recipient-theme, with only some arguments being allowed to be focused in other orders \citep{Choi.1996}. If the noun would otherwise occur bare (i.e. without an article), that is only permitted in the canonical order (recipient-theme). In the theme-recipient order the difference between bare and article possessing recipients is neutralised, since the article is necessary to bear the case information \citep{Draye.1996}.
\begin{exe}
\ex \cite[162]{Draye.1996}
\begin{xlist}
\ex \gll weil er (der) Unehrlichkeit keine Chance gibt.\\
as he.NOM (the) dishonesty.DAT no opportunity.ACC gives\\
\trans `as he gives dishonesty no opportunity.'
\ex \gll weil er keine Chance *(der) Unehrlichkeit gibt.\\
as he.NOM  no opportunity.ACC *(the) dishonesty.DAT gives\\
\trans `as he gives no opportunity dishonesty.'
\end{xlist}
\end{exe}
Given the proper pragmatics, however, and all six possible combinations of agent, recipient and theme are possible.
\begin{exe}
\ex \gll  dann hat die Frau dem Jungen das Buch gegeben\\
then has the woman.NOM the boy.DAT the book.ACC given \\
`then the woman has given the boy the book \citep[ex 1a]{Czepluch.1990}, \citep[20a]{Choi.1996}'
\ex \gll dann hat die Frau das Buch dem Jungen gegeben\\
then has the woman.NOM the book.ACC the boy.DAT given\\
`then the woman has given the book to the boy \citep[ex 1b]{Czepluch.1990}, \citep[20b]{Choi.1996}'
\ex \gll dann hat dem Jungen die Frau das Buch gegeben \\
then has the boy.DAT the woman.NOM the book.ACC given\\
`then the woman has given the boy the book \citep[ex 1c]{Czepluch.1990}, \citep[20c]{Choi.1996}'
\ex \gll dann hat dem Jungen das Buch die Frau gegeben\\
then has the boy.DAT the book.ACC the woman.NOM given\\
`then the woman has given the boy the book \citep[ex 1d]{Czepluch.1990}, \citep[20e]{Choi.1996}'
\ex \gll dann hat das Buch die Frau dem Jungen gegeben\\
then has the book.ACC the woman.NOM the boy.DAT given\\
`then the woman has given the book to the boy \citep[ex 1e]{Czepluch.1990}, \citep[20d]{Choi.1996}'
\ex \gll dann hat das Buch dem Jungen die Frau gegeben \\
then has the book.ACC the boy.DAT the woman.NOM given \\
`then the woman has given the book to the boy \citep[ex 1f]{Czepluch.1990}, \citep[20f]{Choi.1996}'
\end{exe}
In general, indefinite objects remain in their canonical position. If they scramble out of that position then they are required to take a specific interpretation \citep{Diesing.1992}.
\begin{exe}
\ex Scrambing Possibilities \citep{Choi.1996}: 
\begin{xlist}
\ex \gll Ich habe meinem Bruder einen Brief geschickt.\\
I have my brother.DAT a letter.ACC sent\\
\trans `I have sent my brother a letter.'
\ex[*] {\gll Ich habe einen Brief meinem Bruder geschickt.\\
I have a letter.ACC my brother.DAT sent\\
\trans `I have sent a *(certain) letter to my brother.'}
\end{xlist}
\end{exe}
The two objects can be in either order if the recipient is focused (e.g. as part of the answer to a question with the recipient as the wh-element).
\begin{exe}
\ex IO Focus \citep{Choi.1996}:
\begin{xlist}
\ex \gll Wem hast du das Geld gegeben?\\
whom.DAT have you.NOM the money.ACC given\\
\trans `Who did you give the money to?'
\ex \gll Ich habe dem KASSIERER das Geld gegeben.\\
I.NOM have the cashier.DAT the gold.ACC given.\\
\trans `I have given the cashier the gold.'
\ex \gll Ich habe das Geld dem KASSIERER gegeben.\\
I.NOM have the gold.ACC the cashier.DAT given.\\
\trans `I have given the gold to the cashier.'
\end{xlist}
\end{exe}
On the other hand, if the theme is focused, then it must remain in situ (i.e. the theme cannot scramble if it is focused). 
\begin{exe}
\ex DO Focus:
\begin{xlist}
\ex \gll Was hast du dem Kassierer gegeben?\\
what.ACC have you.NOM the cashier.DAT given\\
\trans `What did you give to the cashier?'
\ex \gll Ich habe dem Kassierer das GELD gegeben.\\
I.NOM have the cashier.DAT the gold.ACC given.\\
\trans `I have given the cashier the gold.'
\ex[?*] {Ich habe das GELD dem Kassierer gegeben.}
I.NOM have the gold.ACC the cashier.DAT given.\\
\trans `I have given the gold to the cashier.'
\end{xlist}
\end{exe}
This is only the case if the theme received focus that was not explicitly contrastive. The focus can be explicitly contrastive, either from a contrastive particle like \emph{NUR} `only' or by mentioning an alternative. In this case the theme is able to scramble.
\begin{exe}
\ex Contrastive Focus:
\begin{xlist}
\ex \gll weil Hans NUR ein BUCH dem Mann gegeben hat\\
because Hans only a book.ACC the man.DAT given has\\
\trans `because Hans only gave a book to the man.
\ex \gll weil Hans ein BUCH dem Mann gegeben hat (nicht eine ZEITUNG)\\
because Hans a book.ACC the man.DAT given has (not a newspaper)\\
\trans `because Hans gave a book to the man, (not a newspaper).
\end{xlist}
\end{exe}

\subsection{Passive}
\subsubsection{Subjecthood}
\cite{Besten.1990} argues that Standard German has no subject position. All arguments occur in situ, unless they scramble out of the verb phrase. This means that there is no raised subject in the passive. In the following sections, I will discuss default word orders (i.e. pre-scrambling word orders) and the case marking patterns of the passive.
\subsubsection{Be/Become Passive}
The standard passive auxiliary in High German is \emph{werden} `become'. The recipient is marked with dative case, and the theme is marked with nominative case. The default word order is recipient theme, which can be seen again based on the possible focus patterns. If the recipient is focused, either word order is possible \citep{Lenerz.1977}:
\begin{exe}
\ex
\begin{xlist}
\ex \gll Wem ist das Fahrrad geschenkt worden?\\
who.DAT is the.ACC bicycle granted become\\
\trans `Who was the bicycle granted to?'
\ex \gll Ich glaube, dass das Fahrrad dem KIND geschenkt worden ist.\\
I beleive that the.ACC bicycle the.DAT child granted become is.\\
\trans `I beleive that the bicycle was granted to the child.'
\ex \gll Ich glaube, dass dem KIND das Fahrrad geschenkt worden ist.\\
I beleive that the.DAT child the.ACC bicycle granted become is.\\
\trans `I beleive that the child was granted the bicycle.'
\end{xlist}
\end{exe}
If the theme is focused, however, it must occur after the recipient, since focus-marked elements are not able to scramble:
\begin{exe}
\ex
\begin{xlist}
\ex \gll Was ist dem Kind geschenkt worden?\\
what.ACC is the.DAT child granted become\\
\trans `What was the child granted?'
\ex \gll Ich glaube, dass dem Kind das FAHRRAD geschenkt worden ist.\\
I beleive that the.DAT child the.ACC bicycle granted become is.\\
\trans `I beleive that the child was granted the bicycle.'
\ex[?*] {\gll Ich glaube, dass das FAHRRAD dem Kind geschenkt worden ist.\\
I beleive that the.ACC bicycle the.DAT child granted become is.\\
\trans `I beleive that the bicycle was granted to the child.'}
\end{xlist}
\end{exe}
\subsubsection{Get Passive}
With \emph{bekommen} `receive' as a passive auxiliary, the recipient is marked with nominative case, and the theme receives accusative case. The default word order is to have the recipient before the theme.
\begin{exe}
\ex dass die Tochter von dem Vater ein Buch geschenkt bekommen hat\\
that the daughter.NOM by the father a book.ACC sent got has\\
\trans `that the daughter got sent a book by her father \cite[183]{Draye.1996}.'
\end{exe}

\section{Yiddish}\label{sec:Yiddish}
\subsection{Introduction}
Starting off as a dialect of Middle Bavarian, Yiddish was the dialect of German spoken by the Jewish communities of Central Europe. After being expelled from much of Central Europe, they moved to Eastern Europe, where Yiddish underwent a great deal of influence from Slavic languages. The standardised language was created based on a number of Eastern dialects during the early part of the 20th century.
\subsection{Case and Preposition Marking}
Case is marked both on pronouns as well as the article, with distinct nominative, accusative and dative cases. I have no evidence of prepositional marking of recipients in Yiddish.
\subsection{Active Word Order}
When a sentence contains both kinds of object, the indirect one precedes the direct one:
\begin{exe}
\ex \gll Zi git der snjjer dus pékl \\
she.NOM gives the.DAT daughter-in-law the.ACC parcel\\
\trans 'She gives her daughter-in-law the parcel' \citep[ex 190a]{Birnbaum.1979}
\end{exe}
When one of the objects is a pronoun it takes precedence, following immediately on the finite part of the verb. When both objects are pronouns the word order is not fixed but the sequence, accusative-dative, is preferred; both pronouns follow immediately on the finite part of the verb:
\begin{exe}
\ex \gll Zi darf ys géibn der snjjer \\
she.NOM must it.ACC give the.DAT daughter-in-law\\
\trans 'She has to give it to her daughter-in-law' \citep[ex 190b]{Birnbaum.1979}
\ex \gll Zi darf ir géibn dus pékl \\
she.NOM must her.DAT give the.ACC parcel\\
\trans 'She must give her the parcel'\citep[ex 190b]{Birnbaum.1979}
\ex \gll Er darf ys ir géibn \\
he.NOM must it.ACC her.DAT give\\
\trans 'He must give it her'\citep[ex 190c]{Birnbaum.1979}
\end{exe}
Objects either remain in the verb phrase, and thus after any non-finite verb, or they scramble out of the vP and thus precede both the non-finite verb and any vP adjunct adverbs (like the negative adverb \emph{nit}).
\begin{exe}
\ex \gll Nekhtn hot Maks nit gegebn dem yingl dos bukh.\\
yesterday had Max.NOM not given the boy.DAT the book.ACC\\
\trans `Yesterday, Mas had not given the boy the book.'
\ex[*] {\gll Nekhtn hot Maks nit dem yingl dos bukh gegebn.\\
yesterday had Max.NOM not the boy.DAT the book.ACC given \\
\trans `Yesterday, Mas had not given the boy the book.'}
\ex \gll Nekhtn hot Maks dem yingl dos bukh nit gegebn.\\
yesterday had Max.NOM the boy.DAT the book.ACC not given\\
\trans `Yesterday, Mas had not given the boy the book. \citep[40a]{Diesing.1997}
\end{exe}
\subsection{Passive}
I have no information of the behaviour of Yiddish in the passive.



\chapter{Low German}

\section{Modern Dutch}\label{sec:ModDutch}
\subsection{Introduction}
\subsection{Case and Preposition Marking}
Dutch lost synthetic case marking around 1600, developing from the Middle Dutch system with distinct nominative dative and accusative cases. The pronoun system still maintains a nominative versus oblique distinction. Recipients can be marked with the preposition \emph{aan} `to'.
\subsection{Active Word Order}
Both the recipient-theme and theme-recipient word orders are possible when both objects are full noun phrases. The recipient must be marked with \emph{aan} `to' in the theme-recipient word order, and may be marked with \emph{aan} `to' in the recipient-theme order \citep{SchermerVermeer.1991,Broekhuis.1994,DenDikken.1995,vanBelle.1996b}.

\begin{exe}
\ex 
\begin{xlist}
\ex \gll Ik heb (aan) Jan een boek gegeven\\
I have (to) Jan a book given\\
\trans `I gave Jan a book.' 
\ex \gll Ik heb een boek *(aan) Jan gegeven\\
I have a book *(to) Jan given\\
\trans `I gave a book to Jan.'
\end{xlist}
\end{exe}
If both objects are clitic pronouns, then the theme precedes recipient:
\begin{exe}
\ex \gll Ik heb 't 'm gegeven.\\
I have it him given\\
\trans `I gave it to him \cite[ex 1c]{vanBelle.1996b}'
\end{exe}
\cite{vanBelle.1996b} describes four factors that favour the IONP over the IOPP, the first two of which are essential, the second two are secondary:
\begin{enumerate}
\item Involvement of participants in the state of affairs
\item Occurance of a transfer in the event
\item Topic-focus structure relative to the discourse
\item The nature of the NPs (+clitic, +pronominal, +definite, +specific,+human)
\end{enumerate}
\nocite{Broekhuis.1994}
\cite[1071]{Broekhuis.2012} describes the role of scrambling as predominantly information structural. According to him, scrambled material is part of the presupposition of the clause, while in situ material is part of the focus of the clause. Note that if the theme is presuppositional, while the recipient is focused, then the theme can scramble only if the recipient receives prepositional marking.
\begin{exe}
\ex
\begin{xlist}
\ex \gll dat Jan waarshijnlijk zijn moeder het boek heeft gegeven.\\
that John probably his mother the book has given\\ 
\trans `that John probably gave his mother the book.' \cite[ex 49a]{Broekhuis.2012}
\ex \gll dat Jan zijn moeder waarshijnlijk het boek heeft gegeven.\\
that John his mother probably the book has given\\
\trans `that John probably gave his mother the book.' \cite[ex 49b]{Broekhuis.2012}
\ex \gll dat Jan zijn moeder het boek waarshijnlijk heeft gegeven. \\
that John his mother the book probably has given\\
\trans `that John probably gave his mother the book.' \cite[ex 49c]{Broekhuis.2012}
\ex[*] {\gll dat Jan het boek waarshijnlijk *(aan) zijn moeder heeft gegeven.\\
that John the book probably *(to) his mother has given\\
\trans `that John probably gave the book to his mother.' \cite[ex 49d]{Broekhuis.2012}}
\end{xlist}
\end{exe}
The same principle applies in reverse for pronouns, where the theme must precede the recipient (without any prepositional marking). I do not know if prepositional marking is simply not required here, or if it is prohibited.
\begin{exe}
\ex
\begin{xlist}
\ex[*] {\gll dat Jan waarschijnlijk haar het heeft gegeven.\\
that John probably her it has given\\
\trans `That John probably gave her it. \cite[ex 50a]{Broekhuis.2012}'}
\ex[*] {\gll dat Jan haar waarschijnlijk het heeft gegeven. \\
that John her probably it has given\\
\trans `That John probably gave her it. \cite[ex 50b]{Broekhuis.2012}'}
\ex[*]{\gll dat Jan haar het waarschijnlijk heeft gegeven. \\
that John her it probably has given \\
\trans `That John probably gave her it. \cite[ex 50c]{Broekhuis.2012}'}
\ex {\gll dat Jan het haar waarschijnlijk heeft gegeven. \\
that John it her probably has given \\
\trans `That John probably gave her it. \cite[ex 50d]{Broekhuis.2012}'}
\end{xlist}
\end{exe}

\subsection{Passive}
\subsubsection{Subjecthood}
\cite{Besten.1990} argues that Dutch does not have a subject position. Instead, all movement out of the vP is scrambling. This means that all Dutch passives have the no subject parameter setting. There are, however, some differences in the case/prepositional marking possibilities in Dutch passives.

\subsubsection{Be/Become Passive}
As in the active, scrambling the theme without scrambling the recipient is not permitted. However, if both objects scramble, the theme is permitted to precede the recipient even if the recipient is not preposition marked. The base order without any scrambling is recipient-theme.
\begin{exe}
\ex[?*] {\gll dat het boek waarschijnlijk Marie gegeven wordt\\
that the book probably Mary given was\\
\trans `that the book was probably given to Mary.' \citep{Anagnostopoulou.2003}}
\ex {\gll dat het boek Marie waarschijnlijk gegeven wordt\\
that the book Mary probably given was\\
\trans `that the book was probably given to Mary.' \citep{Anagnostopoulou.2003}}
\ex {\gll dat waarschijnlijk Marie het boek gegeven wordt\\
that probably Mary the book given was\\
\trans `that the book was probably given to Mary.' \citep{Anagnostopoulou.2003}}
\end{exe}
With pronouns, it can be seen that the recipient is always in the oblique case, when the auxiliary is \emph{werden} `become'. Nominative marking on the recipient is ungrammatical.
\begin{exe}
\ex De boeken werden aan Marie aangeboden. \\
the books were to Marie offered
\trans `The books were offered to Mary.' \citep[ex. 5a]{Broekhuis.1994}
\ex De boeken werden haar aangeboden. \\
the books were her.OBL offered
\trans `The books were offered to her.' \citep[ex. 5b]{Broekhuis.1994}
\ex[*]{\gll Zij werd de boeken aangeboden.\\
she.NOM was the books offered\\
\trans `She was offered the books.' \citep[ex. 5c]{Broekhuis.1994} \citep{Anagnostopoulou.2003}}
\end{exe}
\subsubsection{Get Passive}
When the auxiliary is \emph{krijgen} `get', then the recipient has to be marked with nominative case.
\begin{exe}
\ex \gll Zij kreeg de boeken (van mij) aangeboden.\\
she.NOM got the books (by me) given\\
\trans `She was given the books (by me).' \citep[ex. 7]{Broekhuis.1994}
\end{exe}
Almost all ditransitive verbs can take the get-passive, except for give. \cite{Broekhuis.1994} suggest that this is because the semantics of ``get given'' is identical to simply ``get''. 
\begin{exe}
\ex \gll De boeken werden hem (door mij) gegeven.\\
the books were him (by me) given \\
\trans `The books were given to him (by me).' \citep[ex 11b]{Broekhuis.1994}
\ex[*]{\gll Hij kreeg de boeken (van mij) gegeven. \\
he got the books (from me) given \\
\trans `He got given the books (by me).' \citep[ex 11c]{Broekhuis.1994}}
\ex Hij krijgt het boek van mij.\\
he got the book from me
\trans `He got the book from me.' \citep[ex 12]{Broekhuis.1994}
\end{exe}

\section{Afrikaans}\label{sec:Afrikaans}
\subsection{Introduction}
Afrikaans is a descendant of a number of Early Modern Dutch dialects, which coalesced in Dutch South Africa. There are a number of other substrate influences, including native South African languages, colonial Portugese, and colonial English.
\subsection{Case and Preposition Marking}
\cite{Stadler.1996} states that case gone on full noun phrases, nominative versus oblique case on pronouns. There seem to be two different systems of prepositional marking in Afrikaans. I do not know what the distribution of the systems are, since Afrikaans resources seem to describe either one system or the other. This seems to indicate either dialect variation, or conservative vs. innovative forms.

One system maintains the Dutch prepositional marking with \emph{aan} `to'. The other system has developed a Differential Object Marker out of the preposition \emph{vir} `for'. This marker occurs on most animate objects, both direct and indirect, and (at least for some speakers) has replaced \emph{aan} `to' as the marker for recipients.
\subsection{Active Word Order}
\cite{Stadler.1996} states that both the recipient-theme and theme-recipient word order are possible. When the word order is theme-recipient, all sources report that the recipient must be preposition marked. According to \cite{Stadler.1996}, the word order with a prepositionally marked recipient preceding the theme is possible, but does not give an example.
\begin{exe}
\ex 
\begin{xlist}
\ex \gll Ek het hom-IO `n fooitjie gegee.\\
I have him-IO a tip given\\
\trans `I have given him a tip.'
\ex \gll Ek bet `n fooitjie aan hom-IO gegee.\\
I have a tip to him-IO given\\
\trans `I have given a tip to him.'
\end{xlist}
\end{exe}
There is some disagreement in the literature about the possibilities in subordinate clauses. \cite{Stadler.1996} states that as in matrix clauses, the recipient must be preposition marked if it follows the theme (although the example given includes a pronoun, which may need to precede a full noun phrase theme independently).
\begin{exe}
\ex
\begin{xlist}
\ex \gll ... dat ons jou graag daardie stel boeke skenk.\\
... that we you {with pleasure} that set {of books} gifted\\
\trans `...that we gave you with pleasure that set of books.'
\ex[*]{\gll dat ons daardie stel boeke jou graag skenk.\\
that we that set {of books} you {with pleasure} gifted\\
\trans `...that we gave with pleasure that set of books to you.'}
\end{xlist}
\end{exe}
\cite{Louw.2012}, on the other hand, states that either order of objects is possible, even without prepositional marking. This is only possible, however, in subordinate clauses without V2 word order. This is usually clauses with overt complementisers.
\begin{exe}
\ex
\begin{xlist}
\ex \gll dat die man die vrou `n dokument gegee het\\
that the man the woman a document given has\\
\trans `...that the man gave a document to the woman.'
\ex \gll dat die man `n dokument die vrou gegee het\\
that the man a document the woman given has\\
\trans `...that the man gave a document to the woman \citep{Louw.2012}.'
\end{xlist}
\end{exe}
\cite{Donaldson.1993} states that the default word order in Afrikaans is theme-recipient, with the recipient marked with \emph{vir} `for'. The recipient-theme order, according to him, is highly marked. The recipient without prepositional marking only occurs in idiomatic contexts (e.g. \emph{Ek gee hom 'n klap} `I'll gave him a slap'). The recipient-theme order with prepositional marking occurs when the theme is focused.
\subsection{Passive}
\subsubsection{Subjecthood}
\cite{Stadler.1996} states that only subject in Afrikaans can occur immediately following the finite verb in cases where either the sentence is V1 (e.g. yes/no questions) or where there is a topicalised element.
\subsubsection{Be/Become Passive}
According to \citep{Stadler.1996}, both the recipient or the theme can fill the subject position in the passive. When the recipient raises, it can marginally be unmarked (if a full noun phrase) or receive nominative case (if a pronoun). It is preferred, however, for the recipient to receive prepostional marking.
\begin{exe}
\ex 
\begin{xlist}
\ex[?]{\gll hy is `n present gegee.\\
he was a prenent given\\
\trans `He was given a present .'}
\ex \gll Aan hom is `n present gegee.\\
to him was a present given\\
\trans `He was given a present.'
\end{xlist}
\end{exe}
Even when it has prepositional marking, the recipient patterns as a subject occurring after the finite verb in both V1 constructions and with a topicalised element:
\begin{exe}
\ex
\begin{xlist}
\ex \gll Is aan hom ooit 'n geskenk gegee?\\
Was to him ever a present given.\\
\trans `Was he ever given a present?'
\ex \gll Gister is aan hom `n klomp geld gegee.\\
Yesterday was to him a {lot of} money given.\\
\trans `Yesterday he was given a lot of money.'
\end{xlist}
\end{exe}
When the theme raises to subject position, the recipient is obligatorily preposition marked \citep{Stadler.1996}:
\begin{exe}
\ex
\begin{xlist}
\ex \gll Die pragtige vaas word aan my ma geskenk.\\
the beautiful vase was to my mother gifted\\
\trans `The beautiful vase was given as a gift to my mother.'
\ex[*]{\gll Die pragtige vaas word my ma geskenk.\\
the beautiful vase was my mother gifted\\
\trans `The beautiful vase was given as a gift to my mother.'}
\end{xlist}
\end{exe}
\cite{Ponelis.1979} states that the passive of give is generally dispreferred in favour of using \emph{kry} `get':
\begin{exe}
\ex
\begin{xlist}
\ex[?] {\gll ?Hy is 'n present gegee\\
He is a present given\\
\trans `He was given a present.'}
\ex \gll Hy het 'n present gekry\\
he has a present gotten\\
\trans `He got a present'
\end{xlist}
\end{exe}
\subsubsection{Get Passive}
I have no information about get-passives in Afrikaans, although see above for use of get instead of the be/become passive.

\section{Frisian}\label{sec:Frisian}
\subsection{Introduction}
\subsection{Case and Preposition Marking}
\subsection{Active Word Order}
\cite[105]{Tiersma.1985}: ``Some verbs, of which \emph{jaan} `give' is an example, take two noun objects. Word order is crucial - when two objects follow a verb of the above thype, the first is interpreted as the indirect, and the second as the direct object. If both are pronouns, it is the indirect object which comes after the direct object. The indirect object may also be expressed as a prepositional phrase with \emph{oan} `to'; the order of this prepositional phrase in relation to the direct object is not crucial.''
\begin{exe}
\ex \gll se joech jar kammeraatske in skjirre\\
she gave her girlfriend a {pair of scissors}\\
\trans `She gave her girlfriend a pair of scissors'
\ex \gll ik joech it har\\
I gave it her\\
\trans `I gave it to her.'
\ex
\begin{xlist}
\ex ik joech in plant oan Beppe\\
I gave a plant to Grandmother\\
\trans `I gave a plant to Grandmother.'
\ex ik joech oan Beppe in plant\\
I gave to Grandmother a plant \\
\trans `I gave to Grandmother a plant.'
\end{xlist}
\end{exe}
\subsection{Passive}
\cite[110-111]{Tiersma.1985}: ``In contrast to English, the promotion of the indirect object to surface subject by passivization is not possible in Frisian.
\begin{exe}
\ex \gll de apple wurdt fan Boukje oan har mem jûn\\
the apple was by Boukje to her mom given\\
\trans `The apple was given to her mom by Boukje.'
\end{exe}

\section{Low German}\label{sec:LGerman}
\subsection{Introduction}
Low German was the language of the Hanseatic league, a coalition of city states in northern Germany, which controlled much of the Baltic and North Sea trade. After the protestant reformation, Low German lost its prestige status to High German, which was the language of Luther. Low German has been under linguistic pressure from High German, Dutch and Danish for approximately the last 500 years. It is currently spoken natively by people in the far north of Germany, the eastern part of the Netherlands, and southern Denmark.
\subsection{Case and Preposition Marking}
Almost all Low German dialects have a two case system only in pronouns (nominative vs. non-nominative) \citep{Shrier.1965,Lindow.1998}. This change seems to have already started by some of the earliest Middle Low German texts in the 1300s, where the same text will sometimes use an accusative pronoun and sometimes a dative pronoun for the same construction \citep{Lasch.1914,Boden.1993}. Ultimately, the dative form won out, and most of the Low German dialects have the historical dative form as an oblique form. 

\cite{Fleischer.2006} states: ``In Low German, this construction [prepositional dative marking] could eventually be viewed as compensatory to the loss of a distinct dative case; however, from the fact that I could not find any decisive examples of this construction in Low German, I conclude that it is very rare.'' \cite{Lindow.1998} makes no mention of prepositional dative marking (including in a section discussing the uses of various prepositions.
\subsection{Active Word Order}
\cite{Mussaus.1829} states that Low German often has many of the same word order variants as High German, but uses them more freely than they are typically used in Standard German:
\begin{exe}
\ex
\begin{xlist}
\ex \gll ick gaw den Mann dat Brod\\
I gave the man the bread\\
\trans `I gave the man the bread.'
\ex \gll ick gaw dat Brod den Mann, wobei dat Brod zeigend ist.\\
I gave the bread the man who the bread shown shown is\\
\trans `I gave the bread to the man who was shown the bread.'
\end{xlist}
\end{exe}
\subsection{Passive}
The only information I have about the passive in Low German is that the recipient is never receives nominative case. I do not know whether Low German has a subject position or not, although the evidence from the active about word order freedom would suggest that Low German is a subjectless language.
\begin{exe}
\ex \gll En Gulden w\"o\"or den Bedelmann vun `n Herzog geven.\\
A gilder was the begger by the gentleman given.\\
\trans `A gilder was given to the begger by the gentleman.'
\end{exe}
\cite[4.3.1.3.2]{Lindow.1998} states: ``When a sentence has more objects [than one], a dative object stands in a position before the other object (\ref{lowgermanact}). The word order can be disrupted, when a object is extremely emphasised.''
\begin{exe}
\ex \label{lowgermanact}\gll Denn geev ik ehr dat Huus.\\
Then gave I.NOM him.OBL the house.\\
\trans `Then I gave him the house.'
\end{exe}
 
\chapter{English}\label{sec:English}

\section{Old English}
\subsection{Introduction}
Old English is the set of dialects spoken in England from about 500 C.E. to c. 1200 C.E. While there is a small amount of data from a variety of dialects, the bulk of our data comes from the West Saxon scriptoria. During the 9th and 10th centuries, West Saxon became the predominant English speaking force in England (as opposed to the Danish forces in North England). Most of our manuscripts come from the 10th and 11th centuries, although a number of them are copies of texts that were composed from the 8th century onward.

The center of West Saxon culture was in the western Midlands. This means that the majority of our Old English sources are \emph{not} the direct ancestor of Modern English, which descends mostly from the dialects of East Anglia (which is virtually unattested during the Old English period) and Kent (which is only marginally attested). 
\subsection{Case and Preposition Marking}
West Saxon Old English had distinct nominative, accusative and dative case in both nouns and pronouns. The case system already showed signs of weakening, with a large degree of syncretism between various case forms (esp. with feminine nouns and pronouns). Prepositional marking of recipients is only found in very late texts (e.g 12th century) \citep{McFadden.2002}.
\subsection{Active Word Order}
\cite[48-49]{Allen.1999} found with two noun phrase objects, 75 (54\%) had the order ACC DAT and 64 (46\%) had the order DAT ACC. With an expanded corpus, 17 examples of two pronouns, all but one had the order ACC DAT.

\subsection{Passive}
\subsubsection{Subjecthood}
\cite[50-54]{Allen.1999} finds that Coordinate Subject Deletion (CSD) is a test of subjecthood in Old English, with subjects controlling deletion sometimes, and objects never controlling deletion. With the passives of ditransitives, the nominative argument triggers deletion with both argument orders in 15 out of 22 examples (of conjoined clauses where the subject are co-referential), while the dative argument is never coreferential with a deleted subject (see \autoref{CSDOE}).
\begin{table}[ht!]
\begin{tabular}{lcc}
Co-referential Nominative Subects & Deletion & No Deletion\\
Order NOM DAT & 11 (73\%) & 4 (27\%)\\
Order DAT NOM & 4 (57\%) & 3 (43\%)\\
Total & 15 (68\%) & 7 (32\%) \\
\hline
Co-referential Nominative Subects & & \\
Order NOM DAT & 0 (0\%) & 27 (100\%)\\
Order DAT NOM  & 0 (0\%) & 11 (100\%)\\
Total & 0 (0\%) & 38 (100\%)\\
\end{tabular}
\caption {\cite[Table 2.6]{Allen.1999}} shows her corpus findings on CSD with ditransitive passives in the first clause.
\label{tab:CSDOE}
\end{table}
\subsubsection{Be/Become Passive}
As seen above, the passive of ditransitives in Old English always had the theme marked with nominative case, and the recipient marked with dative case. While both theme-recipient and recipient-theme orders exist, the recipient is never able to control Coordinate Subject Deletion, which indicates that it is always the theme, which is the subject in Old English passives, and the recipient-theme orders reflect topicalisation of the recipient.
\subsubsection{Get Passive}
I have no data on the get passive in Old English.
\section{Middle and Early Modern English}
\subsection{Introduction}
After the Norman Conquest, English data becomes quite scanty, with only a small number of texts coming from the late 12th and 13th centuries. By the 14th century, the number of texts increases, with most of the texts coming from the area around London. From the end of Old English around 1200 till about 1500 will be identified as Middle English.

Early Modern English will refer to the English language as used in England from about 1500 till 1800.
\subsection{Case and Preposition Marking}
\cite[213]{Allen.1999} distinct dative case forms were lost in the North during the 12th century, and survived in the South into the 13th century, mostly on 3rd person masculine pronouns. During this time, the prepositional marker \emph{to} increases in use. \cite{McFadden.2002} shows that there is a correlation in the early period between having distinct case forms and use of prepositional marking.
\subsection{Active Word Order}
\begin{knitrout}
\definecolor{shadecolor}{rgb}{0.969, 0.969, 0.969}\color{fgcolor}\begin{kframe}
\begin{alltt}
\hlkwd{library}\hlstd{(epicalc)}
\end{alltt}


{\ttfamily\noindent\color{warningcolor}{\#\# Warning: package 'epicalc' was built under R version 3.1.1}}

{\ttfamily\noindent\itshape\color{messagecolor}{\#\# Loading required package: foreign\\\#\# Loading required package: survival\\\#\# Loading required package: splines\\\#\# Loading required package: MASS\\\#\# Loading required package: nnet\\\#\# \\\#\# Attaching package: 'epicalc'\\\#\# \\\#\# The following object is masked from 'package:plyr':\\\#\# \\\#\#\ \ \ \  rename}}\begin{alltt}
\hlstd{fullact}\hlopt{$}\hlstd{Year}\hlkwb{<-}\hlstd{(fullact}\hlopt{$}\hlstd{YoC}\hlopt{-}\hlkwd{mean}\hlstd{(fullact}\hlopt{$}\hlstd{YoC))}
\hlstd{fullact}\hlopt{$}\hlstd{Year}\hlkwb{<-}\hlstd{fullact}\hlopt{$}\hlstd{Year}\hlopt{/}\hlkwd{sd}\hlstd{(fullact}\hlopt{$}\hlstd{YoC)}
\hlstd{rntn}\hlkwb{<-}\hlkwd{subset}\hlstd{(fullact,NIO}\hlopt{==}\hlstr{'Recipient Noun'}\hlopt{&}\hlstd{NDO}\hlopt{==}\hlstr{'Theme Noun'}\hlopt{&}\hlstd{YoC}\hlopt{>=}\hlnum{1200}\hlopt{&}\hlstd{YoC}\hlopt{<=}\hlnum{1500}\hlstd{)}
\hlstd{rptn}\hlkwb{<-}\hlkwd{subset}\hlstd{(fullact,NIO}\hlopt{==}\hlstr{'Recipient Pronoun'}\hlopt{&}\hlstd{NDO}\hlopt{==}\hlstr{'Theme Noun'}\hlopt{&}\hlstd{YoC}\hlopt{>=}\hlnum{1200}\hlopt{&}\hlstd{YoC}\hlopt{<=}\hlnum{1500}\hlstd{)}
\hlstd{rntp}\hlkwb{<-}\hlkwd{subset}\hlstd{(fullact,NIO}\hlopt{==}\hlstr{'Recipient Noun'}\hlopt{&}\hlstd{NDO}\hlopt{==}\hlstr{'Theme Pronoun'}\hlopt{&}\hlstd{YoC}\hlopt{>=}\hlnum{1200}\hlopt{&}\hlstd{YoC}\hlopt{<=}\hlnum{1500}\hlstd{)}
\hlstd{rptp}\hlkwb{<-}\hlkwd{subset}\hlstd{(fullact,NIO}\hlopt{==}\hlstr{'Recipient Pronoun'}\hlopt{&}\hlstd{NDO}\hlopt{==}\hlstr{'Theme Pronoun'}\hlopt{&}\hlstd{YoC}\hlopt{>=}\hlnum{1200}\hlopt{&}\hlstd{YoC}\hlopt{<=}\hlnum{1500}\hlstd{)}
\hlkwd{lrtest}\hlstd{(}\hlkwd{glm}\hlstd{(}\hlkwc{data}\hlstd{=rntn,OrdVal}\hlopt{~}\hlstd{Year,}\hlkwc{family}\hlstd{=binomial),}\hlkwd{glm}\hlstd{(}\hlkwc{data}\hlstd{=rntn,OrdVal}\hlopt{~}\hlnum{1}\hlstd{,}\hlkwc{family}\hlstd{=binomial))}
\end{alltt}
\begin{verbatim}
## Likelihood ratio test for MLE method 
## Chi-squared 1 d.f. =  2.124 , P value =  0.145
\end{verbatim}
\begin{alltt}
\hlkwd{lrtest}\hlstd{(}\hlkwd{glm}\hlstd{(}\hlkwc{data}\hlstd{=rptn,OrdVal}\hlopt{~}\hlstd{Year,}\hlkwc{family}\hlstd{=binomial),}\hlkwd{glm}\hlstd{(}\hlkwc{data}\hlstd{=rptn,OrdVal}\hlopt{~}\hlnum{1}\hlstd{,}\hlkwc{family}\hlstd{=binomial))}
\end{alltt}
\begin{verbatim}
## Likelihood ratio test for MLE method 
## Chi-squared 1 d.f. =  0.00797 , P value =  0.9289
\end{verbatim}
\begin{alltt}
\hlkwd{lrtest}\hlstd{(}\hlkwd{glm}\hlstd{(}\hlkwc{data}\hlstd{=rntp,OrdVal}\hlopt{~}\hlstd{Year,}\hlkwc{family}\hlstd{=binomial),}\hlkwd{glm}\hlstd{(}\hlkwc{data}\hlstd{=rntp,OrdVal}\hlopt{~}\hlnum{1}\hlstd{,}\hlkwc{family}\hlstd{=binomial))}
\end{alltt}
\begin{verbatim}
## Likelihood ratio test for MLE method 
## Chi-squared 1 d.f. =  1.102 , P value =  0.2939
\end{verbatim}
\begin{alltt}
\hlkwd{lrtest}\hlstd{(}\hlkwd{glm}\hlstd{(}\hlkwc{data}\hlstd{=rptp,OrdVal}\hlopt{~}\hlstd{Year,}\hlkwc{family}\hlstd{=binomial),}\hlkwd{glm}\hlstd{(}\hlkwc{data}\hlstd{=rptp,OrdVal}\hlopt{~}\hlnum{1}\hlstd{,}\hlkwc{family}\hlstd{=binomial))}
\end{alltt}
\begin{verbatim}
## Likelihood ratio test for MLE method 
## Chi-squared 1 d.f. =  4.771 , P value =  0.02895
\end{verbatim}
\end{kframe}
\end{knitrout}
\begin{knitrout}
\definecolor{shadecolor}{rgb}{0.969, 0.969, 0.969}\color{fgcolor}\begin{kframe}


{\ttfamily\noindent\color{warningcolor}{\#\# Warning: Removed 1 rows containing missing values (stat\_smooth).\\\#\# Warning: Removed 1 rows containing missing values (stat\_smooth).\\\#\# Warning: Removed 1 rows containing missing values (geom\_point).\\\#\# Warning: Removed 1 rows containing missing values (geom\_point).}}\end{kframe}\begin{figure}[p!]


{\centering \includegraphics[width=\linewidth]{figure/lang-meactord-graph} 

}

\caption[LOESS lines for Middle and Early Modern active sentence from the English Parsed Corpora]{LOESS lines for Middle and Early Modern active sentence from the English Parsed Corpora.\label{fig:mengactgraph}\label{fig:meactord-graph}}
\end{figure}


\end{knitrout}

\begin{knitrout}
\definecolor{shadecolor}{rgb}{0.969, 0.969, 0.969}\color{fgcolor}\begin{figure}[p!]


{\centering \includegraphics[width=\linewidth]{figure/lang-meactnounpp-graph} 

}

\caption[LOESS lines for Middle and Early Modern active sentence with recipient nouns from the English Parsed Corpora]{LOESS lines for Middle and Early Modern active sentence with recipient nouns from the English Parsed Corpora.\label{fig:mengactgraph}\label{fig:meactnounpp-graph}}
\end{figure}


\end{knitrout}

\begin{knitrout}
\definecolor{shadecolor}{rgb}{0.969, 0.969, 0.969}\color{fgcolor}\begin{figure}[p!]


{\centering \includegraphics[width=\linewidth]{figure/lang-meactpropp-graph} 

}

\caption[LOESS lines for Middle and Early Modern active sentence with recipient pronouns from the English Parsed Corpora]{LOESS lines for Middle and Early Modern active sentence with recipient pronouns from the English Parsed Corpora.\label{fig:mengactgraph}\label{fig:meactpropp-graph}}
\end{figure}


\end{knitrout}
\subsection{Passive}
\subsubsection{Subjecthood}
\subsubsection{Be/Become Passive}
\subsubsection{Get Passive}
\cite[383-385]{Allen.1999} states that the dative fronted passive was lost in the middle of the 14th century,``although impersonal passives in which the indirect object is a prepositional phrase are common.'' The earliest clear example of a recipient passive (with a nominative recipient) is in 1375.
\section{Modern English}
\subsection{Introduction}
\subsection{Case and Preposition Marking}
\subsection{Active Word Order}
\cite{Collins.1995,Bresnan.2007,Hollmann.2007,Bresnan.2009} show a number of different pragmatic constraints on the use of the English dative alternation, where given, pronominal, animate material tends to occur to the left of focused, heavy, inanimate material.

\cite{Gast.2007}:
\begin{exe}
\ex I gave a book the man. Certain northern dialects
\end{exe}
\subsection{Passive}
\subsubsection{Subjecthood}
\subsubsection{Be/Become Passive}
\subsubsection{Get Passive}

\part{Conclusions and Further Questions}\label{Conc}
\chapter{Synchronic Generalisations}\label{chap:synchgen}
\section{Recipient Marking}
There are three different grammars when it comes to the distribution of recipient marking. Some languages, namely all the Old Germanic language, Icelandic, Faroese, Yiddish and High German dialects, mark all of their recipients. Other languages, of which Low German is the only Germanic example, mark none of their recipients. Finally, many of the languages (i.e. Norwegian, Swedish, Danish, Standard High German, Dutch, Afrikaans, Frisian, and English) have a mixed system, where some recipients are marked and others are not. 

There have been a number of claims about the correlation between word order possibilities and recipient marking with case (e.g. \cite{Weerman.1997}), namely that more marking leads to freer word order. Only a very weak version of this claim is tenable when looking at Germanic. Among the languages with obligatory marking, there are both languages with freer word order (e.g. the Older Germanic Languages), as well as languages with more fixed word order (Icelandic, Faroese and Yiddish). On the other hand, the one language with no recipient marking (Low German) has quite free word order. 

The generalisation does hold, however, among the languages with mixed recipient marking. These systems jave two different ways that a recipient can occur: a marked form, which is distinct from the form of themes, and an unmarked form, which is identical to the form of themes. In these systems the marked form always has a broader distribution than the unmarked form. The marked form always seems to be the elsewhere case, while the unmarked form has a restricted distribution (often only occurring in canonical word orders).

\section{Active}
All of the Germanic languages permit the recipient-theme word order with full noun phrases, and for almost all of them it is the canonical word order (some forms of Afrikaans may be the exception). Some of the languages do not permit the theme-recipient word order (Yiddish and Icelandic). Among languages that do permit the theme-recipient word order, it seems to often involve pragmatic/prosodic/information structural constraints on the relationship between the theme and the recipient. The general pattern seems to be that there is a pressure to have short, given, unfocused phrases occur before heavy, new, focused phrases. Since the theme-recipient order is a marked/derived word order, in languages with mixed recipient marking, it often requires the marked variant.

Many of the Germanic languages have some form of special pronominal movement. \cite{Sportiche.1996} argues that clitics are generally generated in object position and then move to adjoin to a higher position on the tree. In Romance, these clitics then undergo a futher operation which attaches them to the finite verb, so that they will move along with the finite verb. In some Germanic language, the first operation seems to occur, but the second operation does not. 

In the West Germanic languages, except for Modern English, if there are two pronouns, they both need to undergo this movement. In these languages, the movement respects locality, so first the recipient moves, and then the theme moves above it, which necessitates a theme-recipient word order. In the North Germanic (Scandinavian) languages, however, while unstressed pronouns obligatorily leave the vP, they do not change their order with respect to one another.
 
\section{Passive}
The get-passive, when it occurs, seems to allow the recipient to receive nominative case, while the theme receives accusative. The get-passive construction, however, seems to be quite restricted in its distribution. 

With the be/become passive, there is variation between two different patterns of argument marking. One pattern has the recipient preposition marked or dative case marked, and the theme receiving nominative case. The other pattern has the recipient receive nominative case, and the theme receive accusative case.

The first pattern is obligatory for most speakers of Icelandic and Faroese, as well as speakers of High German, Dutch, Frisian and Low German. Note that this includes both languages with synthetic case marking (Icelandic, Faroese, and High German), as well as languages that have lost the dative/accusative distinction (Dutch, Frisian and Low German). Also, some modern speakers of Icelandic and Faroese, as well as speakers of Halsa Norwegian, who all have obligatory dative case marking in the active, allow the recipient to receive nominative case in the passive. These facts together argue against an association of overt dative case and the ability of recipients to receive nominative case.

On the other hand, there does seem to be a strong correlation between having a higher subject position, and the ability of the recipient to receive nominative case. The languages which do not have a higher subject position (e.g. High German and Dutch), also do not allow the recipient to be marked with nominative case. Among the languages with an overt subject position, the recipient passive is always a possibility, and there seems to be pressure towards allowing nominative case marking (e.g. Nominative Sickness in Icelandic and Faroese). Thus, it seems that there is a pressure in languages with a higher subject position to associate that position with nominative case.

The theme passive is also universally available among languages with the higher subject position. In most of the languages, it is the preferred passive option, with Icelandic and Modern American English being the main exceptions. For most mixed marking languages, the theme passive, like the theme-recipient word order in the active, requires (or strongly prefers) the marked recipient variant.

I need to figure out something to say about expletive passives, but I don't know what that is right now.
\chapter{Diachronic Generalisations}\label{chap:diagen}
\section{Recipient Marking}
There seems to be a general trend towards a loss of synthetic case. \cite{Bardal.2009} shows that at least for Scandinavian, the loss of case marking cannot be the product of regular sound change, since the same phonological sequences also occur in verbal morphology endings, and are not lost at the same time. For many of the languages (with the exception of Low German), the loss of case co-occurs with a rise in the use of prepositional marking. I suggest that the causation in these languages is that prepositional marking increases, co-occuring with synthetic case marking (e.g. High German dialects). At some point, the preposition is re-analysed as the case marker, and the old case marking system is lost.
\section{Active}
Most of the languages allow both the theme-recipient and recipient-theme word orders. The only languages that do not Icelandic and Yiddish, move in the direction of allowing only the recipient-theme word order. Assuming that the recipient-theme word order is basic, this amounts to losing the movement operation that shifts the theme over the recipient.
\section{Passive}
In the attested history of the Germanic languages, the trend seems to be towards developing a higher subject position out of languages that did not have one. Old Scandinavian still seems to have vestiges of a previous system without the subject position. Afrikaans developed a higher subject position from Dutch, which lacks one. There seems to be a correlation between the switch from OV to VO and the development of a subject position \citep{Besten.1990}, however, Afrikaans made the shift while remaining OV.

\cite{Eythorsson.2000} argues that the change from oblique subjects to nominative subjects in Icelandic is driven by a pressure towards syntax-morphology mapping, while the change from accusative (and sometimes nominative) experiencers to dative can be attributed to a pressure towards semantics-morphology mapping. This matches with the synchronic link between languages with higher subject positions and nominative case marking for recipients.

\chapter{Theoretical Implications}\label{chap:theory}
\chapter{Further Questions}\label{chap:furtherquest}
Old Scandinavian and Icelandic Passive.
Faroese object shift with ditransitives (Only theme pronoun?)
Mainland Scandinavian get-passives
Norwegian and Swedish pronouns
Yiddish Passives
Prepositional marking in Dutch
