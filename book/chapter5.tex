\chapter{Full Case Study in English Diachrony}\label{ch:diachron}
\section{Introduction}
This chapter applies the analysis presented in the previous three chapters to a large scale case study, namely diachronic development in recipient ditransitive syntax in the history of English. As discussed in the introduction, quantitative (and especially) diachronic studies can provide a useful independent verification of analyses developed on the basis of acceptability judgements. Crucially, data from language production can provide independent verification of theories developed primarily from language comprehension (i.e., acceptability judgements).

The problem of finding empirical validation of theoretical claims is made acute by the nature of the types of claims made in theoretical linguistics. Building on the work starting during the cognitive revolution in the 50s and 60s, the goal of generative linguistics has been to study the linguistic competence of speakers, which consists of the language specific information that is needed to use a language natively \citep{Chomsky.1981,Chomsky.1986}. As will be shown below, this linguistic competence can be separated into grammatical and non-grammatical competences. Grammaticality reflects the ability of a grammar to associate a particular utterance with a particular meaning. However, given the rampant ambiguity in natural language, the grammar of natural languages often associates multiple utterances with a particular meaning (and multiple meanings with a particular utterance). An equally crucial aspect of being a native speaker of a language is knowing which of the options produced by the grammar to use in any particular circumstance. These choices are often impacted by language specific implementations of general social or psychological factors (see \cite{Bresnan.2007,Bresnan.2010,Zeevat.2014} and \cite{Tamminga.2016} for a discussion of these issues and their relationship to the grammar). Unfortunately, it has been known since the beginning of this enterprise that there is no direct evidence of linguistic competence (see \cite{Schutze.1996} for a discussion of early claims about this issue), which is typical of knowledge and psychological constructs. Instead, it has been necessary to deduce the nature of the linguistic knowledge by studying its effects on language performance (see \citealt{Stroud.2012,Phillips.2013, Phillips.2013b, Phillips.2013c} for an arguments that acceptability judgements are fundamentally performative).

One of the most prominent types of linguistic performance to be used in theoretical linguistics is the acceptability judgement. These judgements reflect a native speakers sensation of naturalness/unnaturalness upon encountering a particular linguistic utterance. These sensations have a cognitive reality similar to that of pain sensations \citep{Schutze.2014}. A major advantage to the acceptability judgement is that even utterances that would never occur in natural production (due to the combination of factors each of which is extremely infrequent) can still be studied. However, as mentioned in the first chapter, grammaticality is only one aspect that contributes to the sensation of naturalness; other factors such as pragmatic concerns can often render a perfectly grammatical utterance unnatural (e.g., because there is a more concise grammatical way of conveying the same information). Trained linguists (and ideal native language informants) are able to minimise contextual factors that impact naturalness by attempting to evaluate the utterance in a number of hypothetical linguistic contexts, but these techniques cannot rescue a grammatical utterance that is ruled out because of universal, overwhelming problems. These non-grammatical problems often have a gradual impact on acceptability, reflect a gradient notion of pragmatic infelicity or psychological complexity \citep{Bresnan.2007,Bresnan.2010,Schutze.2014}.

Quantitative studies of language performance is useful for isolating these gradient factors, so that they can be factored out when studying grammaticality using performance data. Since corpora (ideally) provide multiple instances of the relevant features in a variety of pragmatic contexts, the gradient effects of non-grammatical factors can be investigated for the observed contexts and statistically extrapolated to unobserved contexts. In addition, corpora provide a means of studying diachronic processes that cannot be studied using traditional acceptability judgements, since the earlier speakers in the diachronic process are unavailable for consultation. Assuming that language change cannot radically alter the underlying grammar (since the speakers of the new variety must participate in a speech community with speakers of the old variety), it is possible to provide independent evidence of the internal structure of the relevant grammatical processes.

This chapter will begin by reviewing the analyses from the previous three chapters and discussing the diachronic implication of these analyses. This will be followed by looking at two independent changes in the history of English. The first change is the change in recipient marking, ranging from synthetic dative case in Old English to the current distribution of `to' in modern American English. Finally, the fall and rise of recipient passivisation will be examined going from Old English to modern American English.

\section{Theoretical Issues}
	Since this dissertation argues that all languages have the same underlying configuration of recipient and theme (i.e., the recipient is introduced as a dative PP in the specifier of an applicative phrase), I predict that there should be no diachronic development in base generation. However, there are a number of transformations that can apply to the base generated order and different stages of the language can vary as to which operations are grammatical and when they should apply.

	One of the major factors that impact the surface realisation of recipients is allomorphy with respect to the morphological realisation of the dative P-head. The P-head itself can receive a null realisation or be spelled out overtly (e.g., as the preposition `to'). It can also trigger concord on its complement, which causes the realisation of synthetic dative case on elements in the noun phrase. As an instance of allomorphy, these variants can be sensitive to contextual information (e.g., the properties of surrounding elements). The nature of these links are the essence of Sassurian arbitrariness and are thus predicted to be subject to drift over centuries of language change.

	In addition to morphological variation, there are a number of syntactic operations that impact both the surface order of elements and their syntactic hierarchy. Since all of these operations are optional, any stage of the language could fail to have them. Thus, grammatical change is predicted to involve gaining or loosing one of the operations. (or change in the effect of extra-grammatical factors on the application of these processes) These operations are one of the main sources of ambiguity that necessitates the non-grammatical component of language competence. Thus, change could also impact the rate at which these grammatical operations apply. VP-internal scrambling derives a theme--recipient order from the underlying recipient--theme order by moving the theme to a higher specifier of the applicative phrase. Cliticisation moves a pronominal element from being an independent syntactic head to being adjoined to a head in the verbal spine (here the head will always be little-v/voice). Finally, P-incorporation can move the dative P-head out of the PP and adjoin it to the next highest head. This renders the complement of the preposition a bare DP, which makes it eligible for receiving structural case.

	Looking at passivisation, the availability and probability of the transformations discussed in the previous chapter alters the availability of the theme and the recipient to raise to subject position and receive nominative case. In addition, languages vary as to the permissability of T in assigning subject properties. The main variation is in the treatment of PPs in the search for a subject. The assignment of nominative case (as a structural case) is restricted to DPs, a fact which does not vary diachronically (modulo the presence/absence of P-incorporation). However, the search for an argument to raise to subject position shows a variety of possible treatments for PPs. PPs can be valid targets for subject raising (oblique subjects), they can be invisible for subject raising (triggering direct theme passivisation), or they can be defective interveners (requiring one of the operations from the previous paragraph to create a non-intervening configuration).

	In the history of English, almost all of the changes discussed above occur. The realisation of dative P shifts from synthetic dative case to `to' alternating contextually with a null realisation. While VP-internal scrambling is grammatical in all stages of English, the conditioning factors change moving from Old English to modern American English. Cliticisation is lost during the history of American English, while P-incorporation becomes common place. Finally, all of the possible treatments of PPs in passivisation are attested (oblique subject, direct theme passivisation and defective intervention). Crucially, in every case of a change in grammaticality, the analyses presented here can account for the surface change by positing a change in the distribution of independent syntactic operations.

\section{Recipient Marking}
	Old English had synthetic dative case marking inherited from proto-Germanic. While there was a great deal of syncretism in the Old English case system \citep{Allen.1999}, there was a reliable distinction between accusative and dative case for many noun classes and in pronouns.

	By the end of the Old English period (11th century), these distinctions were breaking down. Nominal case marking was no longer reliable. While both accusative and dative pronominal forms were still being used, the forms were no longer consistently distributed along the accusative/dative case distinctions (i.e., old dative case forms would be used where previously accusative case was required and vis-a-versa). Around this time, `to' began to be used for the first time to introduce recipients. In Old English, `to' had previously been restricted to goals and addressees, i.e., the indirect object of verbs of communication \citep{Allen.1999,McFadden.2002,OED.2013}. 

	\begin{figure}[p!]
		\includegraphics[width=.95\linewidth]{../images/to-marking-graph}
		\caption{Empirical frequency of `to'-use with predicted rates according to model in different conditions}
		\label{fig:to-use}
	\end{figure}

	Throughout the Middle English period (i.e., up until about 1400), `to' became more prominent across the board. However, during the 15th and 16th centuries, a new grammar arose in which the unmarked form was preferred when the recipient was adjacent to the verb. This adjacency could be satisfied either by (a) having the recipient--theme word order (e.g., ``John gave Mary the book'') or (b) by having a theme pronoun cliticise in the theme--recipient order (e.g., ``John gave it him''). In the rest of this section, I discuss statistical evidence supporting the explanation given above.

	One of the major discoveries coming from the quantitative study of diachronic syntax has been the Constant Rate Effect. This effect has obtained in cases where a single syntactic mechanism can apply in multiple syntactic environments. In these cases, it has been repeatedly found that the slope parameter assigned by logistic regression (i.e., the effect of year of production on the frequency of use of the incoming variant) is constant across its different environments of application (this is true even in cases where the environments themselves show different frequencies of use). 

	The first example of this from \cite{Kroch.1989}, where the use of do-support was studied in a number of different environments (e.g., negative declaratives, affirmative questions, negative questions, imperatives, etc.). Kroch found that while the use frequency of do-support in these environments differed from one another in any given year (see Fig. \ref{fig:kroch-graph}), the rate at which these frequencies changed was constant across environments. He hypothesised that this effect reflected the fact that only one change was taking place (the loss of V-to-T raising). Under this hypothesis, the Constant Rate Effect provides a means of recovering underlying grammatical information from diachronic patterns in language use. If a Constant Rate Effect is found (assuming that one has enough data that it would be possible to fail to find it), the most parsimonious hypothesis is that there is a unified change underlying the variation in each environment.

	\begin{figure}[ht!]
		\includegraphics[width=.5\linewidth]{../images/kroch-graph}
		\caption{Frequency of do-support in different environments: affirmative and negative questions (? and \sout{?}) and affirmative and negative declaratives (+ and ') (Fig. 1 from \citealt{Kroch.1989})}
		\label{fig:kroch-graph}
	\end{figure}

	For English `to'-use, there are two changes to be considered. The first is the introduction of `to' as a means of marking recipients. The second is the introduction of null marking in verb-adjacent contexts. In order to study these changes, I extracted all tokens from the Parsed Corpora of Historical English (CITATIONS) containing the following recipient introducing verbs (verbs that also introduce goals, e.g., SEND, were excluded): ALLOT, APPOINT, ASSIGN, AYEVEN, BEHIEGHT, BEQUEATH, BETAKE, DAELAN, FEED, GIVE, GRANT, LEND, OFFER, OWE, PAY, PROFFER, PROMISE, RESTORE, SELL, SELLAN, SERVE, SHOW, VOUCHSAFE, and YIELD. I also extracted information about whether the arguments were full noun phrases or pronouns, the relative order of the recipient and theme (and their order with respect to the verb to rule out cases of topicalisation), and whether or not the recipient was marked with `to' (passive data was also collected, which is discussed in the subsection below). When the theme is a pronoun, the theme--pronoun order was essentially categorical (31 examples of recipient--theme order over 1000 years). Since there was such poor evidence for the frequency of `to'-use in these environments, their inclusion muddled any attempts at statistical analysis. Therefore, those cases have been excluded for the analyses discussed below (and were excluded from Fig. \ref{fig:to-use} as well).

	As can be seen in Fig. \ref{fig:to-use}, in theme--recipient contexts, the frequency of `to' use increased in the typical S-shaped curve expected from logistic regression \citep{Kroch.1989}. However, as shown in Table \ref{tab:tp-to-rates}, when both the theme was a pronoun, the S-shaped curve took much longer to get to 100\% (and did not get to 100\% with recipient pronouns). As discussed in Chapter \ref{ch:active}, this decreased rate of `to'-use reflects the availability of theme pronoun cliticisation to render the null allomorph available. This variation is not stable, since there is a significant increase in `to'-use from 1500 onwards (according to comparison of logistic regression models). 

	\begin{table}[ht!]
		\begin{tabular}{ccc}
						& Recipient Noun & Recipient Pronoun\\
			Old English		& 0\% (43) & 0\% (20) \\
			Middle English		& 93\% (105) & 28\% (42) \\
			Early Modern English	& 93\% (235) & 48\% (122) \\
			Late Modern English	& 100\% (76) & 63\% (38) \\
		\end{tabular}
		\caption{Rates of `to'-use with theme pronouns in different stages of English}
		\label{tab:tp-to-rates}
	\end{table}

	Turning to cases with full noun phrase themes, the recipient--theme word orders also fail to show a clean S-shaped curve. Instead, as discussed above, `to'-use until about 1400 and then slowly tapers off (Fig. \ref{fig:to-use}). As discussed in \cite{Bacovcin.2016}, this effect can be modelled by multiplying two independent logistic changes together. An advantage of separating the change into two distinct logistics, is that the Constant Rate Effect can be tested for. In other words, it is possible to see if the rise of `to' in theme--recipient and recipient--theme contexts reflected a uniform underlying change. The Constant Rate Effect predicts the \textbf{absence} of significant interactions between year and condition factors (in this case, word order and the status of the theme and recipient). 

	The general structure of logistic regression involves the logistic function: 1/(1+exp(-parameters)), where the parameters are a linear model (i.e. predicting a number ranging from -Inf to +Inf) that is transformed into a number between 0 and 1 by its inclusion in the logistic function. The final model for all environments discussed here was: 1/(1+exp(-(change1parameters))))*(1-(identifier/(1+exp(-(change2parameters))))), where identifier was 1 for recipient--theme orders and 0 for theme--recipient orders. The interpretation of this is that the first logistic equation models the rise of `to'-use, while the second equation generated models the rise of the null allomorphy grammar. The results of the null allomorphy grammar were subtracted from 1 so that the identifier would cause the null allomorphy change to only affect recipient--theme word orders. This complex model was fit using the nloptr R package for non-linear optimisation (see the accompanying R files for more detail). The optimum model was found using stepwise AIC, where parameters were individually added to the model (always including the parameter that created the greatest decrease in AIC) until no further improvement was found.

	The optimal model had the following properties (model fits shown in Fig. \ref{fig:to-use}): (a) the first change was characterised by an intercept, an effect of year, an effect of word order and an effect of recipient status, (b) the second change was characterised by an intercept, an effect of year, and effect of recipient status, and an interaction between year and recipient status. The effects in model (a) indicate a Constant Rate Effect for the introduction of `to'-marking for recipients; there was no significant interaction between conditions. While `to'-marking raises at the same rate across all conditions, recipient--theme orders and recipient pronouns show less `to'-use in any given year. In theory, the Constant Rate Effect could be driven by a lack of power (i.e., there may have been a difference in rates, but there was not enough data to detect it).
	
	The power explanation for the Constant Rate Effect becomes less plausible, given that a significant interaction was identified for the second change (indicating that there was enough power to identify interactions). The identification of a significant interaction strongly suggests that the null allomorphy grammar actually reflects two different changes (one for recipient nouns and one for recipient pronouns).The effect of year is higher for recipient pronouns, suggesting that the null allomorphy grammar affected recipient pronouns faster than full noun phrases.
	
	Since many languages have a strong differentiation between recipient marking on full noun phrases and pronouns (e.g., Romance language differences between full noun phrases marked with \textit{a} `to' and clitic pronouns marked with synthetic case marking), this differentiation between a nominal and pronominal grammar is not unexpected. It also supports the idea that some of the null recipients in theme--recipient word orders with recipient pronouns (e.g., ``gave it him'') is driven by a pronoun specific grammar that promotes the null allomorph.

	This modelling technique allowed for confirmation of another case of the Constant Rate Effect, involving a complex interaction of properties. Data from the corpus on the spread of `to'-marking is consistent with it involving a single change across all recipient constructions. The complex allomorphy seen in Modern British English can be explained as the interaction of this simple morphological change in the base realisation of recipient marking in combination with two additional morphological operations. The first is a time independent dispreference for marking recipient pronouns with overt case markers (a fact that may be linked to the maintenance of a synthetic case distinction on pronouns). The other is the introduction of the adjacency trigger null allomorphy, which becomes prominent during the 15th and 16th centuries, and reflects two sub-changes affecting full noun phrases and pronouns differently.

\section{Passivisation Order}
	\begin{figure}[p!]
		\includegraphics[width=.9\linewidth]{../images/rec-pas-graph}
		\caption{Rates of recipient fronting in passives (points indicate raw frequencies; lines represent LOESS smooths)}
		\label{fig:rec-pas}
	\end{figure}


	Given that there are a larger number of relevant mechanisms that impact passivisation, the changes in this domain prove more complex. One simplification is that, for the British data, there is no variation when the theme is a pronoun. In this case, the theme always raises to subject position. Given that there are only 123 examples of ditransitive passives with theme pronouns over the more than 1000 years covered by the corpora, the evidence that recipient passivisation with theme pronouns was ungrammatical is only suggestive. However, given the lack of variation, I focus on cases with full noun phrase themes for the remainder of the discussion of the British data.
	
	The situation in Old English is quite complex. \cite{Allen.1999} provides evidence that monotransitive datives are able to become oblique subjects in Old English, but she suggests that in ditransitives, putative oblique subjects are actually topics. To discuss this distinciton, she introduces the term ``fronted dative'', which is agnostic as to whether the fronted element is a topic or a subject. The argument about the status of fronted datives in ditransitive passives is made on the basis of Coordinate Subject Deletion facts. In Old English (as in Modern English), arguments are generally obligatory (i.e., neither subject nor object drop is generally licensed). However, when two sentences are coordinated and share the same subject, the subject does not need to be expressed in the second sentence (\ref{ex:OECSD}). In a corpus investigation, none of the fronted datives in ditransitive passives triggered Coordinate Subject Deletion, while a number of fronted nominatives did (see Table \ref{tab:AllenOECSD}). 

	\begin{table}[t]
		\begin{tabular}{cccc}
			Nominative Coreferential & & Deletion & No Deletion \\
			& Order NOM DAT & 11 & 4 \\
			& Order DAT NOM & 4 & 3 \\
			& Total & 15 & 7 \\
			\hline
			Dative Coreferential & & Deletion & No Deletion \\
			& Order NOM DAT & 0 & 27 \\
			& Order DAT NOM & 0 & 11 \\
			& Total & 0 & 38 \\
		\end{tabular}
		\caption{Allen's counts of Coordinate Subject Deletion with ditransitive passive in OE prose (Table 2-6, \citealt{Allen.1999})}
		\label{tab:AllenOECSD}
	\end{table}

	\begin{exe}
		\ex \label{ex:OECSD} Old English:
		\gll and him comon \textbf{englas} to, and him ðenodon\\
		and him.DAT came \textbf{angels.NOM} to, and him.DAT served\\
		\trans ` and to him angels came and him (they) served \citep[ex. 34]{Allen.1999}.''
	\end{exe}

	The main problem with this conclusion is that there were only a small number of Old English coordinated examples, such that the lack of deletion for datives could be accidental. For the rest of this section, I do not take a stand on whether the arguments in Old English were subjects or not. I focus instead on general changes in word order (independently of whether that order is derived from subject raising or topicalisation). Before turning to changes in word order, changes in recipient marking need to be addressed.

	Old English did not permit dative-to-nominative conversion. If Old English allowed dative subjects in ditransitive passives, they were oblique subjects \'{a} la Icelandic. Given that synthetic case marking was lost by the end of the Old English period, it is difficult to identify whether a sentence in Middle English has an oblique or nominative subject (since the distinction is not clearly marked).\footnote{This is also before `to'-use has been fully adopted, so the absence of `to' would not be evidence of nominative case.} \cite{Allen.1999}, after carefully studying the extant Middle English corpus, identifies that the first unambiguous case of nominative recipient passivisation occurs around 1375. 

	This timing ends up supporting the connection between dative-to-nominative conversion and pseudopassivisation. \cite{Sigursson.2014}, using the Parsed Corpora of Historical English, shows that Old English lacks pseudo-passives, but that while the first examples occur in the 13th century, pseudo-passivisation only become prevalent during the 14th and 15th centuries, which is the same time period that recipient passives begin to show nominative case marking.

	Overall, given the relative rarity of both recipient ditransitives and passivisation, there is only tentative evidence from the parsed corpora about the behaviour of ditransitive passivisation in British English. The evidence (Table \ref{tab:rec-pas-brit}) does support certain claims: (a) recipient fronted passivises are more common with recipient pronouns than recipient nouns (reflecting the preference for recipient--theme orders in actives with recipient pronouns), (b) recipient fronted passives become less frequent from Old English till the 20th century, and (c) the introduction of dative-to-nominative conversion does not substantially alter the change in (b). There is also some evidence that recipient passivisation may be increasing during the late 19th and 20th centuries, which is supported by the existance of a significant positive effect of year in a logistic regression model looking at the data from 1800 onwards.

	\begin{table}[ht!]
		\begin{tabular}{ccc}
				&	Recipient Noun	& Recipient Pronoun \\
		900--1099	&	16\% (19)	& 100\% (12) \\
		1100--1299	&	17\% (6)	& 83\% (6) \\
		1300--1499	&	18\% (39)	& 42\% (33) \\
		1500--1699	&	10\% (124)	& 36\% (85) \\
		1700--1899	&	5\% (116)	& 12\% (59) \\
		1900--1914	&	33\% (12)	& 60\% (5) \\
		\end{tabular}
		\caption{Rates of recipient first passives with full noun phrase themes from the Parsed Corpora of Historical English (total token numbers in parentheses)}
		\label{tab:rec-pas-brit}
	\end{table}

	Under the system proposed here, the increase in nominative recipient passivisation reflects an increased use of P-incorporation. While P-incorporation becomes grammatical in the Middle English period (as seen by pseudopassivisation and rare nominative recipient passives), it is not frequently used in ditransitive passives. Thus, P-incorporation is a case where the grammar can generate possibilities that are subsequently dispreferred. In other words, for pseudo-passives P-incorporation is the only possible way of demoting the agent (assuming agent demotion is the main pragmatic role of passivisation). However, for ditransitives, P-incorporation is only one of a set of possible operations that license ditransitive passivisation. Building on the rarity of recipient fronted passives in earlier stages of English, there was a dispreference for recipient fronting, which extended to a dispreference for nominative recipient passivisation using P-incorporation, given that other options are possible (namely theme passivisation).

	While there is tentative evidence from British English that P-incorporation began to loose its stigma during the 19th century, the stigma is clearly lost over the history of American English. Looking at data from the Corpus of Historical American English \citep{Davies.2010}, I extracted all instances of passive sentences involving the verbs `give' or `offer'. For `give', I hand coded all of the examples,\footnote{The GIVE examples were initially extracted using the web interface, before the corpus was available for download. The hand coding of all the web examples were converted into the same structure as the downloaded examples so that anyone with access to the downloaded corpus can check the coding.} while for OFFER I coded a sample of 50 tokens for each year.\footnote{For years that had less than 50 examples, all of the examples for that year were coded.} Each token was coded for the type of passive (theme, recipient, or monotransitive), the status of the recipient (full noun phrase or pronoun), and the marking of the recipient (presence/absence of `to'). For `offer', I also hand coded the status of the theme (full noun phrase or pronoun). For `give', I use a script to determine where the theme was (based on the type of passive coding) and then determined the status of the theme automatically.

	
	The result of this study shows that the main change in ditransitive passives in American English was an increase in the rate of recipient passivisation. In the early 19th century, there are essentially no examples of recipient passives. Going through the 20th century and into the 21st, recipient passivisation becomes more prevalent (see Table \ref{tab:am-rec-rates}). Logistic regression models confirm that the change in recipient passivisation during this period is significant. In Old English, recipient pronouns frequently fronted, but recipient nouns rarely did. This distinction does not hold for American English, where both recipient pronouns and recipient nouns begin to raise to subject position, even when the theme is a pronoun (which never happens in the British English data).

	\begin{table}
		\begin{tabular}{cccc}
	&			&	Theme Noun	&	Theme Pronoun\\
19th Century& Recipient Noun	&	5\% (2238)	&	4\% (212)\\
	& Recipient Pronoun	&	11\% (1294)	&	2\% (224)\\
20th Century& Recipient Noun	&	26\% (3974)	&	8\% (292)\\
	& Recipient Pronoun	&	46\% (1703)	&	16\% (309)\\
21st Century& Recipient Noun	&	43\% (189)	&	18\% (22)\\
	& Recipient Pronoun	&	82\% (111)	&	15\% (13)\\
		\end{tabular}
		\caption{Rates of recipient--theme order in American 21st century GIVE and OFFER from COHA (total number of tokens in parentheses)}
		\label{tab:am-rec-rates}
	\end{table}

	According to the analyses proposed here, this change in American English (and possibly British English) reflects a normalisation of P-incorporation. In Early Modern English (and at least early 20th century British English), P-incorporation was a marked operation. If there was no other way of implementing passivisation (i.e., in the case of pseudo-passivisation), P-incorporation was available. However, if possible, speakers preferred to use other mechanisms for licensing passivisation. In American English, however, P-incorporation became a normal state of affairs. 
	
	I propose that this possibly comes from a reinterpretation of active data. The null allomorph for the dative P-head is triggered by being adjacent to the verb, but since it has no phonological realisation, it is impossible to precisely locate it. Learners of American English may have started to reanalyse active sentences with a null pronoun as involving P-incorporation. Once P-incorporation had become assumed in the active, it is plausible that it came to be used more frequently in the passive. Under this story, language learners only had indirect evidence for the rate of P-incorporation with recipients (unlike in pseudo-passivisation); this reliance on indirect evidence creates a situation that is ripe for language change.

\section{Conclusions}
	In this chapter, I have presented two case studies of change in the history of English. The first captured changes in morphological marking of recipients, which supports the interchangeability of prepositional and synthetic case marking. The second the fall and rise of recipient passivisation. The first change provided an example of how even complex changes involving the interaction of a number of moving parts (two changes each of which involved scaling factors) can be broken apart using relatively simple statistical processes. The breaking apart of interacting changes provides evidence for the structure of the underlying grammar using the implications of the Constant Rate Effect. The second change provided some insight into the underlying mechanism driving changes in use frequency of syntactic mechanisms. In at least some cases, it is the surface outcome of the mechanisms that is subject to change (i.e., recipient passives), independently of how that surface property is derived (i.e., oblique subjects vs. dative-to-nominative raising). While in many cases the surface properties that drive use-frequency changes are the direct reflexes of grammatical properties (as was the case with changes in allomorphy in the `to'-marking change), in other cases the relationship between use and grammar is more indirect with surface properties that have multiple grammatical causes being unified by language learners.

%\bibliography{diss}
