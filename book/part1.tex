%\chapter{Introduction}
%\label{introduction}
%
%
%
%%bibliography
%\usepackage{natbib}
%\bibpunct[:]{(}{)}{,}{a}{}{,}
%
%% phonological examples
%%\usepackage{simplex}
%\usepackage{amsmath}
%
%% fonts
%%\usepackage{mathspec}
%%\setmainfont[Mapping=tex-text]{Linux Libertine}
%%\setmathfont(Digits,Greek,Latin){Linux Libertine}
%%\usepackage{microtype}
%%\usepackage{coptic}
%
%
%% tables and figures
%\usepackage{booktabs}
%\usepackage{graphicx}
%\usepackage{floatrow}
%\usepackage{multirow}
%\usepackage{enumitem}
%\newfloatcommand{capbtabbox}{table}[][\FBwidth]
%\setlist{noitemsep}
%
%% Add packages and definitions you want to use here:
%\usepackage{times}
%\usepackage{multirow,sectsty}
%\usepackage{setspace}
%\usepackage{subfigure,graphicx}
%\usepackage{amsmath,amsthm,amsfonts, amssymb}
%\theoremstyle{definition} \newtheorem{definition}{Definition} 
%\usepackage{linguex}
%% \usepackage{betababel}
%\usepackage[english,greek]{betababel}
%\usepackage{tikz-qtree}
%\usepackage{tikz}
%\usetikzlibrary{arrows,automata,chains,matrix,positioning,scopes}
%
%\usepackage[normalem]{ulem}
%
%\usepackage{pdfpages}
%
%\usepackage{natbib}
%
%\usepackage{epigraph}
%\usepackage{hyperref}
%
% \usepackage[only, llbracket,rrbracket]{stmaryrd}
% \newcommand{\sem}[1]{\ensuremath{\{ #1 \} }}
% \newcommand{\pair}[1]{\ensuremath{\langle #1 \rangle}}
% \newcommand{\la}{\ensuremath{\lambda}}
% \newcommand{\inter}[1]{\ensuremath{\llbracket#1\rrbracket}}
%
%\newcommand*\circled[1]{\tikz[baseline=(char.base)]{
%            \node[shape=circle,draw,inner sep=2pt] (char) {#1};}}
%
%
%\newcommand{\comm}[1]{}
%\long\def\symbolfootnote[#1]#2{\begingroup%
%\def\thefootnote{\fnsymbol{footnote}}\footnote[#1]{#2}\endgroup}
%
%\begin{document}

%\setcounter{chapter}{0}
\chapter{Introduction}
\label{ch:introduction}

This dissertation argues for two main conclusions. First, we provide evidence suggesting that prepositional object constructions with recipient ditransitives (e.g. "John gave the book to Mary") should be analysed in the same way as scrambled accusative--dative structures in languages with overt case marking. This conclusion provides support for the syntactic unification of prepositions and case markers. Secondly, we argue that symmetric passivization with ditransitives (INSERT CITATIONS HERE) are derived by parameterising the relationship between nominative case assignment and movement to subject position. This parameter interacts with the availability of dative--to--nominative conversion to create the surface patterns of passivisation found in the Germanic languages.

The dissertation focuses on recipient ditransitives (e.g. those introduced by GIVE, PROMISE, SEND) from Germanic languages, although benefactive ditransitives and data from non-Germanic languages will be discussed. Recipient ditransitives form the most canonical of the ditransitive constructions (INSERT CITATION HERE) and exist in all of the Germanic languages, which make them an ideal case study for comparison.

The structure of the dissertation is as follows. I will finish the introduction section with a description of the Germanic languguages (i.e. where they are spoken, data sources, general properties of the languages). In Part II, we will discuss active ditransitive clauses focusing on word order possibilities between the two objects and morphosyntactic marking of the recipient theta role. In Part III, we discuss passive ditransitives and argue for the interaction of locality and case marking in producing the attested surface patterns. Finally, in Part IV, we present a formalisation of Part II and III in the Minimalist--Distributed Morphology framework, discuss implications of the results for the study of historical syntax, and conclude with a summary of the argument and implications for other domains (esp. the syntax of ergativity).

\chapter{Germanic Languages}

\label{ch:germanic-languages}
\section{Introduction to Northwest Germanic}
Northwest Germanic consists of the modern Germanic languages and their ancestors, to the exclusion of East Germanic (of which Gothic is the only daughter language for which any substantial evidence remains). The major divide in the family is between North Germanic (Scandinavian) and West Germanic. Among the North Germanic languages, historically there was a split between Norse in the west, which was the ancestor of Icelandic, Faroese and Norwegian, and East Scandinavian, which was the ancestor of Danish and Swedish. By the Middle Ages, however, the division between the Insular Scandinavian languages on the one hand (Icelandic and Faroese) and the Mainland Scandinavian languages on the other (Norwegian, Swedish and Danish) became much more relevant. Among the West Germanic group, the first major historical division is between the West Germanic group proper (the ancestor of Dutch, Afrikaans, High German and Yiddish), and the Ingvaeonic languages (the ancestors of Frisian, Low German, and Old English). Again, more recently the difference between Insular West Germanic (i.e. English) and the mainland languages has become more pronounced. Also Dutch (and its daughter Afrikaans) have begun to pattern more with the continental Ingvaeonic languages (i.e. Frisian and Low German). Both Frisian and Low German, on the other hand, have undergone strong sociolinguistic pressure from both Dutch, High German (and to a smaller extent Danish).

For each of the language family below, the structure of the description is the same. I start with older forms of the languages, and then move towards the modern period, when significantly different stages can be identified. Within each stage of the language, I start with an introduction to that stage, describing its location in both time and space, and describing the extant data, either in the form of written corpora, or current speakers. I then discuss the case marking system of the language, as well as the possibility of using prepositional marking with recipients. Then I go through the possible Active configurations and discuss what word order and case/prepositional marking patterns are permitted in that configuration, or demonstrate that the configuration itself is unattested/ungrammatical. Finally, I discuss the passive, starting with a discussion of subject properties (or lack thereof) in the language, and followed by a description of the parameter settings for all of the possible passive configurations (which will include both `'be/become'' and `'get'' passives in all the languages, but -s- passives only in Scandinavian).
\section{Scandinavian}
\subsection{Old Scandinavian}\label{sec:OldScand}
Old Scandinavian is the set of dialects spoken in the Scandinavian nations before 1500. There are four distinct dialects, although they are often described as being fairly homogenous with respect to syntax. The most studied dialect is Old Norse, which consists of documents written in Iceland. Closely related to Old Norse is Old Norwegian. More divergent are the Eastern languages Old Swedish and Old Danish.

\subsection{Modern Icelandic}\label{sec:Icelandic}
Spoken in Iceland from 1500.

\subsection{Faroese}\label{sec:Faroese}

\subsection{Norwegian}\label{sec:Norwegian}
There are two different standardised Norwegian languages, bokmål and nynorsk, both of which were standardised as part of the rise of Norwegian nationalism during the 19th century. Part of this nationalism was defined by anti-Danish sentiment, since the Danes had controlled Norway for centuries. Bokmål retains mainly of the Danish features that entered the language during Danish control, while nynorsk was created by looking at the most conservative Norwegian dialects, as well as attempting to predict what Old Norwegian would have looked like without Danish influence.


\subsection{Swedish}\label{sec:Swedish}
The language of Sweden from 1500 onwards.

\subsection{Danish}\label{sec:Danish}
The language of Denmark after 1500.

\section{High German}
\subsection{High German}\label{sec:HGerman}
High German consists of both Modern Standard German, and the dialects of the southern half of Germany, Austria and Switzerland.

\subsection{Yiddish}\label{sec:Yiddish}
Starting off as a dialect of Middle Bavarian, Yiddish was the dialect of German spoken by the Jewish communities of Central Europe. After being expelled from much of Central Europe, they moved to Eastern Europe, where Yiddish underwent a great deal of influence from Slavic languages. The standardised language was created based on a number of Eastern dialects during the early part of the 20th century.



\section{Low German}
\subsection{Dutch}\label{sec:Dutch}

\subsection{Afrikaans}\label{sec:Afrikaans}
Afrikaans is a descendant of a number of Early Modern Dutch dialects, which coalesced in Dutch South Africa. There are a number of other substrate influences, including native South African languages, colonial Portugese, and colonial English.

\subsection{Frisian}\label{sec:Frisian}

\subsection{Low German}\label{sec:LGerman}
Low German was the language of the Hanseatic league, a coalition of city states in northern Germany, which controlled much of the Baltic and North Sea trade. After the protestant reformation, Low German lost its prestige status to High German, which was the language of Luther. Low German has been under linguistic pressure from High German, Dutch and Danish for approximately the last 500 years. It is currently spoken natively by people in the far north of Germany, the eastern part of the Netherlands, and southern Denmark.

\subsection{English}
