\chapter{Introduction}
\label{ch:introduction}

The main goal of this dissertation is to argue that recipients are universally generated as dative KPs in the specifier of an applicative phrase. A secondary goal is to provide another example of how typological and diachronic data can supplement traditional synchronic acceptability judgements in arguing for syntactic analyses. Both goals are served by focusing on data from Germanic languagues and their histories. Typological and diachronic data are their most powerful when irrelevant variation between the languages can be ignored and variation in the property of interest can be focused on. This property is maximised by using a closely related set of languages, which still show a wide range of variation, a situation that obtains with respect to the Germanic languages.

The dissertation has the following structure. Chapter 2 explains the nature of the KP/applicative analysis, situates the analysis among competing analyses from the literature, and explores the empirical predictions of the at issue analyses. Chapter 3 explores data about active ditransitives from the various Germanic languages and how that data supports the KP/applicative analysis. Chapter 4 builds on the results from Chapter 3 and explores data about passivisation with ditransitives and their implications. Finally, Chapter 5 summarises the results of Chapters 2--4 and discusses the larger implications of the work.

%\bibliography{diss}
