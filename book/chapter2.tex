\chapter{Theoretical Background}
\label{ch:theoryback}
\section{Introduction}
The goal of this section is to introduce the theoretical options relevant to the claim of the dissertation: namely that recipients are universally merged as dative PPs in the specifier of an Applicative Phrase. This claim has three parts, each of which is explicated below. The first part of the claim concerns the nature of recipients. The first section in this chapter introduces the notion of theta roles, situates the work in the context of Dowty's Proto-Role theory and defines the notion of Recipient used here.

The second section deals with the second aspect of the claim, namely that Recipients are universally introduces as dative PPs. The possible difference between syntactic case and morphological case is explored with the claim that morphological case is the phonological realisation of syntactic case features. These features are then separated into structural and non-structural types, with dative case as an example of a non-structural case. The PP analysis is introduced as a way of capturing the structural/non-structural distinction. Structural case is a property of DPs, while non-structural case is the realisation of a P-head (or the reflex of concord with a P-head).

The third section addresses the structural claim of the dissertation, namely that recipients are introduced in the specifier of an applicative. Before explaining the applicative analysis, alternative analyses are introduced. The most radically different analysis introduces recipients as prepositional objects of verbs. \cite{Pylkkanen.2001} argues for a similar structure in her Low Applicative analysis, in so far as the recipient is introduced as an object of the verb. Another analysis, which argues that the recipient is introduced below the verb, suggests that recipients are the subject of small clauses. Finally, I adopt the analysis that place recipients in the specifier of an Applicative Phrase attached above the verb.

The final section introduces further morphosyntactic operations that are motivated in the following chapters. These operations are used to account for surface variation from the base generated order described here. In this chapter, I focus on introducing the operations and citing relevant background material; the arguments supporting the use of these operations and the evidence supporting the particular versions proposed here are found in the following three chapters. 

In the conclusion, I bring together a summary of the dative PP + Applicative analysis of recipients. I argue on purely theoretical grounds that (assuming it has empirical coverage) the dative PP + Applicative analysis is to be preferred as being more parsimonious. The next two chapters argue that the dative PP + Applicative analysis has at least as good empirical coverage (and sometimes better) than alternative theories.

\section{Thematic Roles}
This dissertation is about the morphosyntactic nature of recipients. This section is focused on defining what the dissertation considers recipients to be (and what it does not). The use of recipient here assumes that morphosyntactic constructions cluster around theta roles as privileged atoms of argument structure (this underlies the UTAH from \citealt{Baker.1988}). The only necessary aspect of this assumption is that the term recipient selects a set of semantically related arguments whose morphosyntactic realisation can be compared across languages. Theta roles are intended to classify the arguments of verbal events into related classes, for example Agent, Patient, Experiencer and Recipient. 

In particular, I am assuming a system similar to that of \cite{Dowty.1991}, who argued that theta roles like Agent and Patient are prototypes that particular arguments cluster around. Any particular argument may share properties with multiple different prototypes. In such cases where multiple proto-types are implicated, the role assigned to any particular argument of any particular verb is linguistically/culturally determined. The version of the theory I am assuming here assumes that roles like Recipient can be accessed by the morphosyntax to distinguish between different arguments (and the constructions they appear in). Languages (and possibly speakers) may differ as to which of these prototypical roles is assigned to any particular argument by any particular verb.

The prototypical recipient is the caused possessor in a transfer of possession event. Recipients are a particularly useful thematic role to study, because they almost always occur in triadic constructions (since there are almost always also an object transferred and a previous owner). Since ownership (and the transfer thereof) is an (essentially) universal property of human cultures, the recipient role guarantees a uniform point of comparison across various languages. 

The prototypical verb that introduces this role is thus GIVE, which indicates a semantically neutral transfer of a \textbf{theme} from an \textbf{agent} to a \textbf{recipient} without encoding anything about the manner of the transfer. Since the non-theme object of GIVE is the proto-type of the recipient role, the equivalent of GIVE across languages should be the focus point for studying recipient constructions. Other verbs \textit{may} introduce recipient roles, but the putative recipient could be construed (in that particular linguistic/cultural context) as being more similar to some other thematic role, and thus outside of the claims being made in this dissertation. The existence of ditransitive verbs that do not exhibit the behaviour expected from the dative PP + Applicative analysis can only count as counter-examples if it can be proven that the relevant argument is being treated as a recipient in that linguistic context.

\section{Recipient Case}
Since \cite{Vergnaud.1977} a distinction has been made between syntactic (or abstract) case and morphological case. Syntactic case has been viewed as a crucial property in licensing DPs. Morphological case refers to the affixes used in various languages to indicate semantic/grammatical roles (e.g., nominative, accusative, ablative, etc). In this dissertation, I assume that morphological case is the morphological realisation of syntactic case features \citep{Legate.2008}. This morphological realisation implies a grammatical relationship between abstract features and phonological forms, for example the operation of vocabulary insertion from Distributed Morphology \citep{Halle.1993}. In many languages the morphological reflex of syntactic case is null, which means that the evidence for syntactic case in those languages can only come from its impact on syntactic operations. Similarly, in a language with overt case realisations, morphological syncretism can cause distinct syntactic cases to have the same morphological reflex (e.g., German \textit{das} ``the'' is both nominative and accusative neuter). Finally, the same syntactic case can have multiple morphological reflexes in the same language, representing case allomorphy similar to multiple reflexes of plurality in English (e.g., dogs vs. children vs. women).

Two different analyses for the distribution of syntactic case has been proposed. The system dating back to \cite{Vergnaud.1977} argues that case is assigned in the syntax and plays a crucial role in licensing A-positions and triggering A-movement. Another strand, going back to \cite{Yip.1987}, argues that abstract case features are assigned post-syntactically dependent on the relative structural position of the arguments after syntactic operations are complete. Under this dependent case approach (further explored in \citealt{Marantz.1991,McFadden.2004} and others), syntactic operations cannot reference the abstract case properties of arguments (since they have not yet been assigned). While this dissertation does not make a strong claim on either side of this debate (i.e., the main claim of this dissertation is compatible with both accounts), the restriction of subject movement to elements capable of receiving nominative case in most languages suggests a close relationship between structural case availability and syntactic movement, which is more difficult to account for under the dependent case account.

Both analyses of case make a distinction between structural cases (e.g., nominative and accusative) and non-structural cases (e.g., ablative). The fundamental distinction between these two classes is their sensitivity to (relative) syntactic position \citep{Woolford.2006}. Structural case forms are manipulated by valency altering operations (e.g., passivisation or causitivisation), while non-structural cases are unaltered. The classic example of this is the transformation of accusative objects to nominative subjects in passives.

\begin{exe}
	\ex High German:\label{ex:hg-accnom}
	\begin{xlist}
		\ex \gll Ich habe den Mann gesehen\\
		I.NOM have the.\textbf{ACC} man seen\\
		\trans `I saw the man.'
		\ex \gll Der Mann wurde gesehen\\
		the.\textbf{NOM} man was seen\\
		\trans `The man was seen.'
	\end{xlist}
\end{exe}

Non-structural case, rather than being sensitive to syntactic position/valency, is associated with either particular semantic roles or idiosyncratic lexical assignment (see \citealt{Woolford.2006}). Dative case is generally considered a non-structural case, since it is associated with a specific semantic role (recipient) and generally is not altered by valency change operations (although see Chapter \ref{ch:passive} for a discussion of dative-to-nominative conversion). 

The PP analysis captures the structural/non-structural distinction syntactically. \cite{Bayer.2001} argues the non-structural properties of dative case in German can be captured by adding another structural layer above the dative DP: called the KP (for Kase Phrase). \cite{Asbury.2005,Asbury.2007}, looking at Hungarian and Finnish, shows how K and P occupy parallel structural positions and form similar roles (classification of the semantic role of the DP in the event structure). This follows a long tradition of associating certain types of (semantic) cases with prepositional phrases (\citealt{McFadden.2004} and citations therein). In this dissertation, I adopt use the term PP to refer both to classical prepositional phrases and also to Bayer-style KPs.

\cite{Asbury.2005} also explains why it appears that P-heads in many languages govern DPs that seem to have their own case marking (e.g., in High German, certain prepositions take arguments that have dative, accusative or genitive marking). Asbury argues that this phenomenon represents cases of preposition stacking, which can be supported by a comparison between English and German. 

In English, there is a distinction between `in' and `into' that represents the difference between a locative and goal interpretation of `in'. In German, the same distinction is made by changing the case marking on the DP (in + dative = `in' and in + accusative = `into'). The dative and accusative case can be seen as the corresponding elements to the plain `in' and the `to' in English `in' and `into' respectively. Thus, the accusative and dative forms in these cases do not reflect syntactic accusative and dative P-heads, but instead a locative and goal P-head respectively, which happen to be syncretic in their realisation on noun phrases with the dative and accusative case.

\begin{exe}
	\ex High German:\label{ex:hg-Pcomp}
	\begin{xlist}
		\ex \gll in + P$_{goal}$ den Baum\\
			 in + P$_{goal}$ the.ACC tree\\
			 \trans `into the tree'
		\ex \gll in + P$_{location}$ dem Baum\\
			 in + P$_{location}$ the.DAT tree\\
			 \trans `in the tree'
	\end{xlist}

\end{exe}

Traditional dative marked elements (as in German) do not surface with a separate lexical item indicating dative case (as in traditional prepositional phrases). Instead, the case information is represented on various elements of the DP (including the determiner, adjective or head noun). The transfer of the abstract case properties from the P head to the rest of the nominal elements is attributable to the same operation that spreads gender and number feature throughout the DP in cases of adjective/determiner agreement (see \citealt{Norris.2012} for a modern analysis of this phenomenon under the label concord). Once the features are attached to each of the elements in the DP, they can be associated with the appropriate morphological reflexes.

\begin{exe}
\ex Dative PP: \\
\xymatrix@=2pt{
 & PP\ar@{-}[dl]\ar@{-}[dr]\\
 P_{\text{dat}}\ar@{-}[d]& & DP\ar@{-}[dl]\ar@{-}[dr]\\
 \emptyset& \text{D}\ar@{-}[d] & & NP\ar@{-}[d]\\
 & \text{den}& & DP\ar@{-}[d]\\
 & \text{the.DAT.PL}& & \text{Kindern}\\
 & & & \text{children.DAT.PL}}
\end{exe}

Two different cases have been proposed for Germanic recipients: accusative and dative. As discussed above, accusative case is structural and dative case is non-structural. Given that some Germanic languages show a morphological distinction between accusative and dative case (with recipients receiving dative), proponents of the accusative case analysis argue that languages (and constructions within languages) vary as to the case assigned to recipients. The dative PP analysis predicts that there should be syntactic and/or morphological evidence for the dative P. The accusative analysis predicts that the recipient should behave like other accusative predicates for all purposes.

\section{Argument Structure}
The final component of the analysis of recipients discussed in this dissertation is their syntactic position. As was the case with the accusative case analysis of recipients discussed above, it is often the case that combinations of these analysis are assumed for different languages (or constructions within languages). I introduce alternative analyses first, starting with analyses that have recipients introduced as (part of) the complement of the main verb, and then conclude with the analysis that I am arguing for.

The first analysis holds that recipients are introduced as prepositional objects. This means that they have the same syntactic position as prepositional object in cases like ``John put the book \textbf{on the table}''. These analyses predict that recipients (of this type) should behave like other prepositional objects for all relevant purposes. The structure, which is assumed to be shared between these cases, has the theme in the specifier of the main verb and the recipient as its complement (see \citealt{Larson.1988}:ex. 13 and citations therein).

\begin{exe}
	\ex Prepositional Object Construction:\label{ex:POC} \\
\xymatrix@=2pt{
	& VP\ar@{-}[dl]\ar@{-}[dr] \\
 DP_{\text{Theme}} && \bar{V}\ar@{-}[dl]\ar@{-}[dr]\\
 &  V & & PP_{\text{Recipient}}}
\end{exe}


A similar analysis, which has the recipient as part of the complement of the main verb, is the Low Applicative analysis of \cite{Pylkkanen.2001}. This analysis places an applicative phrase as the complement of the main verb, with the recipient in the specifier and the theme as the complement of the applicative. 
\begin{exe}
\ex Low Applicative Construction: \\
\xymatrix@=2pt{
 & VP\ar@{-}[dl]\ar@{-}[dr]\\
 V & & ApplP\ar@{-}[dl]\ar@{-}[dr]\\
 & DP_{\text{Recipient}} && \bar{Appl}\ar@{-}[dl]\ar@{-}[dr]\\
 && Appl && DP_{\text{Theme}}}
\end{exe}

The main argument that Pylkkanen makes for her claim that recipients are introduced by a separate type of applicative from High Applicatives (e.g., instruments) is based on a claim about the semantics of recipients, namely that ``low applied arguments bear no semantic relation to the verb whatsoever; they bear only a transfer-of-possession relation to the direct object'' \citep{Pylkkanen.2008}. However, \cite{Larson.2010} shows that those semantics do not properly capture the meaning of recipients used by the relevant verb, which proves problematic for her system (see \citealt{Georgala.2012} for an alternative that captures the semantics, but is similar to the high applicative account argued for here).

Another analysis that has a similar structure to Pylkkanen's is the small clause structure proposed by \cite{DenDikken.1995} and adopted by \cite{Harley.2002,Harley.2015} and \cite{Ormazabal.2012}. Under this analysis, ditransitives are small clauses that are in the complement of the main verb. These small clauses place the recipient as the complement of a preposition.

\begin{exe}
	\ex Small Clause Analysis \citep[ simplified from ex. 38]{DenDikken.1995}:\\
			\xymatrix@=2pt{
			 & VP\ar@{-}[dl]\ar@{-}[dr]\\
			 V & & SC\ar@{-}[dl]\ar@{-}[dr]\\
			 & ``BE'' & & PP\ar@{-}[dl]\ar@{-}[dr]\\
			 && DP_{\text{Theme}} && \bar{P}\ar@{-}[dl]\ar@{-}[dr]\\
			 &&& P && DP_{\text{Recipient}}}
\end{exe}

Finally, the analysis argued for in this dissertation has the recipient introduced in the specifier of a head introduced above the main verb. \cite{Larson.1988} introduced the notion that this head is a purely formal copy of the main verb (or plausibly part of a lexical decomposition of the main verb) as part of his VP shell analysis. Building on work on Bantu, going back to \cite{Baker.1988b}, this head has been called an applicative head, since Bantu (and other languages) show an overt morpheme on the verb (called the applicative) that co-varies with the presence of recipients (and other elements). For ease of exposition, I adopt the applicative terminology, but none of my arguments hinge on this; the Larsonian VP-shell structure is equally compatible with my claims. 

\begin{exe}
	\ex Applicative Analysis (with dative PP):\\
			\xymatrix@=2pt{
			& ApplP\ar@{-}[dl]\ar@{-}[dr]\\
			PP_{\text{Recipient}} & & \bar{Appl}\ar@{-}[dl]\ar@{-}[dr]\\
			& Appl & & VP\ar@{-}[dl]\ar@{-}[dr]\\
			& & V & & DP_{\text{Theme}}}
\end{exe}

In many theories of applicatives (e.g., \citealt{Pylkkanen.2008} and \citealt{McGinnis.2001}, the applicative assigns the theta role to its specifier. For me, however, the specifier already has a theta role assigned by its P-head. Thus, the distinction between different types of applicatives (as seen in Bantu languages) cannot reflect different types of arguments introduced by the applicatives. Instead, it must reflect verbal agreement with the types of argument that occur in the clause (i.e., in the same way that subject agreement reflects the person/number of the subject without introducing those semantic features). Applicatives provide a formal role in providing a functional projection for additional arguments to be added, but do not themselves introduce the thematic role that the arguments in their specifiers play in the clause.

There is a split between the analysis I adopt (the applicative analysis) and all the alternatives, namely the position of the recipient vis-a-vis the main verb. All of the alternative analyses have the recipient as or as part of the complement of the main verb. The applicative analysis places the recipient higher than the main verb. Thus, the applicative analysis makes different empirical predictions about the relative C-command relationship between the recipient and the main verb (or material attached to the main verb).

\section{Morphosyntactic Operations}
There are five major morphosyntactic operations that I rely on to derive surface variation in Germanic ditransitives from the base generated structure that I argued for above: (i) contextual allomorpy, (ii) VP-internal scrambling, (iii) P-incorporation, (iv) cliticisation and (v) locality and intervention effects. In this subsection, I describe these operations and the assumptions that I rely on. While I will provide some examples of the kinds of surface structures that these operations generate, the evidence and arguments in support of these operations are presented in the next three chapters. I also show in the following chapters, that all of these operations are either independently necessary components of the grammar, or clearly necessary in at least some Germanic languages. Given that the operations are already necessary, there is no loss in parsimony to extend their coverage to cases that they were not previously used to account for (e.g., English ditransitives).

Contextual allomorphy is the operation that determines that the plural of \textit{book} is \textit{books}, but that the plural of \textit{sheep} is \textit{sheep}. In both cases, the syntax/semantics has a plural element and it is necessary to know what the phonological reflex of plurality is. The contextual aspect comes from the fact that plurality has different realisations depending on what noun they are adjacent to (see \citealt{Embick.2010} for an indepth discussion on the locality constraints on allomorphy). 

In this dissertation, I argue that many of the Germanic languages show allomorphy in the realisation of the dative P-head. In particular, many of the languages show an alternation between an overt and a null allomorph for the dative P-head. The null allomorph is often restricted to contexts adjacent to the verb. This alternation can be seen in English with `to' as the overt allomorph:
\begin{exe}
	\ex English, Dative Shift: \label{ex:dat-shift}
	\begin{xlist}
		\ex \label{ex:english-1} I sent the woman the book.
		\ex \label{ex:english-2} I sent the book to the woman.
	\end{xlist}
\end{exe}

The second operation is VP-internal scrambling. Given the base generated structure that I am assuming, the active word order should always be recipient--theme (e.g., ``I gave John the book''). However, in many of the Germanic languages, theme--recipient word orders are also grammatical (e.g., ``I gave the book to John''). Following \cite{Takano.1998} for English and a tradition going back to \cite{Lenerz.1977} for High German, I propose that these are derived via VP-internal scrambling.

\cite{McGinnis.1998} calls this operation A-scrambling and proposes that it targets a higher specifier of the applicative phrase (\ref{ex:doubspec-tree}). For McGinnis, this created a situation of Equidistance, where the theme and the recipient were unordered with respect to one another, since they were both specifiers of the same phrase. For me, the theme is asymmetrically c-commands the recipient; there is no assumption of Equidistance. I need the theme to be in a higher specifier of the same clause in order to be able to block P-incorporation as explained below. 

\begin{exe}
	\ex Scrambling Analysis:\label{ex:doubspec-tree}
\xymatrix@=2pt{
 & vP\ar@{-}[dl]\ar@{-}[dr]\\
DP\ar@{-}[d] & & \bar{v}\ar@{-}[dl]\ar@{-}[dr]\\
\text{Mary} & \text{v} & & ApplP\ar@{-}[dl]\ar@{-}[dr]\\
 & & DP\ar@{-}[d] & & \bar{Appl}\ar@{-}[dl]\ar@{-}[dr]\\
 & & \text{the book$_{i}$}\ar@{<-}@(dl,dl)[ddrrr] & DP\ar@{-}[d] & & \bar{Appl}\ar@{-}[dl]\ar@{-}[dr]\\
 & & & \text{to John} & Appl & & VP\ar@{-}[dl]\ar@{-}[dr]\\
 & & & & & DP_{i} & & \bar{V}\ar@{-}[d]\\
 & & & & & & & V\ar@{-}[d] & & CP\\
 & & & & & & & \text{give}}
 end{exe} 

 The third operation is P-incorporation provides a method for rendering dative PPs available for nominative case assignment. This argument assumes that nominative (as a structural case) is only available for DPs, and thus that the PP layer needs to be removed in order for the recipient to be visible for nominative case assignment. \cite{Alexiadou.2014} suggest that P-incorporation as a mechanism for removing the PP layer, but do not give an explicit account of P-incorporation.

 P-incorporation can be thought of as an exceptional instance of head movement (building on some of the original suggestions of head movement in \citealt{Baker.1988}). Head movement seems to be driven by properties of the mover rather than the landing site. An example of this principle can be seen in English V to T movement (as seen by do-support). Most English verbs do not move to T \citep{ex:main-verb-do}, but there are a class of auxiliary verbs that do \citep{ex:aux-verb-do}. Crucially, it is the class of verb (i.e., the type of head) that determines whether or not head movement occurs, not the tense of the clause.

 \begin{exe}
	 \ex Main Verb:\label{ex:main-verb-do}
	\begin{xlist}
		\ex[ ]{John did not eat the apple.}
		\ex[*]{John ate not the apple.}
	\end{xlist}
	\ex Auxiliary Verb:\label{ex:aux-verb-do}
	\begin{xlist}
		\ex[*]{John did not have eaten the apple.}
		\ex[ ]{John has not eaten the apple.}
	\end{xlist}
 \end{exe}

 In this case, P-heads in languages can vary as to whether or not the P-head undergoes head movement or not. An exceptional aspect of the head movement of P-heads is that it can be triggered from a specifier. Since head movement has been previously defined in terms of head-comp relationships, I provide a more general definition of head movement in (\ref{ex:headmovement}), which covers movement of heads out of phrases in both complement and specifier positions.

 \begin{exe}
	 \ex Head Movement Targeting Condition: When head movement is triggered, the head triggering the movement adjoins to the nearest head that asymmetrically c-commands the triggering head. \label{ex:headmovement}
 \end{exe}

 The condition in (\ref{ex:headmovement}) assumes Bare Phrase Structure \citep{Chomsky.1993}, where Phrases are labelled with their heads. A consequence of this is that entire phrases can satisfy the ``nearest head'' sub-condition. Therefore, the theme after VP-internal scrambling would be the target of P-incorporation from the recipient, since the D-head of the theme is the nearest head that asymmetrically c-commands the recipient (\ref{ex:doubspec-tree}). If the theme was not there, then the dative P would adjoin to v, since it would be the nearest head that asymmetrically c-commands the recipient.

 The fourth operation is cliticisation, which involves some combination of head movement and/or scrambling of weak pronouns. For this dissertation, the crucial elements of cliticisation are: (a) that the clitic is in an A-bar and not an A position and thus does not intervene for syntactic locality and (b) that the clitic ends up being incorporated into an adjacent word and thus does not intervene for linear locality, which is used in contextual allomorphy.

 Finally, when looking at passivisation the issue of locality becomes central. Assuming that there is no P-incorporation, VP-internal scrambling or cliticisation, the recipient will intervene between T and the theme when T is looking for an argument to move to subject position. Three possible intervention effects are seen: (a) the PP recipient is a valid target of subject movement (oblique subjects), (b) the PP recipient is invisible for subject movement and the theme raises past the recipient (relativised minimality), or (c) passivisation is impossible without P-incorporation, VP-internal scrambling or cliticisation (defective intervention). In all cases, it seems that these differences reflect different properties in the search mechanism of T as it looks down the tree for a possible argument to raise to subject position. I thus assume that different types of locality are stored as featural properties of T heads, which denote what sorts of phrases are examined and the possibility of defective intervention when looking for potential subjects.

\section{Conclusions}
This chapter gave further specification about the main claim of this dissertation. Recipients are defined as the proto-role, which is prototypically introduced by the verb GIVE (or its counterpart in other languages). Focusing on a thematic role eases cross-linguistic comparison, since all languages have some means of conveying a concept and those means can then be directly compared. One of the assumptions of this dissertation, however, is that the linguistic association of a particular verbal argument with the recipient theta role is culturally/linguistically determined. Thus, while the object of GIVE and its counterparts are always going to be recipients, that is not always the case for other verbs. At certain points in the following chapters, I mention possible counterexamples to my generalisations and claim that there are good reasons to think that these cases involve theta roles other than recipients, in particular I focus on the common confusion between Recipient and Goal arguments.

As mentioned in the previous two sections, most analyses of recipients claim that there is a diversity of constructions needed to analyse recipients, even across the closely related Germanic languages. This dissertation makes the strong (and more parsimonious claim) that only one analysis is needed for the syntax of all recipients. The complexity of surface forms comes from the interaction of the universal base order and \textbf{independently necessary} syntactic (scrambling, passivisation, and P-incorporation) and morphological (allomorphy) operations. The next two chapters shows how this analysis is able to capture the range of data from Germanic languages and explicates the syntactic and morphological operations alluded to in the previous sentence.

\bibliography{../diss}
