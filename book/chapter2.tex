\chapter{Theoretical Background}
\section{Introduction}
The goal of this section is to introduce the theoretical options relevant to the claim of the dissertation: namely that recipients are universally merged as dative KPs in the specifier of an Applicative Phrase. This claim has three parts, each of which is explicated below. The first part is that the claim is about recipients. The first section in this chapter introduces the notion of theta roles, situates the work in the context of Dowty's Proto-Role theory and defines the notion of Recipient used here.

The second section deals with the second aspect of the claim, namely that Recipients are universally introduces as dative KPs. The possible difference between syntactic case and morphological case is explored with the claim that morphological case is the phonological realisation of syntactic case features. These features are then separated into structural and non-structural types, with dative case as an example of a non-structural case. The KP analysis is introduced as a way of capturing the structural/non-structural distinction. Structural case is a property of DPs, while non-structural case is the realisation of a K-head.

The final section addresses the structural claim of the dissertation, namely that recipients are introduced in the specifier of an applicative. Before explaining the applicative analysis, alternative analyses are introduced. The most radically different analysis introduces recipients as prepositional objects of verbs. \cite{P

\section{Thematic Roles}
-Theta roles
-Dowty and Proto-Roles
-Universal prototypes with language specific allotment of semantic concepts to categories
	-Emphasise usefulness of recipients
	-Focus on core cases (e.g., GIVE) instead of border cases (e.g., TEACH), which may not be recipients

\section{Recipient Case}
-Morphological vs syntactic (case allomorphy)
-Structural vs non-structural
	-
-KP vs DP
\section{Argument Structure: Recipient--Theme}
-Pylkannen's lower applicative structure
	-Explain theory
	-Larsonian and Bruening criticisms
-Small clause (den Dikken)
-Small clause/lexical causative of get (Harley)
-High applicative/VP shell
\section{Argument Structure: Theme--Recipient}
-Base Generation
	-PP object
-Transformation
	-DOC derived from PP object
	-Scrambling of theme
\section{Conclusions}
-Summary of my conclusion

\bibliograph{diss}
