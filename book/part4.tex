%\chapter{Introduction}
%\label{introduction}
%
%
%
%%bibliography
%\usepackage{natbib}
%\bibpunct[:]{(}{)}{,}{a}{}{,}
%
%% phonological examples
%%\usepackage{simplex}
%\usepackage{amsmath}
%
%% fonts
%%\usepackage{mathspec}
%%\setmainfont[Mapping=tex-text]{Linux Libertine}
%%\setmathfont(Digits,Greek,Latin){Linux Libertine}
%%\usepackage{microtype}
%%\usepackage{coptic}
%
%
%% tables and figures
%\usepackage{booktabs}
%\usepackage{graphicx}
%\usepackage{floatrow}
%\usepackage{multirow}
%\usepackage{enumitem}
%\newfloatcommand{capbtabbox}{table}[][\FBwidth]
%\setlist{noitemsep}
%
%% Add packages and definitions you want to use here:
%\usepackage{times}
%\usepackage{multirow,sectsty}
%\usepackage{setspace}
%\usepackage{subfigure,graphicx}
%\usepackage{amsmath,amsthm,amsfonts, amssymb}
%\theoremstyle{definition} \newtheorem{definition}{Definition} 
%\usepackage{linguex}
%% \usepackage{betababel}
%\usepackage[english,greek]{betababel}
%\usepackage{tikz-qtree}
%\usepackage{tikz}
%\usetikzlibrary{arrows,automata,chains,matrix,positioning,scopes}
%
%\usepackage[normalem]{ulem}
%
%\usepackage{pdfpages}
%
%\usepackage{natbib}
%
%\usepackage{epigraph}
%\usepackage{hyperref}
%
% \usepackage[only, llbracket,rrbracket]{stmaryrd}
% \newcommand{\sem}[1]{\ensuremath{\{ #1 \} }}
% \newcommand{\pair}[1]{\ensuremath{\langle #1 \rangle}}
% \newcommand{\la}{\ensuremath{\lambda}}
% \newcommand{\inter}[1]{\ensuremath{\llbracket#1\rrbracket}}
%
%\newcommand*\circled[1]{\tikz[baseline=(char.base)]{
%            \node[shape=circle,draw,inner sep=2pt] (char) {#1};}}
%
%
%\newcommand{\comm}[1]{}
%\long\def\symbolfootnote[#1]#2{\begingroup%
%\def\thefootnote{\fnsymbol{footnote}}\footnote[#1]{#2}\endgroup}
%
%\begin{document}

%\setcounter{chapter}{0}
\chapter{Introduction}
\label{ch:introduction}

\section{Diachronic Patterns}\label{sec:History}
The historical trajectory of ditransitive passives and their theme marking has been extensively discussed \citep{Allen.1999,Falk.1997,Platzack.2005,Arnadottir.2013}. There are two main changes that occur: (1) change in passivization strategies (i.e. whether recipient passive, theme passive or both are available), (2) change in case marking patterns within a particular strategy. The first reflects changes in the purely syntactic parameters, while the second relates to morphosyntactic parameters. Each change can be explained as a local simplification of one of the parameter values (i.e. moving from the more complex parameter value to the simpler parameter value). 

Recipient passivization derives from syntactic simplification. Since theme passivization requires either A-scrambling or relativized locality, simplification in those parameters leads to recipient passivization. With the A-scrambling parameter, children need to learn that A-scrambling is a part of the language to be acquired. If there is insufficient evidence for A-scrambling, the child will fail to learn about the existence of that operation, which would lead to a grammar where only the base generated recipient--theme order is generated. 

Without A-scrambling, unrelativized locality leads to recipient passivization. For any case of relativization, children would need to learn the additional restriction on locality. As in any case of learning language specific material, there is a risk of language learners failing to acquire it. An additional degree of complexity in T-argument interactions is the possibility of splitting the properties of the subjecthood and case assignment/agreement operations. This split is also capable of being targeted for local simplification.

There are two types of morphosyntactic complications. The first is the issue of double syntactic case marking. When both dative and nominative syntactic case is assigned to recipient subjects and only one of the syntactic cases is realized morphologically, there is only indirect evidence for the existence of both syntactic cases. Over time, there should be a tendency for double case marking to be lost (i.e. a move toward nominative themes).

The second type of complication is a split in the semantic/pragmatic meaning of nominative case. Under this analysis, nominative case is the realization of a syntactic relationship between a DP and T. The canonical relationship is subjecthood. Nominative objects reflect a complication in the semantic/pragmatic meaning of nominative case which no longer merely reflect subjecthood. 

As mentioned in the Introduction, there is no optimal solution to this set of parameters. Optimization in the syntactic domain (no A-scrambling and no relativised locality on T) leads to recipient passivization, which requires morphosyntatic complexity in the form of double syntactic case marking. Optimization in the morphosyntactic domain (no double syntactic case marking and no nominative objects) leads to theme passivization, which requires syntactic complexity (either A-scrambling or relativized locality). 

Given the absence of a globally optimal solution, there is a prediction that language change should occur in a number of locally optimizing cycles, where simplification in one domain ends up creating complexity in another domain. In the next two subsections, I show these processes in work, demonstrating both cases of syntactic simplification (i.e. loss of either A-scrambling or relativised locality) and cases of morphological simplification (i.e. gaining relativised locality leading to theme passivization).

The loss of A-scrambling can be seen in the histories of Icelandic and Faroese. Modern American English shows an example of the loss of relativized locality, reflected by a rise in recipient passivization with an accusative theme. Recipient passivization allows for a simpler type of locality (find the highest DP), but requires multiple case assignment on the recipient.

The second type was already noticed by \cite{Arnadottir.2013}. Nominative case assignment and subject raising are constrained to share locality properties (i.e. subjects should be nominative or nominative elements should be subjects), as seen with experiencer predicates and monotransitive passives. When combined with a pressure against double case marking, this shared locality requirement leads to a situation with almost exclusively theme passivization. This situation, which obtains in Faroese and Modern Swedish (for at least one class of recipient verbs), is advantageous in so far as it allows a biunique relationship between nominative case and subject position. However, it requires a more complex version of locality in order to guarantee the relationship.

One question that remains is the rarity of oblique subject marking. There is an empirical tendency to prohibit clauses without morphologically realized nominative case (e.g. dative recipients and accusative themes). Under the current analysis, given a doubly syntactic case marked subject (i.e. with both dative and nominative case), there is no a priori reason to prefer the realization of nominative. One of the goals of the dissertation is to refine the morphological side of the analysis to account for this empirical tendency.

\subsection{Syntactic Simplification}
\subsubsection{Loss of A-scrambling}



Modern Icelandic provides a good example of the loss of A-scrambling. Based on the Icelandic Parsed Historical Corpus (IcePaHC, \cite{Wallenberg.2011}), Old Icelandic (c. 1200--1500) had robust evidence of A-scrambling. Both the recipient--theme and theme--recipient word orders were robustly possible at the beginning of the period. Throughout the Old Icelandic period, the theme--recipient order decreases. Early Modern Icelandic (c. 1500--1600) already had the recipient-theme word order almost exclusively (at a rate of $\sim$90.68\%) in all environments except one. The only environment, where the theme-recipient word order remained a viable possibility, was where the recipient was a full noun phrase and the theme a pronoun. Starting around 1800 this condition began to change, so that by the present day it has joined all the other conditions in requiring the recipient-theme word order (see \autoref{fig:modscanactgraph}).

\begin{knitrout}
\definecolor{shadecolor}{rgb}{0.969, 0.969, 0.969}\color{fgcolor}\begin{figure}[ht!]


{\centering \includegraphics[width=\linewidth]{figure/lang--miact-graph} 

}

\caption[LOESS lines for Modern Icelandic active sentence from IcePaHC]{LOESS lines for Modern Icelandic active sentence from IcePaHC.\label{fig:modscanactgraph}\label{fig:miact-graph}}
\end{figure}


\end{knitrout}


\subsubsection{Loss of Relativised Locality for Subjecthood}\label{sec:lossofrlforsubject}



The loss of relativized locality for subjecthood can be seen both in changes in American English\footnote{American English began to diverge from British English as soon as colonists arrived in the early 17th century. Large scale production of American works, however, only begins to occur during the early 19th century. The data discussed here comes from the Corpus of Historical American English (COHA, \cite{Davies.2010}), which covers the period from 1810 till 2009. The earlier part of this period has less data than the latter parts.} and Modern Icelandic, with an increase in the number of recipient subjects. There are two pieces of evidence for the loss of relativized locality for subject raising. The first is a change in the allomorphy of dative case in theme passivization. For Early Modern English (as well as Early American English), recipients in theme passives showed variable dative case realizations. By the 21st century, bare recipients in the theme passive become extremely rare in the corpus (see \autoref{fig:am--act}), and are judged ungrammatical by many modern speakers.

\begin{exe}
\ex
\begin{xlist}
\ex Early American English: The books were given (to) me.
\ex Modern American English: The books were given $^{*}$(to) me.
\end{xlist}
\end{exe}

As discussed in \autoref{sec:shiftloc}, the optionality of the earlier system reflected variation in the derivation of the theme passive. Examples with 'to' were derived by A-scrambling of the theme and subsequent raising to subject position, with the trace of the theme intervening between the verb and the recipient blocking the licensing of null dative case. Examples without 'to' were derived by directly raising the theme across the recipient, relying on relativized locality to bypass the syntactically case marked recipient and select the syntactically caseless theme. In Modern American English, the second option has been lost, with A-scrambling being the only means of generating theme passives.











\begin{figure}
{\centering \includegraphics[width=\linewidth , height=3in]{figure/lang--am--to--graph} 

}
\caption[caption]{LOESS regression lines for 'to'-use rates for American English sentences with theme passives: \textcolor{col4}{It/the book was given (to) John/him}.}
\label{fig:am--act}

{\centering \includegraphics[width=\linewidth , height=3in]{figure/lang--am--pas--graph} 

}
\caption[caption]{Multinomial LOESS regression lines for American English passive sentences: \textcolor{col1}{To John/him was given the book/it}, \textcolor{col2}{John/he was given the book/it}, \textcolor{col3}{The book/it was given to John/him}, \textcolor{col4}{The book/it was given John/him}.}
\label{fig:am--pas}

\end{figure}

The second piece of evidence about relativized locality, comes from the rate of theme passivization vis-a-vis recipient passivization. In Early American English, theme passivization was essentially categorical. Through the 200 year history of American English, theme passivization (both with and without 'to') gets replaced by recipient passivization. This loss of theme passivization is exactly what is expected to occur when relativized locality is lost. The highest argument raises in the passive, which is the recipient in the base order and the theme after A-scrambling. The recipients are marked with nominative case because of a change in English during the Early Modern English period (see \autoref{sec:lossrlcase})

Before discussing the trajectory of change in Modern Icelandic, it is necessary to establish what the state of affairs was in Old Icelandic, which is a matter of some dispute. \cite{Bardal.1997,Bardal.1998,Bardal.2000,Bardal.2001} argues that Old Scandinavian shows the same constructions as Modern Icelandic, and therefore should be considered to have oblique subjects (Her focus is on experiencer verbs, instead of recipient verbs). \cite{Rognvaldsson.1991}, again about dative experiencers, states: ``I have not found a single case where inverted oblique subject-like DPs follow the main verb, they always immediately follow the finite verb.'' The Old Scandinavian facts seems to indicate that, as in Modern Icelandic, the standard V2 subject position exists (i.e. immediately before the finite verb, if there is no fronted element, and immediately after the finite verb if some element has been fronted) and can be filled with oblique subjects. However, I have found at least one example in IcePaHC where there is nothing in the subject position. The following (passive) example has a fronted temporal adverb, and then nothing in between the finite verb and the main verb (i.e. in the subject position). Both of the objects are presumably still in-situ in the verb phrase.
\begin{exe}
\ex \gll Nú er sagt konungi dráp ármannsins\\
now was said king.DAT killing.NOM/ACC Herman.GEN.his.GEN\\
\trans `Now the king was told of the killing of Herman. (IcePaHC, 1275.MORKIN.NAR-HIS,.667)'
\end{exe}
Far more examples, however, have a nominal element which does occur in one of the subject positions. 
\begin{exe}
\ex
\begin{xlist}
\ex \gll Málróf er gefið mörgum\\
Málróf is given much\\
\trans `Málróf is given much.'
\ex \gll en annan dag jóla var hann í jörð lagður\\
and the day Christmas was he in earth laid\\
\trans `and on the day after Christmas he was laid in the earth. (IcePaHC, 1210.THORLAKUR.REL-SAG,.489)'  
\end{xlist}
\end{exe}
 \cite{Kinn.2010} suggests, discussing Old Norwegian, that in the early period, the subject position existed, but was not obligatorily filled. It is clear that if the position existed, oblique DPs were able to fill it. The same situation obtains in IcePaHC, where most clauses have subjects (including a number of oblique subjects), but some small number do not. This also supports the comparative evidence, since this predicts that ever older Scandinavian would be like Standard German. which has never had an obligatory subject position (see \autoref{sec:subjpos}).

There are not enough passive tokens in IcePaHC to see the details of any change. However, when collating the data from the entire Old Icelandic period, it can be seen that recipient passivization seems to have been the preferred strategy across the board (see the following table\footnote{The theme empty category includes both wh-moved elements and pro-dropped elements, either under pragmatic conditions or under conjunction. Since Modern Icelandic has an explicit subject position, it is possible to see if the recipient is a subject or not, even in the absence of the theme (i.e. is the recipient the clause initial element, or in between the finite verb and the passive participle).}). The predominance of recipient passivization can be seen, especially when the recipient is a pronoun.

% latex table generated in R 3.1.0 by xtable 1.7-3 package
% Fri Feb 06 21:51:42 2015
\begin{table}[ht]
\centering
\begin{tabular}{rlll}
  \hline
 & Theme Empty & Theme Noun & Theme Pronoun \\ 
  \hline
Recipient Noun & 0\% (3) & 76.2\% (21) & 66.7\% (3) \\ 
  Recipient Pronoun & 100\% (13) & 90.5\% (21) & 100\% (4) \\ 
   \hline
\end{tabular}
\caption{Recipient Passivisation Rates in Old Icelandic (Total Token Numbers in Parentheses)} 
\end{table}



For Modern Icelandic (1500--), the recipient passive is the predominant form in most conditions. While the number of tokens is small, the condition with a theme pronoun and a recipient full noun phrase is the only condition that seems to prefer the theme passive. The passivization data is consonant with the pattern of change in the active, where in the Early Modern Icelandic period, this condition was the main holdout for the theme-recipient.

% latex table generated in R 3.1.0 by xtable 1.7-3 package
% Fri Feb 06 21:51:42 2015
\begin{table}[ht]
\centering
\begin{tabular}{rlll}
  \hline
 & Theme Empty & Theme Noun & Theme Pronoun \\ 
  \hline
Recipient Noun & 85.7\% (7) & 81.2\% (16) & 16.7\% (6) \\ 
  Recipient Pronoun & 85\% (20) & 96.4\% (28) & 66.7\% (9) \\ 
   \hline
\end{tabular}
\caption{Recipient Passivisation Rates in Modern Icelandic (Total Token Numbers in Parentheses)} 
\end{table}



While the evidence is tentative, Modern Icelandic may have moved towards losing relativized locality for subject raising just like American English.  Unlike American, however, Icelandic does not have A-scrambling, so loss of relativized locality would mean a complete loss of theme passivization.

\subsubsection{Loss of Relativised Locality for Nominative Case Assignment/Agreement}\label{sec:lossrlcase}

As discussed in \autoref{sec:dattonom}, one of the major signs of dative to nominative raising is a change in the behavior of passives of monotransitives with dative objects. In English \citep{Allen.1999}, Swedish \citep{Falk.1997}, and Faroese \citep{Arnadottir.2013} these constructions moved from having oblique subjects marked with morphological dative case to having nominative subjects. In all three languages, the change from dative-to-nominative subjects in monotransitive passives happened independently of a rise in recipient passivization with ditransitives. It also seems to be connected to a change in the semantic/pragmatic properties of nominative case, where there becomes a biunique relationship between nominative case and subjecthood (all subjects have morphological nominative case and all DPs marked with morphological nominative case are subjects), which can be satisfied by theme passivization (see \autoref{sec:morphsimp}).

These changes have a shared property where double syntactic case marking is avoided, when possible. For monotransitive passives, there is only one nominal argument, so assuming that there is an obligatory subject position and that an expletive is not licensed, the dative argument must raise to subject position. The creation of an obligatory subject position is at least sometimes connected to introduction of an \textbf{unrelativized} locality option. In many cases, however, this option seems to be restricted to cases of last resort. If relativized locality does not lead to a crashed derivation (i.e. if there is a caseless argument in the clause), then relativized locality is used. This is similar to the situation of Person Case Constraint effects, where a violation only develops depending on the properties of both of the arguments of a ditransitive. \cite{Richards.2008,Nevins.2011} both have proposed distinct formal mechanisms for capturing these sorts of cases, where properties of multiple arguments need to be considered before continuing with a derivation. 

American English showed an example of where relativized locality was lost for subject raising, and nominative case assignment followed. Current changes in Icelandic and Faroese seem to show the loss of relativized locality for nominative case assignment in a grammar that had a split between subjecthood locality and case assignment locality (i.e. languages with oblique recipient subjects).

There also seems to be a change that is beginning among some Icelandic speakers to lose relativized locality for nominative case assignment as well, as can be seen by having the theme occur in accusative case with a dative marked recipient subject. A fairly substantial number of speakers are marking the theme with the accusative case in the recipient passive \citep{Arnadottir.2013}, although with dative case realized on the recipient and default agreement.\footnote{My understanding is that these examples are distinct from the so--called "New Passive", in which the recipient and theme remain in their base position, and the subject position is filled by an expletive. In the constructions discussed here, the dative--marked recipient remains the clausal subject.}
\begin{exe}
\ex[\%] {\gll mér var gefið bílana \\
me.DAT was.3SG given.DEF cars.the.M.PL.ACC \\
\trans `I was given the cars.' }
\end{exe}
As mentioned above, the set of possible case/agreement patterns are unclear. In the dative subject/accusative object constructions, at least some speakers allow for agreement with surface accusative objects, other speakers allow at least number agreement with the recipient, and the remaining speakers prefer default agreement (3.sg.masc.). 

\cite{Eyorsson.2012} reports a similar change in Faroese, from the results of a grammatical survey of native Faroese speakers looking into possible case marking patterns with recipient passives. Half of the speakers rejected even the canonical case marking pattern (dative recipient, nominative theme). The difficulty in collecting this data derives from the fact that recipient passivization is nearly ungrammatical in Faroese, with theme passivization being the dominant strategy.
\begin{exe}
\ex 25.8\% acceptance vs. 50\% rejection\\
\gll Gentuni bleiv givin ein telda.\\
the.girl.DAT was given.NOM a.Nom computer.NOM\\
\trans `The girl was given a computer. \cite[ex 45a]{Eyorsson.2012}'
\end{exe}
More people rejected a version with an accusative theme and a dative subject, with an accusative passive participle. It would be interesting to see how subjects would react to a nominative participle (i.e. showing default agreement).
\begin{exe}
\ex 17.7\% acceptance vs. 61.3\% rejection\\
\gll Gentuni bleiv givið eina teldu.\\
the.girl.DAT was given a.ACC computer.ACC\\
\trans `The girl was given a computer. \cite[ex 45b]{Eyorsson.2012}'
\end{exe}
Even fewer people accepted marking the recipient with nominative case, and having it trigger agreement on both the finite verb and the participle. While this was clearly marginal, more than 1 out of 10 speakers still found this construction grammatical.  This may indicate that the change in case marking that has been seen with Faroese experiencers (dative $\rightarrow$ nominative) is also happening with passive recipient subjects. It would be hard to study this in a corpus, however, due to the overwhelming preference for theme passivization.
\begin{exe}
\ex 14.5\% acceptance vs. 77.4\% rejection\\
\gll Gentan bleiv givin telduna\\
the.girl.NOM was given.NOM the.computer.ACC\\
\trans `The girl was given a computer. \cite[ex 48b]{Eyorsson.2012}'
\ex[*] {\gll Gentan bleiv givin teldan.\\
the.girl.NOM was given.NOM the.computer.NOM\\
\trans `The girl was given a computer. \cite[ex 48a]{Eyorsson.2012}'}
\end{exe}

\subsection{Morphosyntactic Simplification}\label{sec:morphsimp}
Swedish and Faroese both show examples of simplification in the morphological domain, which led to complications in the syntactic domain. As discussed in \autoref{sec:lossofrlforsubject}, Old Scandinavian seems to have had Oblique Subjects, which means that unrelativized locality must have been a possibility, at least for subject raising. After Faroese and Swedish gain a biuniqueness relationship between subjecthood and nominative case, theme passivization becomes more likely/obligatory, since it guarantees a nominative subject, without requiring double syntactic case marking.

Swedish shows a lexical split in the behavior of ditransitive verbs. With simple verbs (i.e. verbs without preverbs, unlike \emph{er--bjöds}), both recipient--theme and theme--recipient word orders are possible in the active. In the theme-recipient word order, however, the recipient must be marked with the preposition \emph{til} `to'. As far as I can tell, this also applies if either argument is a pronoun, although I have not found that to be explicitly mentioned anywhere.
\begin{exe}
\ex 
\begin{xlist}
\ex \gll Jag gav Johan en bok.\\
I gave John a book.\\
\trans `I gave John a book \citep{Holmberg.1995}.'
\ex[*] {\gll Jag gav en bok Johan.\\
I gave a book John.\\
\trans `I gave a book to John.'}
\ex \gll Jag gav en bok til Johan.\\
I gave a book to John.\\
\trans `I gave a book to John \citep{Holmberg.1995}.'
\end{xlist}
\end{exe}

With these verbs, only the theme passive is possible, and only when the recipient is marked with \emph{til} `to' \citep{Anward.1989,Lundquist.2006}.

\begin{exe}
\ex 
\begin{xlist}
\ex[\%] {\gll Pelle gavs ett äpple.\\
 Pelle gave.PASS an apple.\\
\trans `Pelle was given an apple \citep{Anward.1989,Lundquist.2006}.'}
\ex[\%] {\gll Ett äpple gavs Pelle.\\
 An apple gave.PASS Pelle.\\
\trans `An apple was given to Pelle \citep{Anward.1989,Lundquist.2006}.'}
\ex[\%] {\gll det gavs Pelle en vacker bok.\\
it gave.PAS Pelle a beautiful book.\\
\trans `There was given Pelle a beautiful book. \citep{Anward.1989}}
\ex \gll Ett äpple gavs till Pelle.\\
 an apple gave.PASS to Pelle.\\
\trans `An apple was given to Pelle \citep{Anward.1989,Lundquist.2006}.'
\end{xlist}
\end{exe}

As discussed in \autoref{sec:lossrlcase}, recipient passivization with either dative or nominative recipients is essentially ungrammatical for many speakers of Modern Faroese. Since Old Norse and earlier stages of Faroese permitted recipient passivization, this must be a new development in Faroese. 

In Swedish, simplification of the morphosyntax (i.e. avoidance of double case marking and enforcing the biuniquness of nominative case and subjecthood) is enforced by obligatory A-scrambling. In Faroese, the same optimization pressures led to enforced relativized locality for both subjecthood and case assignment/agreement. In both cases, this leads to a surface condition in which theme passivisation is obligatory.

\section{Conclusions and Further Work}
I proposed that three parameters are able to capture the diversity among Germanic recipient ditransitives. These parameters capture the variation seen among the Modern Germanic languages. They are also able to account for the trajectories of change seen in the history of the Germanic languages in terms of local simplification.

I have already mentioned a number of open questions, which I hope to address in the dissertation itself. One of the biggest questions is expanding from recipient ditransitives case to benefactives. This will only work for non-Scandinavian varieties, since the Scandinavian languages do not allow benefactive DPs \citep{Lundquist.2013b}. One of the interesting properties of benefactives that make them distinct from recipients, is that they can be realized with prepositional marking in all languages (unlike recipients which only allow prepositional marking in languages with dative shift). 

I also plan to quantitatively replicate the descriptive results from \cite{Allan.1995}, using the Parsed Corpora of Historical English (and if possible the Early English Books Online). This replication should solidify the empirical facts about other dative constructions, especially the passives of dative-object monotransitives. A more complete set of English dative facts will aid in constructing a coherent picture of the syntactic and morphological properties of English dative case. Once the English facts are confirmed, it will be possible to integrate them with the extant research about the properties of datives in the other Germanic languages.

In the dissertation, I will also deal with non-canonical passives in the Germanic languages (e.g. Icelandic -s- passives, and German/Dutch kriegen passives). These passives often show dative-to-nominative raising in languages where normally dative case is preserved in the passive (see \cite{Alexiadou.2013,Alexiadou.2013b} and citations therein). 

Other languages than English also have dative shift. Norwegian is described as having a grammar that may be similar to Early Modern English. Danish has a grammar that has been described as similar to Modern American English. As already mentioned, Swedish ditransitive verbs show a lexical split. Dutch ditransitives have the property where prepositional marking is always licit, but bare marking is \textbf{also} possible in certain configurations. Since Dutch is an OV language, verb adjacency will probably not suffice to explain the Dutch data. However, Dutch may be parallel to the Early Middle English situation, where variation is a reflex of grammar competition during a change in progress. In each case, I plan to investigate whether the morphological analysis of dative shift introduced here can be applied to these cases. I will also present in the dissertation a discussion of the historical origins of Dative Shift.

Also the Swedish lexical split between verbs with and without preverbs needs further examination (although \citep{Lundquist.2004}). Simple verbs have already been discussed, and seem to behave similarly to Modern American ditransitives (with obligatory theme passivization). Complex verbs (i.e. verbs with preverbs) do not allow prepositional marking in the active, and allow any type of passive. The behavior of recipients in these constructions seems similar to symmetric passivization in Bantu language with and without applicatives \citep[and others]{Pylkkanen.2001,McGinnis.2001,McGinnis.2001b,Baker.2012}. In the dissertation, I will pursue the connection between Swedish preverbs and Bantu applicatives.

For all of these questions and for the data given in this proposal, I plan to use available annotated corpora, internet searches, as well as a grammaticality surveys. With Modern Icelandic (and possibly Faroese), I hope to confirm recent grammaticality judgments with corpus data. After data collection, I will update my analysis to account for the new empirical generalizations, and provide a formalization of the new syntactic and morphological analyses discussed here in the DM/Minimalist framework.
