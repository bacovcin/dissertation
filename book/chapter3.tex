\chapter{Passive Syntax of Recipient Ditransitives}
\section{Introduction}
This chapter analyses the passivisation of recipient ditransitives. Looking at passivisation sheds light on the internal properties of subjecthood. \cite{McCloskey.1997} describes how one of the major innovations of the generative program was to remove subjecthood as a primitive notion, instead associating different properties of subjecthood with distinct structural positions. The two properties focused on here are: (a) the nature of the higher subject position (e.g., spec-TP) and (b) the assignment of nominative case and triggering of subject agreement on the finite verb. Ditransitive passives show how arguments are chosen for the assignment of these two properties, since multiple arguments are available for selection.

Similar to \cite{Platzack.2005}, I propose a theory that unites the two main theories of argument selection in passivisation, namely: case--based theories \citep{Larson.1988,Baker.1988,Pesetsky.1996,Holmberg.2001} and locality based theories \citep{Falk.1990,Holmberg.1995,McGinnis.1998,Anagnostopoulou.2003}.  Case based theories assume that only non-inherently case marked elements (or direct objects instead of indirect objects) are available to receive subject properties. The strongest version of case-based theories is impossible given the possibility of oblique subjects (see \citealt{Zaenen.1985} and below), which has led to a general rise in prominence of locality-based theories.

Locality-based theories state that only the structurally highest DP is available to receive subject properties. In Chapter \ref{ch:active}, I concluded that recipients are base generated higher than the theme, which means that the locality approach predicts that, baring intervening factors, the recipient should always become the subject (recipient passivisation). However, among Germanic languages, theme passivisation (where the theme becomes the subject) is available indicating mechanisms for obviating the locality violation. These mechanisms will be discussed below.

This chapter will start by analysing recipient passivisation. Since (as argued in Chapter \ref{ch:active}) the recipient always receives dative case, which is represented by a KP, full recipient passivisation (with a nominative recipient) requires dative--to--nominative conversion. Building on the analysis of \cite{Alexiadou.2014}, I propose that dative--to--nominative conversion involves incorporation of the K head into a verbal element, which turns the recipient into a bare DP and makes it available for structural case assignment. This analysis follows suggestions in the KP literature that the difference between inherent/lexical case and structural case is the presence of the KP layer \citep{Bayer.2001}. Evidence for the incorporation analysis will be brought from recipient passives in German, Dutch and Swedish. In the following subsection, I discuss oblique subjects in Icelandic and the evidence they provide for separating the movement to subject position and nominative case assignment.

The second section focuses on theme passivisation. I show that there are two mechanisms by which the locality constraint can be violated, namely: (a) restriction of case assignment/subject movement to bare DPs or (b) movement of either the recipient or the theme. The first mechanism is a consequence of the K-incorporation analysis for recipient passivisation. If K-incorporation is unavailable, then the recipient is not a valid target for nominative case assignment. I give evidence for the following outcomes in the situation in which a movement strategy is not employed: (a) direct theme passivisation (i.e., selection of the theme for subject properties in its base merged position) or (b) defective intervention (i.e., failure to passivise).

\section{Recipient Passivisation}
In this dissertation, recipient passivisation is defined as cases where the recipient is in the higher subject position (i.e., spec-TP). There are two sub-cases of this situation, which will be addressed in turn. The first (dative-to-nominative raising) is a case where the recipient receives nominative case and has all subject properties. The second (oblique subjects) is a case where the subject properties are split with the recipient occupying the higher subject position, but the theme receives nominative case.

\subsection{Dative-to-Nominative Raising}

As argued for in Chapter \ref{ch:active}, all recipients are dative case marked (i.e., all recipients have a dative KP layer). Therefore, any case of a nominative recipient is a case of dative-to-nominative conversion. As will be shown below, this property can be seen on the surface in a number of Germanic languages (namely Faroese, Halsa Norwegian, and Standard German). 

Faroese\footnote{Faroese is currently changing from having oblique subjects like Icelandic (discussed below) and having dative-to-nominative raising. The data presented below are from the speakers that have adopted the new grammar with dative-to-nominative raising (see \cite{Eyorsson.2012} for a discussion of this change and survey data attesting to the existence of this sub-population).} and Halsa Norwegian both show the availability of dative-to-nominative conversion, although they do not elucidate the mechanism. Both languages have a clear morphological distinction between dative and accusative case.

\begin{exe}
	\ex Faroese:
		\begin{xlist}
			\ex[ ]{\gll Teir góvu \textbf{gentuni} telduna \\
				they gave \textbf{girl-the.DAT} computer-the.ACC \\
			            \trans `They gave the girl the computer.'}
				    \ex[*]{\gll Teir góvu \textbf{gentuna} telduna \\
				they gave \textbf{girl-the.ACC} computer-the.ACC \\
			    \trans `They gave the girl the computer.'}
		\end{xlist}
	\ex Halsa Norwegian:
	\begin{xlist}
		\ex \gll Ho erta \textbf{katt\aa} \\
		she teased \textbf{cat.DEF.ACC} \\
			\trans `She teased the cat.'
			\ex \gll Ho ga \textbf{katt\aa} inn mat \\
			she gave \textbf{cat.DEF.DAT} food \\
			\trans `She gave the cat food.'
	\end{xlist}
\end{exe}

Both languages also allow the dative argument to surface as nominative in the passive. Oblique subjects (of ditransitive passives) are marginal/ungrammatical \citep{Eyorsson.2012}.

\begin{exe}
	\ex Faroese:
	\begin{xlist}
		\ex[ ]{\gll Gentan bleiv givin telduna\\
			    girl-the.NOM was given.NOM computer-the.ACC\\
		    	    \trans `The girl was given the computer.'}
		\ex[??]{\gll Gentuni bleiv givin ein telda\\
			    girl-the.DAT was givn.NOM a.NOM computer.NOM\\
		    	    \trans `The girl was given the computer.'}
	\end{xlist}
	\ex Halsa Norwegian:
	\begin{xlist}
		\ex[ ]{\gll Hainn vart gjevinn ei skei.\\
He.NOM was given a spoon\\
\trans `He was given a spoon.' \cite[ex 50c]{Eyorsson.2012}}
\ex[*]{\gll Hånnå vart gjevinn ei skei.\\
He.DAT was given a spoon\\
\trans `He was given a spoon.' \cite[ex 50c]{Eyorsson.2012}}
	\end{xlist}
\end{exe}

In order to explain how dative-to-nominative conversion occurs, a theory of nominative case assignment needs to be given. \cite{Bayer.2001} and \cite{Asbury.2007} argue that KPs may be restricted to inherent/lexical case, and be absent in DPs marked with structural case. That analysis is adopted here, which means that in order for an element to receive nominative case, it must be a bare DP. While the theme in recipient ditransitives is a DP, the recipient is a KP and thus should be unavailable for nominative case assignment. For it to become available, the KP layer must be removed.

\cite{Alexiadou.2014} propose a mechanism for removing the KP layer, namely K-incorporation (for them P-incorporation). This unites dative-to-nominative conversion with theories of pseudopassivsation, where passivisation of the object of preposition required incorporation of the preposition into the verbal domain \citep{Herslund.1984}. As will be shown below, this incorporation can be seen on the surface in a few cases.

Recipient passivisation is not normally available in Dutch or German, instead the theme must receive nominative case (see below for further discussion of theme passivisation).

\begin{exe}
\ex High German:
\begin{xlist}
	\ex[ ]{\gll Ich glaube, dass \textbf{den} \textbf{Kindern} das Fahrrad geschenkt worden ist.\\
	I beleive that \textbf{the.DAT.PL} \textbf{children} the.NOM bicycle granted become be.3sg\\
\trans `I believe that the children were granted the bicycle.'}
\ex[*]{\gll Ich glaube, dass \textbf{die} \textbf{Kindern} das Fahrrad geschenkt worden sind.\\
I beleive that \textbf{the.NOM.PL} \textbf{children} the.ACC bicycle granted become be.3pl\\
\trans `I believe that the children were granted the bicycle.'}
\end{xlist}
\ex Dutch:
\begin{xlist}
	\ex[ ]{\gll De boeken \textbf{werden} werden haar aangeboden.\\
		the books \textbf{became.PL} her given\\
	\trans `The books were given to her.' \citep[ex. 5b]{Broekhuis.1994}}
	\ex[*]{\gll Zij \textbf{werd} de boeken aangeboden.\\
	she.NOM \textbf{became.SG} the books given\\
	\trans `She was given the books.' \citep[ex. 5c]{Broekhuis.1994}}
\end{xlist}
\end{exe}


However, when the passive auxiliary changes from \textit{werden} `become' to \textit{bekommen}/\textit{krijgen} `get', recipient passivisation becomes obligatory. \cite{Alexiadou.2014} argue that the change in auxiliary is the direct reflection of P-incorporation, i.e., that \textit{werden} is the realisation of the passive on its own, while \textit{bekommen}/\textit{krijgen} is the realisation of the passive with the dative K incorporated. For Standard German, this is a clear case of dative-to-nominative conversion, since dative case is marked on the surface.

\begin{exe}
\ex Standard German:
\begin{xlist}
	\ex \gll dass der Vater \textbf{der} \textbf{Tochter} ein Buch geschenkt hat\\
	that the.NOM father \textbf{the.DAT} \textbf{daughter} a.ACC book sent has\\
	\trans `that the father sent the daughter a book.'
	\ex \gll dass \textbf{die} \textbf{Tochter} von dem Vater ein Buch geschenkt bekommen hat\\
	that \textbf{the.NOM} \textbf{daughter} by the father a.ACC book sent got has\\
	\trans `that the daughter got sent a book by her father \cite[183]{Draye.1996}.'
\end{xlist}
\ex Dutch:
\gll \textbf{Zij} kreeg de boeken (van mij) aangeboden.\\
\textbf{she.NOM} got the books (by me) given\\
\trans `She was given the books (by me).' \citep[ex. 7]{Broekhuis.1994}
\end{exe}

For German and Dutch, there is evidence that the \textit{bekommen}/\textit{kreign} passive is actually a passive construction. This evidence comes from the availability of by-phrases (as seen above) and productivity \citep{Broekhuis.1994}. In Dutch, the construction can be productively used with almost all verbs that assign a recipient or addressee theta role. The only exception is the verb \textit{geben} `give', which \cite{Broekhuis.1994} argue is ruled out on pragmatic grounds, since `get given' is pleonastic for `get'.

Another case of overt incorporation can be seen in Swedish. \cite{Hersulnd.1984} argued that K-incorporation for pseudopassivisation in Scandinavian languages appears as particle verbs rather than K-stranding as in English. 

\begin{exe}
	\ex Danish:
	\begin{xlist}
		\ex[ ]{\gll Revisionen blev \textbf{p\aa begyndt} i maj\\
		revision-the was \textbf{on-begun} in May\\
		\trans `The revision was begun in May'}
		\ex[*]{\gll Revisionen blev \textbf{begyndt} \textbf{p\aa} i maj\\
		revision-the was \textbf{begun} \textbf{on} in May\\
		\trans `The revision was begun in May'}
	\end{xlist}
	\ex English: 
	\begin{xlist}
		\ex[*]{The bed was inslept.}
		\ex[ ]{The bed was slept in.}
\end{xlist}
\end{exe}

I showed in Chapter \ref{ch:active} that Swedish shows a split between ditransitive verbs with and without particles. There, I suggested that the particle verbs represented the incorporation of dative case into the verb. This explanation is consonant with the Swedish passivisation data; only verbs with particles allow recipient passivisation (see below for theme passivisation strategies in Swedish). Recipient passivisation is not available for non-particle verbs \citep{Lundquist.2006}.

\begin{exe}
	\ex Modern Swedish:
	\begin{xlist}
		\ex[ ]{Particle Verb:
		\gll Han erbjöds ett nytt jobb\\
			he.NOM offered.PASS a new job\\
			\trans `He was offered a new job \citep{Anward.1989, Lundquist.2006}.'}
		\ex[*]{Non-Particle Verb:
		\gll Pelle gavs ett äpple\\
			Pelle gave.PASS a apple\\
			\trans `Pelle was given an apple \citep{Anward.1989, Lundquist.2006}.'}
\end{xlist}	
\end{exe}


Most of the Germanic languages do not show any overt signs of K-incorporation (including Faroese and Halsa Norwegian discussed above). Given the morphological description of dative case realisation discussed in Chapter \ref{ch:active}, this is not surprising. Most of these langauges (e.g., Danish, Standard Norwegian and English) seem to share the distribution of null dative case realisation with English (namely the null realisation is restricted to contexts locally adjacent to the verb). When the K-head incorporates, it is maximally adjacent to the verb. Thus, a null realisation is expected.

\begin{exe}
	\ex English: He was K=$\emptyset$-given \sout{he} the ball.
	\ex Standard Norwegian:
	\gll Han vart K=$\emptyset$-gitt \sout{hann} ein medalje\\
	he.NOM was given \sout{he.NOM} a medal\\
	\trans `He was given a medal.'
	\ex Danish:
	\gll Han blev K=$\emptyset$-tilbudt \sout{hann} en stilling\\
	he.NOM was offered \sout{he.NOM} a job\\
	\trans `He was offered a job.'
\end{exe}

This K-incorporation process seems to be sensitive to OV vs VO word order, a generalisation observed in \cite{Sprouse.1995}. In languages like Dutch and German with OV word order, K-incorporation happens with the auxiliary, which is the element to the left of the recipient, and thus recipient passivisation is restricted to cases with a different auxiliary. In VO languages, like Swedish, the verb is the element to the left of the recipient, and thus recipient passivisation is restricted to particle verb cases. In many of the languages, K-incorporation is invisible, since the K element has a null realisation.

\subsection{Oblique Subjects}
The previous subsection dealt with cases in which K-incorporation occurred. In that situation, the highest argument (i.e., the recipient) was available both for movement to subject position and nominative case assignment. Most of the rest of this chapter will focus on cases where K-incorporation does not occur. In these situations, the recipient is not available for nominative case assignment. This subsection describes cases where the two subjecthood properties split: the recipient moves to a higher subject position (oblique subject) and the theme receives nominative case (nominative object). This split can be encoded in the featural content of T, where the T head that licenses subject properties has the movement and case assignments distinct. See the discussion of theme passivisation below for further discussion of how the assignment of nominative case to the theme proceeds.

\cite{Zaenen.1985} gives the classic presentation of the evidence in Modern Icelandic for oblique subjects. In Icelandic, only subjects can occupy the post-finite verb position.

\begin{exe}
\ex Topicalization
\begin{xlist}
	\ex \gll Refinn skaut \textbf{Ólafur} með  þessari byssu.\\
	fox.DEF.ACC shot \textbf{Olaf.NOM} with this shotgun\\
\trans `The fox, Olaf shot with this shotgun \citep[ex. 19a]{Zaenen.1985}.'
\ex[*]{\gll Með  þessari byssu skaut \textbf{refinn} Ólafur.\\
	with this shotgun shot \textbf{fox.DEF.ACC} Olaf.NOM\\
\trans `The fox, Olaf shot with this shotgun \citep[ex. 19b]{Zaenen.1985}.'}
\end{xlist}
\ex Direct Question
\begin{xlist}
	\ex \gll Hafði \textbf{Sigga} aldrei hjálpað Haraldi?\\
	had \textbf{Sigga.NOM} never helped Harald.DAT\\
\trans `Had Sigga never helped Harald \citep[ex. 20b]{Zaenen.1985}?'
\ex[*]{\gll Hafði \textbf{Haraldi} Sigga aldrei hjálpað?\\
	had \textbf{Harald.DAT} Sigga.NOM never helped\\
\trans `Had Sigga never helped Harald \citep[ex. 20c]{Zaenen.1985}?'}
\end{xlist}
\end{exe}

In cases of ditransitive passives, the dative phrase is capable of filling this position patterning with undisputed subjects.

\begin{exe}
\ex Topicalization
\begin{xlist}
	\ex \gll Um veturinn voru \textbf{konunginum} gefnar amb\'{a}ttir.\\
In winter.the were \textbf{king.the.DAT} given slaves.NOM\\
\trans `In the winter the king was given slaves \citep[ex. 47a]{Zaenen.1985}.'
\end{xlist}
\ex Direct Question
\begin{xlist}
	\ex \gll Voru \textbf{konunginum} gefnar amb\'{a}ttir?\\
were \textbf{king.the.DAT} given slaves.NOM\\
\trans `Was the king given slaves \citep[ex. 48a]{Zaenen.1985}?'
\end{xlist}
\end{exe}

Note that in all cases with dative subjects, the verb obligatorily agrees with the nominative object in number only \citep{Arnadottir.2013}. Thus, the finite verb must enter into a relationship with both of the object DPs: the recipient in order to trigger movement to the subject position, and the object in order to assign nominative case and to trigger verbal agreement. The weaker nature of agreement without movement is seen in other languages with both preverbal and postverbal positions for the agreement triggering phrase (e.g. Arabic subject agreement).

\section{Theme Passivisation}
Theme passivisation occurs when the theme is in the higher subject position (i.e., spec-TP). In these case, the theme always receives nominative case (i.e., there are no oblique theme subjects). However, the theme being in subject position is a violation of locality without any intervening opertation, since the recipient is always base generated higher than the theme. This section addresses two mechanisms by which the locality violation can be licensed: case sensitivity (a type of relativised minimality) and movement (of either the theme or the recipient).


\subsection{Case Licensed Locality Violation}
This subsection deals with the situation where the dative case of the recipient does not K-incorporate, oblique subjects are not licensed, and no movement operation has altered the initial structure. In order for oblique subjects to be prevented, movement to subject position in these cases must be restricted to nominative elements (i.e., the two subjecthood properties are linked on T). The recipient intervenes for locality purposes between T and the theme, and both relativised minimality and defective intervention are observed.

Relativised minimality \citep{Rizzi.1990} solves locality violations be restricting the domain of possible targets for the search. The classic case is that arguments in A-bar positions are not in the domain of A-movement operations. In the case of ditransitive passivisation, KPs are not in the domain for an operation that is looking for DPs. Since K-incorporation has not occurred, the recipient is a KP and thus not in the domain of the nominative case assignment operation. The recipient is invisible for the search and the theme can receive nominative case and move to the higher subject position directly from its base merged position. I call this construction, where the theme becomes a subject from its base merged position, \textbf{direct theme passivisation}.

Evidence for direct theme passivisation comes from a number of different Germanic languages. One piece of evidence that nominative case assignment can target the theme in its base merged position comes from German and Dutch. In both of these languages, there is no requirement that the higher subject position be filled \citep{Besten.1990}. Nominative elements in all clauses can stay in their base merged positions. In the passives of ditransitives with the normal passive auxiliary \textit{werden}, only the theme can receive nominative case (for the behaviour with alternative passive auxiliaries, see above). The nominative theme can be in its base merged position, underneath the recipient, suggesting that the nominative case assignment occurred past the recipient, which was invisible since it was a KP.

\begin{exe}
\ex High German:
\gll Ich glaube, dass den Kindern das Fahrrad geschenkt worden ist.\\
I beleive that the.DAT.PL children the.NOM bicycle granted become be.3sg\\
\trans `I believe that the child was granted the bicycle.'
\ex Dutch:
\gll Er werd mij een boek gegeven.\\
There became.3sg me a book given\\
\trans `A book was given to me. \cite[pg 245]{Donaldson.2008}'
\end{exe}

Certain dialects of British English and historical dialects of English provide evidence for direct theme passivisation. In these dialects, theme passivisation can occur with bare recipients (\ref{ex:baretpeng}). In Chapter \ref{ch:active}, I argued that lower copies of movement are able to intervene for determining the realisation of dative K, namely that they prevent the null allomorph from being realised. Thus, the existence of a null allomorph in theme passive contexts must be due to the theme moving to subject position from its base merged position without an intermediate stage of VP-internal scrambling.

\begin{exe}
	\ex English Dialects: 
		\begin{xlist}
			\ex[ ]{\label{ex:baretpeng}The book was given K=$\emptyset$ the man \sout{the book}.}
			\ex[*]{\label{ex:badtpeng}The book was given \sout{the book} K=$\emptyset$ the man \sout{the book}.}
		\end{xlist}
\end{exe}

Icelandic also provides an example of direct theme passivisation. As discussed in Chapter \ref{ch:active}, Icelandic lacks VP-internal scrambling. However, theme passivisation is still a robust possibility in Icelandic. Either the theme is moving directly from its base merged position in the passive, or the passive shows evidence of a covert operation (VP-internal scrambling) that can only feed further transformation, but cannot occur on its own. While such operations have been argued for in the literature \citep[119ff]{Richards.2001}, direct theme passivisation gives a simpler analysis of Icelandic clauses, using only operations that are independently necessary.

\begin{exe}
\ex Icelandic:
\begin{xlist}
	\ex \gll Um veturinn voru \textbf{amb\'{a}ttin} gefin konunginum \sout{amb\'{a}ttin}.\\
	In winter.the was \textbf{slave-the.NOM} given king.the.DAT \sout{slave-the.NOM}\\
\trans `In the winter the slave was given to the king \citep[ex. 47b]{Zaenen.1985}.'
\ex \gll Var \textbf{amb\'{a}ttin} gefnar konunginum \sout{amb\'{a}ttin}?\\
were \textbf{slave-the.NOM} given king.the.DAT \sout{slave-the.NOM}\\
\trans `Was the slave given to the king \citep[ex. 48b]{Zaenen.1985}?'
\ex \gll \textbf{B\'{o}kin} var gefin J\'{o}ni \sout{B\'{o}kin}\\
\textbf{book-the.NOM} was given John.DAT \sout{book-the.NOM}\\
\trans `The book was given to John \citep{Holmberg.1995,Bardal.2001}.'
\end{xlist}
\end{exe}

Swedish verbs without particles (e.g., \textit{gav} `give'), Danish and Modern American English all show a defective intervention effect. In these cases, direct theme passivisation is ungrammatical, since the recipient intervenes between T and the theme. The recipient is not a valid target for subjecthood (since K-incorporation has not occurred), but the recipient is still included in the search domain for the assignment of nominative case (since in these languages relativised minimality with respect to case does not hold). Some movement operation is necessary to allow passivisation in these cases.

\begin{exe}
	\ex Swedish (verbs without particles):
	\sn[*]{
	\gll Ett äpple gavs Pelle.\\
	 An apple gave.PASS Pelle.\\
	 \trans `An apple was given to Pelle \citep{Anward.1989,Lundquist.2006}.'}
	 \ex Danish:
	 \sn[*]{
	 \gll En stilling blev tilbudt ham.\\
A job was offered him.OBL.\\
\trans `A job was offered to him \citep{Falk.1990}.'}
	\ex Modern American English: *The book was given John \sout{the book}.
\end{exe}

\subsection{Movement Licensed Locality Violation}
As already hinted to above, VP-internal scrambling is a straightforward solution to the locality problem. If the theme has moved to be structurally higher than the recipient, then the theme is both available for nominative case assignment and the closest element from a locality standpoint. In English (and other languages with a similar case realisation pattern), this entails that the non-null realisation dative K head be used, since the null allomorph will not be licensed as the copy of the theme will intervene between K and the verb.

\begin{exe}
	\ex English:
	\begin{xlist}
			\ex[ ]{The book was given \sout{the book} K=to the man \sout{the book}.}
			\ex[*]{The book was given \sout{the book} K=$\emptyset$ the man \sout{the book}.}
	\end{xlist}
\end{exe}

VP-internal scrambling solves the locality problem by moving the theme. \cite{Anagnostopoulou.2003} shows that movement of the recipient is also able to obviate locality violations. Germanic languages show two different types of recipient movement. Anagnostopoulou argued that scrambling in Dutch (outside of the VP) is an A-bar operation that makes the recipient invisible for A-movement to subject position (i.e., standard relativised minimality). This type of scrambling can be identified by the placement of the argument to the left of VP-level adverbs (e.g. \textit{waarschijnlijk} `probably').

\begin{exe}
	\ex Dutch:
	\begin{xlist}
		\ex[ ]{\gll dat het boek \textbf{Marie} waarschijnlijk gegeven wordt\\
	that the book \textbf{Mary} probably given was\\
		\trans `that the book was probably given to Mary.'}
		\ex[?*]{\gll dat het boek waarschijnlijk \textbf{Marie} gegeven wordt\\
		that the book probably \textbf{Marie} given was\\
		\trans `that the book was probably given to Mary.'}
	\end{xlist}
\end{exe}

Anagnostopoulou shows that for other languages, e.g., Modern Greek, clitic doubling of the recipient also suffices. For Modern English (and many of the mainland Scandinavian languages), pronoun cliticisation seems to be a sufficient movement operation. Since many of these languages also have direct theme passivisation (see above), only usage data is able to show the existence of a pronoun cliticisation operation. For English dialects in which both direct theme passivisation and cliticisation are available, theme passivisation with bare full noun phrase recipients is rare in corpus data ($\sim$3\%--10\% of all passives). On the other hand, theme passivisation with bare pronominal recipients is common ($\sim$50\%).\footnote{Corpus estimates are drawn from historical data in COHA (1810--2009) \citep{Davies.2010} and the Parsed Corpora of Modern British English (1700--1910) \citep{Kroch.2010}. See the following chapter for a discussion of diachronic patterns.} The difference in usage rates suggests that there may be multiple operations at play (namely a rare direct theme passivisation operation and a more common cliticisation operation). Cliticisation of the recipient removes it from further movement and from being an intervener between T and the theme.

\begin{exe}
	\ex English Dialects (cliticisation): The book was given=me \sout{the book}.
\end{exe}

Another piece of evidence for cliticisation is the availability of theme passivisation comes from languages/dialects in which cliticisation is available, but direct theme passivisation is not. For many modern British English dialects from the Northwest of England (around Manchester and Liverpool), theme passives with bare recipients are only available with pronominal recipients (suggesting that cliticisation is the only available strategy) \citep{Haddican.2010,Myler.2011,Haddican.2012,Biggs.2015}.

\begin{exe}
	\ex
	\begin{xlist}
		\ex[ ]{The book was given me.}
		\ex[*]{The book was given John.}
	\end{xlist}
\end{exe}

Another case can be seen in Swedish. I argued above that verbs with particles in Swedish reflected overt K-incorporation. Once K-incorporation has occurred, the recipient should always raise to subject position (i.e., direct theme passivisation is not available). Cliticisation of pronominal recipients explains why pronominal recipients can stay low in these cases.

\begin{exe}
	\ex \gll Ett nytt jobb erbjöds=honom.\\
A job offered.PASS=him.OBL.\\
\trans `A job was offered to him \citep{Anward.1989,Falk.1990,Lundquist.2006}.'
\end{exe}

Swedish gives further data about the cliticisation process, since theme passivisation with unmarked recipients is \textbf{only} available with particle verbs. This suggests that in Swedish, the cliticisation process is restricted to DP pronouns. Pronouns with the dative KP (i.e., in non particle verbs) are unable to cliticise and thus serve as defective interveners for direct theme passivisation (see previous subsection).

\begin{exe}
	\ex[*]{\gll Ett äpple gavs honom.\\
	 An apple gave.PASS him.\\
	\trans `An apple was given to him \citep{Anward.1989,Lundquist.2006}.'}
\end{exe}

\section{Conclusions}
This chapter analysed passivisation of recipient ditransitives. K-incorporation converted dative recipients into unmarked DPs, licensing dative-to-nominative conversion. This incorporation was seen on the surface in Dutch, German and Swedish. Oblique subjects were analysed by splitting the movement and case assignment properties of T into different searches (with different domains of application). In addition, surface theme passivisation with nominative themes were shown to arise from a number of possible mechanisms for avoiding locality violation, namely: relativised minimality, VP-internal scrambling, and recipient scrambling/cliticisation. Relativised minimality was argued to result in direct theme passivisation, where the theme moved to subject position directly from its base merged position.

%\bibliography{diss}
