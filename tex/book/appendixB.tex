\chapter{Germanic Ditansitive Examples}
This appendix collates all of the examples from the main body of the text by language. At the beginning of each language section, I also list all of the works that I referenced to learn about the behaviour of recipient ditransitives in that language. There are details for each of the languages that did not make it into the broader focus of this dissertation. I urge the reader who is interested in the details of a particular language to consult the listed references.

Languages are grouped by language sub-family: North Germanic (Icelandic, Faroese, Norwegian, Swedish and Danish) and then West Germanic (High German, Yiddish, Dutch, Afrikaans, Frisian, Low German and English).

\section{North Germanic}
\subsection{Icelandic}
\subsubsection{Relevant Citations}
\cite{Haugen.1982,Zaenen.1985,Yip.1987,Falk.1990,Maling.1990,Rognvaldsson.1991,Ottosson.1991,Mrck.1992,Ottosson.1993,Kristoffersen.1994,Sprouse.1995,Holmberg.1995,Rognvaldsson.1996,Bardal.1997,Haugen.1998,Holmberg.1998,Maling.1998,Maling.2001,Holmberg.2002,Eythorsson.2005,Bardal.2000,Faarlund.2001,Bardal.2001,Bardal.2001b,Askedal.2001,Dehe.2004,Bardal.2006,Bardal.2007,Thrainsson.2007,Jonsson.2009b,Wallenberg.2011,Norris.2012,Eyorsson.2012,Sigursson.2012,Sigursson.2012b,Alexiadou.2013,Arnadottir.2013,Lundquist.2013,Lundquist.2013b,Alexiadou.2014}
\subsubsection{Active Data}
\begin{exe}
	\exr{ex:ice-rt} Icelandic:
		\gll P\'{e}tur gaf konunginum amb\'{a}ttina.\\
		Peter.NOM gave king.DEF.DAT maid-servant.DEF.ACC.\\
		\trans `Peter gave the king the maid-servant.'

	\exr{ex:ice-tr} Icelandic:
		\gll ?*Hann gaf amb\'attina konunginum.\\
		He.NOM gave maid-servant.DEF.ACC king.DEF.DAT. \\
		\trans `He gave the king the maid-servant \citep[ex 14b]{Dehe.2004}.'
	\exr{ex:icelandic-goals} Icelandic \citep{Thrainsson.2007}:
		\begin{xlist}
		\ex \gll \'{E}g gaf b\ae kurnar til H\'ask\'olab\'okasafnsins\\
		I.NOM gave books.the.ACC to University.Library.the.GEN\\
		\trans `I gave the books to the University Library'
		\ex \gll \th eir seldu skipi\dh til Englands\\
		they.NOM sold ship.the.ACC to England.GEN\\
		\trans `They sold the ship to England.'
		\end{xlist}

\end{exe}
\subsubsection{Passive Data}
\begin{exe}
	\exr{ex:ice-topic} Icelandic, Topicalization:
\begin{xlist}
	\ex \gll Refinn skaut \textbf{Ólafur} með  þessari byssu.\\
	fox.DEF.ACC shot \textbf{Olaf.NOM} with this shotgun\\
\trans `The fox, Olaf shot with this shotgun \citep[ex. 19a]{Zaenen.1985}.'
\ex[*]{\gll Með  þessari byssu skaut \textbf{refinn} Ólafur.\\
	with this shotgun shot \textbf{fox.DEF.ACC} Olaf.NOM\\
\trans `The fox, Olaf shot with this shotgun \citep[ex. 19b]{Zaenen.1985}.'}
\end{xlist}
\exr{ex:ice-dq} Icelandic, Direct Question:
\begin{xlist}
	\ex \gll Hafði \textbf{Sigga} aldrei hjálpað Haraldi?\\
	had \textbf{Sigga.NOM} never helped Harald.DAT\\
\trans `Had Sigga never helped Harald \citep[ex. 20b]{Zaenen.1985}?'
\ex[*]{\gll Hafði \textbf{Haraldi} Sigga aldrei hjálpað?\\
	had \textbf{Harald.DAT} Sigga.NOM never helped\\
\trans `Had Sigga never helped Harald \citep[ex. 20c]{Zaenen.1985}?'}
\end{xlist}
	\exr{ex:ice-dittop} Icelandic, Ditransitive Topicalization:
\begin{xlist}
	\ex \gll Um veturinn voru \textbf{konunginum} gefnar amb\'{a}ttir.\\
In winter.the were \textbf{king.the.DAT} given slaves.NOM\\
\trans `In the winter the king was given slaves \citep[ex. 47a]{Zaenen.1985}.'
\end{xlist}
\exr{ex:ice-ditdq} Icelandic, Ditransitive Direct Question:
\begin{xlist}
	\ex \gll Voru \textbf{konunginum} gefnar amb\'{a}ttir?\\
were \textbf{king.the.DAT} given slaves.NOM\\
\trans `Was the king given slaves \citep[ex. 48a]{Zaenen.1985}?'
\end{xlist}


	\exr{ex:ice-directthe} Icelandic:
	\begin{xlist}
		\ex \gll Um veturinn voru \textbf{amb\'{a}ttin} gefin konunginum \sout{amb\'{a}ttin}.\\
		In winter.the was \textbf{slave-the.NOM} given king.the.DAT \sout{slave-the.NOM}\\
		\trans `In the winter the slave was given to the king \citep[ex. 47b]{Zaenen.1985}.'
		\ex \gll Var \textbf{amb\'{a}ttin} gefnar konunginum \sout{amb\'{a}ttin}?\\
		were \textbf{slave-the.NOM} given king.the.DAT \sout{slave-the.NOM}\\
		\trans `Was the slave given to the king \citep[ex. 48b]{Zaenen.1985}?'
		\ex \gll \textbf{B\'{o}kin} var gefin J\'{o}ni \sout{B\'{o}kin}\\
		\textbf{book-the.NOM} was given John.DAT \sout{book-the.NOM}\\
		\trans `The book was given to John \citep{Holmberg.1995,Bardal.2001}.'
	\end{xlist}

	\exr{ex:ice-ppsbj} Icelandic:
	\begin{xlist}
		\ex[*]{\gll {\'{I} gar} var um þessa konu oftast talað\\
		yesterday was about this woman often talked\\
		\trans `Yesterday, this woman was often talked about'}
		\ex[*]{\gll {\'{I} gar} var \'{i} r\'{u}minu sofið\\
		yesterday was in bed.DEF slept\\
		\trans `Yesterday, the bed was slept in.'}
	\end{xlist}
\end{exe}

\subsection{Faroese}
\subsubsection{Relevant Citations}
\cite{Haugen.1982,Barnes.1986,Hoskuldurrainsson.2004,Bardal.2007,Jonsson.2009,Eyorsson.2012,Arnadottir.2013,Lundquist.2013,Lundquist.2013b}
\subsubsection{Active Data}
\begin{exe}
	\exr{ex:far-rt} Faroese:
		\gll Hon gav Mariu troyggiuna.\\
		She gave Maria.DAT sweater.DEF.ACC.\\
		\trans `She gave Maria the sweater \citep{Lundquist.2013b}.'
	\exr{ex:far-tr}[\%]{Faroese:
		\gll Hon gav telduna til gentuna.\\
		she gave computer-the.ACC to girl-the.ACC\\
		\trans `She gave the computer to the girl.'}
	\exr{ex:far-case} Faroese:
		\begin{xlist}
			\ex[ ]{\gll Teir góvu \textbf{gentuni} telduna \\
				they gave \textbf{girl-the.DAT} computer-the.ACC \\
			            \trans `They gave the girl the computer.'}
				    \ex[*]{\gll Teir góvu \textbf{gentuna} telduna \\
				they gave \textbf{girl-the.ACC} computer-the.ACC \\
			    \trans `They gave the girl the computer.'}
		\end{xlist}

\end{exe}
\subsubsection{Passive Data}
\begin{exe}
		\exr{ex:far-pass} Faroese:
		\begin{xlist}
		\ex[ ]{\gll Gentan bleiv givin telduna\\
			    girl-the.NOM was given.NOM computer-the.ACC\\
		    	    \trans `The girl was given the computer.'}
		\ex[??]{\gll Gentuni bleiv givin ein telda\\
			    girl-the.DAT was givn.NOM a.NOM computer.NOM\\
		    	    \trans `The girl was given the computer.'}
		\end{xlist}
\end{exe}

\subsection{Norwegian}
\subsubsection{Relevant Citations}
\cite{Haugen.1982,Afarli.1992,Sprouse.1995,Holmberg.1995,Kristoffersen.1994,Askedal.2001,Holmberg.2002,Bardal.2007,Kinn.2010,Afarli.2012,Eyorsson.2012,Lundquist.2013,Lundquist.2013b,Haddican.2014}
\subsubsection{Active Data}
\begin{exe}
	\exr{ex:nor-rt} Standard Norwegian:
		\gll Jeg har gitt mannen boken.\\
		I have given man.DEF book.DEF.\\
		\trans `I gave the man the book \citep[ex 10]{Sprouse.1995}.'

	\exr{ex:nor-tr} Norwegian:
		\gll Vi har lånt den interessante boken du nevnte *(til) Petter.\\
		we have lent the interesting book you mentioned to Peter.\\
		\trans `We have lent the interesting book you mentioned to Peter \citep{Larson.1988}.'
	\exr{ex:halsa-case} Halsa Norwegian:
	\begin{xlist}
		\ex \gll Ho erta \textbf{katt\aa} \\
		she teased \textbf{cat.DEF.ACC} \\
			\trans `She teased the cat.'
			\ex \gll Ho ga \textbf{katt\aa} inn mat \\
			she gave \textbf{cat.DEF.DAT} food \\
			\trans `She gave the cat food.'
	\end{xlist}
	
\end{exe}
\subsubsection{Passive Data}
\begin{exe}
	\exr{ex:nor-null-incorp} Standard Norwegian:
		\gll Han vart P=$\emptyset$-gitt \sout{hann} ein medalje\\
		he.NOM was given \sout{he.NOM} a medal\\
		\trans `He was given a medal.'
	\exr{ex:halsa-pass} Halsa Norwegian:
	\begin{xlist}
		\ex[ ]{\gll Hainn vart gjevinn ei skei.\\
		He.NOM was given a spoon\\
		\trans `He was given a spoon.' \cite[ex 50c]{Eyorsson.2012}}
		\ex[*]{\gll Hånnå vart gjevinn ei skei.\\
		He.DAT was given a spoon\\
		\trans `He was given a spoon.' \cite[ex 50c]{Eyorsson.2012}}
	\end{xlist}
\end{exe}
\subsection{Swedish}
\subsubsection{Relevant Citations}
\cite{Haugen.1982,Falk.1990,Falk.1993,Holmberg.1995,Falk.1997,Anward.1989,Holmberg.2002,Lundquist.2004,Platzack.2005,Lundquist.2006,Bardal.2007,Lundquist.2013,Lundquist.2013b,Haddican.2014,Haddican.2015}
\subsubsection{Active Data}
\begin{exe}
	\exr{ex:sw-rt} Swedish:
		\gll Jag gav Johan en bok.\\
		I gave John a book.\\
		\trans `I gave John a book \citep{Holmberg.1995}.'
	\exr{ex:sw-tr} Swedish:
		\gll Jag gav en bok *(til) Johan.\\
		I gave a book to John.\\
		\trans `I gave a book to John \citep{Holmberg.1995}.'

	\exr{ex:sw-act-simple} Swedish:
			\begin{xlist}
				\ex \gll Han gav Jan bollen\\
				he.NOM gave John ball.the\\
				\trans `He gave John the ball'
				\ex \gll Han gav bollen *(til) Jan\\
				he.NOM gave ball.the to John\\
				\trans `He gave the ball to John'
			\end{xlist}
	\exr{ex:Swedish-complex-act} Swedish:
			\begin{xlist}
				\ex[ ]{\gll Han erbjöd Jan ett nytt jobb\\
				he.NOM offered John a new job\\
			\trans `He offered John a new job'}
				\ex[??]{\gll Han erbjöd ett nytt jobb til Jan\\
				he.NOM offered a new job to John\\
			\trans `He offered a new job to John'}
				\ex[*]{\gll Han erbjöd ett nytt jobb Jan\\
				he.NOM offered a new job John\\
			\trans `He offered a new job to John'}
			\end{xlist}

\end{exe}
\subsubsection{Passive Data}
\begin{exe}
	\exr{ex:swe-part} Swedish:
	\begin{xlist}
		\ex[ ]{Particle Verb:
		\gll Han erbjöds ett nytt jobb\\
			he.NOM offered.PASS a new job\\
			\trans `He was offered a new job (\citealt{Anward.1989}, \citealt{Lundquist.2006}).'}
		\ex[*]{Non-Particle Verb:
		\gll Pelle gavs ett äpple\\
			Pelle gave.PASS a apple\\
			\trans `Pelle was given an apple (\citealt{Anward.1989}, \citealt{Lundquist.2006}).'}
	\end{xlist}	
	\exr{ex:swe-nopart-pass} Swedish (verbs without particles):
		\sn[*]{
			\gll Ett äpple gavs Pelle.\\
			 An apple gave.PASS Pelle.\\
			 \trans `An apple was given to Pelle (\citealt{Anward.1989},\citealt{Lundquist.2006}).'}
	\exr{ex:sw-offer-pass} Swedish:
		\gll Ett nytt jobb erbjöds=honom.\\
		A new job offered.PASS=him.OBL.\\
		\trans `A new job was offered to him (\citealt{Anward.1989},\citealt{Falk.1990},\citealt{Lundquist.2006}).'
	\exr{ex:sw-give-pass} Swedish:
		\gll *Ett äpple gavs honom.\\
		 An apple gave.PASS him.\\
		 \trans `An apple was given to him (\citealt{Anward.1989},\citealt{Lundquist.2006}).'
	 \exr{ex:sw-offer-thepas} Swedish:
		\gll Jobbet erbjöds mannen med den långa svarta kappan.\\
		job.DEF offered.PASS man.DEF with the long black coat\\
		'The job was offered to the man with the long black coat \citep[ex 26]{Lundquist.2004}.'

	\exr{ex:sw-relpass} Swedish:
		\begin{xlist}
			\ex[ ]{\gll \textbf{DET} \textbf{jobbet} har Kalle tilldelats.\\
			that job.DEF has Kalle assigned.PART.PASS\\
			\trans `THAT job, Kalle has been assigned \citep[ex. 59]{Lundquist.2004}.'}
			\ex[??]{\gll DEN mannen har jobbet tilldelats.\\
			that man.DEF has job.DEF assigned.PART.PASS\\
			\trans `To THAT man, the job has been assigned \citep[ex. 58]{Lundquist.2004}.'}
		\end{xlist}
	\exr{ex:sw-relpass2} Swedish:
		\begin{xlist}
			\ex[ ]{\gll Jobbet som erbjöds \textbf{mannen} var mycket slitsamt.\\
			job.DEF which offered.PASS \textbf{man.DEF} was very tiring\\
			\trans `The job, which was offered to the man, was very tiring \citep[ex. 49]{Lundquist.2004}.'}
			\ex[ ]{\gll Jobbet som \textbf{mannen} erbjöds var mycket slitsamt.\\
			job.DEF which \textbf{man.DEF} offered.PASS was very tiring\\
			\trans `The job, which the man was offered, was very tiring \citep[ex. 50]{Lundquist.2004}.'}
		\end{xlist}
\end{exe}



\subsection{Danish}
\subsubsection{Relevant Citations}
\cite{Haugen.1982,Herslund.1986,Vikner.1989,Falk.1990,Sprouse.1995,Allan.1995,Bardal.2007,Lundquist.2013,Lundquist.2013b}
\subsubsection{Active Data}
\begin{exe}
	\exr{ex:dan-rt} Danish:
		\gll Peter viste jo Marie bogen.\\
		Peter showed indeed Mary book.DEF.\\
		\trans `Peter indeed showed Mary the book \citep{Vikner.1989}.'
	\exr{ex:dan-tr} Danish:
		\gll Jeg gav bogen *(til) Anna.\\
		I gave book.the to Anna.\\
		\trans `I gave the book to Anna\citep{Holmberg.1998}.'
\end{exe}
\subsubsection{Passive Data}
\begin{exe}
	\exr{ex:dan-pass} Danish:
	 \sn[*]{
	 \gll En stilling blev tilbudt ham.\\
A job was offered him.OBL.\\
\trans `A job was offered to him \citep{Falk.1990}.'}

	\exr{ex:dan-null-incorp} Danish:
	\gll Han blev P=$\emptyset$-tilbudt \sout{hann} en stilling\\
	he.NOM was offered \sout{he.NOM} a job\\
	\trans `He was offered a job.'
	\exr{ex:dan-pseudopass} Danish:
	\begin{xlist}
		\ex[ ]{\gll Revisionen blev \textbf{p\aa begyndt} i maj\\
		revision-the was \textbf{on-begun} in May\\
		\trans `The revision was begun in May'}
		\ex[*]{\gll Revisionen blev \textbf{begyndt} \textbf{p\aa} i maj\\
		revision-the was \textbf{begun} \textbf{on} in May\\
		\trans `The revision was begun in May'}
	\end{xlist}
\end{exe}




\section{West Germanic}
\subsection{High German}

\subsubsection{Relevant Citations}
\cite{Shrier.1965,Lenerz.1977,Werner.1982,Hohle.1982,Webelhuth.1984,Scherpenisse.1986,Abraham.1986,Webelhuth.1989,Besten.1990,Czepluch.1990,Frey.1993,Lee.1994,Sprouse.1995,Draye.1996,Leirbukt.1997,Holmberg.1998,McGinnis.1998,Maling.2001,Frey.2001,Seiler.2001,Askedal.2001,Bayer.2001,Seiler.2003,McFadden.2004,Platzack.2005,McFadden.2006,Meinunger.2006,Eythorsson.2005,Bardal.2006,Fleischer.2006,Georgala.2011,Georgala.2011b,Alexiadou.2013b,Alexiadou.2014}
\subsubsection{Active Data}
\begin{exe}
	\exr{ex:german-forms} High German, Dative--Preposition Alternation:
	\begin{xlist}
		\ex  \gll Ich habe der Frau das Buch geschickt\\
			I.NOM have the.DAT woman the.ACC book sent\\
			 \trans `I sent the woman the book.'
		 \ex \gll Ich habe das Buch der Frau geschickt\\
			 I.NOM have the.ACC book the.DAT woman sent\\
			 \trans `I sent the woman the book.'
		 \ex \gll Ich habe das Buch an die Frau geschickt\\
			 I.NOM have the.ACC book to the.ACC woman sent\\
			 \trans `I sent the book to the woman.'
	\end{xlist}
	\exr{ex:hg-rt} High German:
		\gll weil er der Unehrlichkeit keine Chance gibt.\\
		as he.NOM the.DAT dishonesty no.ACC opportunity gives.\\
		\trans `as he gives dishonesty no opportunity \citep[162]{Draye.1996}.'
	\exr{ex:hg-tr} High German:
		\gll weil er keine Chance der Unehrlichkeit gibt.\\
		as he.NOM no.ACC opportunity the.DAT dishonesty gives.\\
		\trans `as he gives no opportunity to dishonesty'
	\exr{ex:german-goals} High German:
		\begin{xlist}
			\ex[\#]{\gll Er hat \textbf{Maria} einen Brief geschickt, aber er ist bei ihr nicht angekommen\\
			he.NOM has \textbf{Maria} a.ACC letter sent, but he.NOM is by her.DAT not arrived\\
			\trans `He sent Maria a letter, but it has not reached her.'}
			\ex[\#]{\gll Er hat einen Brief \textbf{Maria} geschickt, aber er ist bei ihr nicht angekommen\\
			he.NOM has a.ACC letter \textbf{Maria} sent, but he.NOM is by her.DAT not arrived\\
			\trans `He sent a letter to Maria, but it has not reached her.'}
			\ex[ ]{\gll Er hat einen Brief \textbf{an} \textbf{Maria}  geschickt, aber er ist bei ihr nicht angekommen\\
			he.NOM has a.ACC letter \textbf{to} \textbf{Maria} sent, but he.NOM is by her.DAT not arrived\\
			\trans `He sent a letter to Maria, but it has not reached her.'}
		\end{xlist}
	\exr{ex:german-VP-top} High German, VP-topicalisation:
		\begin{xlist}
			\ex \gll  Dem Mann das Buch gegeben habe ich, (nicht der Frau dEN Film geschenkt).\\
			the.DAT man the.ACC book given have I, (not the.DAT woman the.ACC film sent).\\
			\trans `It was giving the man the book that I did (not sending the woman the film).'
			\ex \gll Das Buch dem Mann gegeben habe ich, (nicht dEN Film der Frau geschenkt).\\
			the.ACC book the.DAT man given have I, (not the.ACC film the.DAT woman sent).\\
			\trans `It was giving the book to the man that I did (not sending the film to the woman).\\
		\end{xlist}
	\exr{ex:hg-vp-top-goals} High German, VP-topicalisation:
		\begin{xlist}
			\ex[*]{ \gll  An den Mann das Buch geschickt habe ich, (nicht an die Frau dEN Film übergeben).\\
			to the.ACC man the.ACC book sent have I, (not to the.ACC woman the.ACC film delivered).\\
		\trans `It was sending to the man the book that I did (not delivering to the woman the film).'}
		\ex[ ]{\gll Das Buch an den Mann gegeben habe ich, (nicht dEN Film an die Frau übergeben).\\
			the.ACC book to the.ACC man given have I, (not the.ACC film to the.ACC woman delivered).\\
		\trans `It was sending the book to the man that I did (not delivering the film to the woman).\\}
	\end{xlist}

	\exr{ex:hg-rec-focus} High German, Recipient Focus \citep{Choi.1996}:
		\gll Wem hast du das Geld gegeben?\\
		whom.DAT have you.NOM the money.ACC given\\
		\trans `Who did you give the money to?'
		\begin{xlist}
			\ex \gll Ich habe dem KASSIERER das Geld gegeben.\\
			I.NOM have the cashier.DAT the money.ACC given.\\
			\trans `I have given the cashier the money.'
			\ex \gll Ich habe das Geld dem KASSIERER gegeben.\\
			I.NOM have the money.ACC the cashier.DAT given.\\
			\trans `I have given the money to the cashier.'
		\end{xlist}
	\exr{ex:hg-theme-focus} High German, Theme Focus \citep{Choi.1996}:
		\gll Was hast du dem Kassierer gegeben?\\
		what.ACC have you.NOM the cashier.DAT given\\
		\trans `What did you give to the cashier?'
		\begin{xlist}
			\ex[ ]{\gll Ich habe dem Kassierer das GELD gegeben.\\
			I.NOM have the cashier.DAT the money.ACC given.\\
			\trans `I have given the cashier the money.'}
			\ex[?*]{\gll Ich habe das GELD dem Kassierer gegeben.\\
			I.NOM have the money.ACC the cashier.DAT given.\\
			\trans `I have given the money to the cashier.'}
		\end{xlist}
	\exr{ex:hg-binding-rt} High German, recipient--theme:
		\begin{xlist}
			\ex[ ]{\gll dass Maria jedem seinen Nachbarn vorgestellt hat.\\
			that Maria everyone.DAT his.ACC neighbour.ACC introduced has.\\
			\trans `that Maria introduced everyone his neighbor \citep[ex. 11a]{Lee.1994}.'}
			\ex[*]{\gll dass Maria seinem Nachbarn jeden vorgestellt hat.\\
			that Maria his.DAT neighbour.DAT everyone.ACC introduced had.\\
			\trans `that Maria introduced everyone to his neighbour \citep[ex. 9a]{Lee.1994}.'}
		\end{xlist}
	\exr{ex:hg-binding-tr} High German, theme--recipient:
		\begin{xlist}
			\ex[ ]{\gll dass Maria jeden seinem Nachbarn vorgestellt hat.\\
			that Maria everyone.ACC his.DAT neighbour.DAT introduced had.\\
			\trans `that Maria introduced everyone to his neighbour \citep[ex. 10a]{Lee.1994}.'}
			\ex[\%]{\gll dass Maria seinen Nachbarn jedem vorgestellt hat.\\
			that Maria his.ACC neighbour.ACC everyone.DAT introduced had.\\
			\trans `that Maria introduced everyone his neighbour \citep[ex. 12a (note 10)]{Lee.1994}.'}
		\end{xlist}


	\exr{ex:hg-creation} High German \citep[ex 31]{Frey.2001}:
	\begin{xlist}
		\ex[ ]{\gll dass Hans geschickt eine Flöte schnitzte\\
		that John skillfully a.ACC flute carved\\
		\trans `that John skillfully carved a flute.'}
		\ex[*]{\gll dass Hans eine Flöte geschickt schnitzte\\
		that John a.ACC flute skillfully carved\\
		\trans `that John skillfully carve a flute.'}
	\end{xlist}
	\exr{ex:hg-VPadverb} High German, VP-level adverbs:
		\begin{xlist}
			\ex \gll Ich habe nicht dem Mann das Buch gegeben, SONDERN DER FRAU DEN FILM GESCHENKT.\\
			I have not the.DAT man the.ACC book given, but the.DAT woman the.ACC film sent.\\
			\trans `I didn't give the man the book, instead I sent the woman the film.'
			\ex \gll Ich habe nicht das Buch dem Mann gegeben, SONDERN DEN FILM DER FRAU GESCHENKT.\\
			I have not the.ACC book the.DAT man given, but the.ACC film the.DAT woman sent.\\
			\trans `I didn't give the book to the man, insead I sent the film to the woman.'
		\end{xlist}
	\exr{german-nom} High German, Nominalisation:
		\begin{xlist}
			\ex[ ]{\gll Oswald hat den P\"{a}sident errnordet\\
			Oswald has the president.ACC assassinated\\
			\trans `Oswald assassinated the president \citep[ex 5a]{Bayer.2001}'}
			\ex[ ]{\gll die Ermordung des Pr\"{a}sidenten\\
			the.NOM assassination the.GEN president\\
			\trans `the assassination of the president \citep[ex 5c]{Bayer.2001}'}
			\ex[ ]{\gll Oswald hat dem Pr\"{a}sidenten gehuldigt\\
			Oswald has the.DAT president given-homage\\
			\trans `Oswald gave homage to the president \citep[ex 6a]{Bayer.2001}'}
			\ex[*]{\gll die Huldigung des/dem Pr\"{a}sidenten\\
			the.NOM homage-giving the.GEN/the.DAT president\\
			\trans `the homage giving to the president \citep[ex 6]{Bayer.2001}'}
		\end{xlist}

	\exr{ex:zurich} Zürich German:
		\gll si schänkt äine \textbf{a} \textbf{de} \textbf{Tristane}\\
		they.NOM sent one.ACC to the Tristan\\
		\trans `The sent one to Tristan \citep[pg. 175]{Seiler.2003}.'
	\exr{ex:luzern} Luzern German:
		\gll miir verchauggid \textbf{i} \textbf{de} \textbf{Chunde} nur Mère-Josephine-Poulets\\
		we.NOM sold to the clients only Mere-Josephine chicken\\
		`We sold the clients only Mere-Josephine chicken \citep[pg. 175]{Seiler.2003}.'
	\exr{ex:hg-Pcomp} High German:
		\begin{xlist}
			\ex \gll in + P$_{goal}$ den Baum\\
				 in + P$_{goal}$ the.ACC tree\\
				 \trans `into the tree'
			\ex \gll in + P$_{location}$ dem Baum\\
				 in + P$_{location}$ the.DAT tree\\
				 \trans `in the tree'
		\end{xlist}

\end{exe}
\subsubsection{Passive Data}
\begin{exe}
	\exr{ex:hg-accnom} High German:
		\begin{xlist}
			\ex \gll Ich habe den Mann gesehen\\
			I.NOM have the.\textbf{ACC} man seen\\
			\trans `I saw the man.'
			\ex \gll Der Mann wurde gesehen\\
			the.\textbf{NOM} man was seen\\
			\trans `The man was seen.'
		\end{xlist}
	\exr{ex:hg-insitu-sbj} High German:
		\gll Ich glaube, dass den Kindern das Fahrrad geschenkt worden ist.\\
		I beleive that the.DAT.PL children the.NOM bicycle granted become be.3sg\\
		\trans `I believe that the child was granted the bicycle.'
	\exr{ex:hg-normal-pass} High German:
		\begin{xlist}
			\ex[ ]{\gll Ich glaube, dass \textbf{den} \textbf{Kindern} das Fahrrad geschenkt worden ist.\\
			I beleive that \textbf{the.DAT.PL} \textbf{children} the.NOM bicycle granted become be.3sg\\
			\trans `I believe that the children were granted the bicycle.'}
			\ex[*]{\gll Ich glaube, dass \textbf{die} \textbf{Kindern} das Fahrrad geschenkt worden sind.\\
			I beleive that \textbf{the.NOM.PL} \textbf{children} the.ACC bicycle granted become be.3pl\\
			\trans `I believe that the children were granted the bicycle.'}
		\end{xlist}
	\exr{ex:hg-get-pass} High German:
		\begin{xlist}
			\ex \gll dass der Vater \textbf{der} \textbf{Tochter} ein Buch geschenkt hat\\
			that the.NOM father \textbf{the.DAT} \textbf{daughter} a.ACC book sent has\\
			\trans `that the father sent the daughter a book.'
			\ex \gll dass \textbf{die} \textbf{Tochter} von dem Vater ein Buch geschenkt bekommen hat\\
			that \textbf{the.NOM} \textbf{daughter} by the father a.ACC book sent got has\\
			\trans `that the daughter got sent a book by her father \cite[183]{Draye.1996}.'
		\end{xlist}

\end{exe}
\subsection{Yiddish}
\subsubsection{Relevant Citations}
\cite{Birnbaum.1979,Holmberg.1998}
\subsubsection{Active Data}
\begin{exe}
	\exr{ex:yid-rt} Yiddish:
		\gll Zi git der snjjer dus pékl. \\
		she.NOM gives the.DAT daughter-in-law the.ACC parcel.\\
		\trans 'She gives her daughter-in-law the parcel \citep[ex 190a]{Birnbaum.1979}.'
\end{exe}
\subsubsection{Passive Data}
No examples.


\subsection{Dutch}
\subsubsection{Relevant Citations}
\cite{Scherpenisse.1986,Besten.1990,Hoekstra.1991,SchermerVermeer.1991,Broekhuis.1994,Sprouse.1995,DenDikken.1995,Sprouse.1995,Holmberg.1995,vanBelle.1996b,Holmberg.1998,Holmberg.2002,Anagnostopoulou.2003,Donaldson.2008,Colleman.2009,Colleman.2009b,Colleman.2010,Colleman.2010b,Colleman.2011,Colleman.2010c,Colleman.2012,Broekhuis.2012,Alexiadou.2013b,Alexiadou.2014}
\subsubsection{Active Data}
\begin{exe}
	\exr{ex:dut-rt} Dutch:
		\gll Ik heb (aan) Jan een boek gegeven.\\
		I have (to) Jan a book given.\\
		\trans `I gave Jan a book \citep{Tiersma.1985}.'
	\exr{ex:dut-tr} Dutch:
		\gll Ik heb een boek *(aan) Jan gegeven.\\
		I have a book *(to) Jan given.\\
		\trans `I gave a book to Jan.'
	\exr{ex:dutch-rec-marking} Dutch:
		\begin{xlist}
			\ex \gll Ik heb een boek *(aan) Jan gegeven\\
			I have a book to John given\\
			\trans `I gave a book to John \citep{Tiersma.1985}.'
			\ex \gll Ik heb (aan) Jan een boek gegeven\\
			I have to John a book given\\
			\trans `I gave John a book \citep{Tiersma.1985}.'
		\end{xlist}

\end{exe}
\subsubsection{Passive Data}
\begin{exe}
	\exr{ex:dutch-normal-pass} Dutch:
		\begin{xlist}
			\ex[ ]{\gll De boeken \textbf{werden} haar aangeboden.\\
			the books \textbf{became.PL} her given\\
			\trans `The books were given to her.' \citep[ex. 5b]{Broekhuis.1994}}
			\ex[*]{\gll Zij \textbf{werd} de boeken aangeboden.\\
			she.NOM \textbf{became.SG} the books given\\
			\trans `She was given the books.' \citep[ex. 5c]{Broekhuis.1994}}
		\end{xlist}
	\exr{ex:dut-insitu-sbj} Dutch:
		\gll Er werd mij een boek gegeven.\\
		There became.3sg me a book given\\
		\trans `A book was given to me. \cite[pg 245]{Donaldson.2008}'
	\exr{ex:dut-get-pass} Dutch:
		\gll \textbf{Zij} kreeg de boeken (van mij) aangeboden.\\
		\textbf{she.NOM} got the books (by me) given\\
		\trans `She was given the books (by me).' \citep[ex. 7]{Broekhuis.1994}

	\exr{ex:dut-scram} Dutch:
	\begin{xlist}
		\ex[ ]{\gll dat het boek \textbf{Marie} waarschijnlijk gegeven wordt\\
		that the book \textbf{Mary} probably given was\\
		\trans `that the book was probably given to Mary.'}
		\ex[?*]{\gll dat het boek waarschijnlijk \textbf{Marie} gegeven wordt\\
		that the book probably \textbf{Marie} given was\\
		\trans `that the book was probably given to Mary.'}
	\end{xlist}

\end{exe}

\subsection{Afrikaans}
\subsubsection{Relevant Citations}
\cite{Donaldson.1993,Stadler.1996,Louw.2012}
\subsubsection{Active Data}
\begin{exe}
	\exr{ex:af-rt} Afrikaans:
		\gll dat die man die vrou `n dokument gegee het.\\
		that the man the woman a document given has.\\
		\trans `...that the man gave a document to the woman \citep{Louw.2012}.'
	\exr{ex:af-tr} Afrikaans:
		\gll Ek het `n fooitjie aan hom gegee.\\
		I have a tip to him given.\\
		\trans `I have given a tip to him \citep{Stadler.1996}.'

\end{exe}
\subsubsection{Passive Data}
\begin{exe}
	\exr{ex:af-rec-pass1} Afrikaans: 
		\begin{xlist}
			\ex \gll Is aan hom ooit 'n geskenk gegee?\\
			Was to him ever a present given.\\
			\trans `Was he ever given a present \citep[ex. 49]{Stadler.1996}?'
			\ex \gll Gister is aan hom `n klomp geld gegee.\\
			Yesterday was to him a {lot of} money given.\\
			\trans `Yesterday he was given a lot of money \citep[ex. 50]{Stadler.1996}.'
	\end{xlist}

	\exr{ex:af-rec-pass2} Afrikaans:
		\begin{xlist}
			\ex[?]{\gll hy is `n present gegee.\\
			he was a present given\\
			\trans `He was given a present \citep[ex. 35]{Stadler.1996}.'}
			\ex[ ]{\gll Aan hom is `n present gegee.\\
			to him was a present given\\
			\trans `He was given a present \citep[ex. 44]{Stadler.1996}.'}
		\end{xlist}

\end{exe}


\subsection{Frisian}
\subsubsection{Relevant Citations}
\cite{Tiersma.1985}
\subsubsection{Active Data}
\begin{exe}
	\exr{ex:fri-rt} Frisian:
		\gll se joech jar kammeraatske in skjirre.\\
		she gave her girlfriend a {pair of scissors}.\\
		\trans `She gave her girlfriend a pair of scissors.'

	\exr{ex:fri-tr} Frisian:
		\gll ik joech in plant oan Beppe.\\
		I gave a plant to Grandmother.\\
		\trans `I gave a plant to Grandmother \citep{Tiersma.1985}.'

\end{exe}
\subsubsection{Passive Data}
No examples.

\subsection{Low German}
\subsubsection{Relevant Citations}
\cite{Mussaus.1829,Lasch.1914,Keseling.1970,Ponelis.1979,Ponelis.1993,Boden.1993,Lindow.1998,Appel.2007}
\subsubsection{Active Data}
\begin{exe}
	\exr{ex:lg-rt} Low German:
		\gll ick gaw den Mann dat Brod.\\
		I gave the man the bread.\\
		\trans `I gave the man the bread \citep{Mussaus.1829}.'

	\exr{ex:lg-tr} Low German:
		\gll ick gaw dat Brod den Man, wobei dat Brod zeigend ist.\\
		I gave the bread the man who the bread shown is.\\
		\trans `I gave the bread to the man who was shown the bread \citep{Mussaus.1829}.'
\end{exe}
\subsubsection{Passive Data}
No examples.

\subsection{English}
\subsubsection{Relevant Citations}
\cite{Fillmore.1965,Emonds.1972,Langendoen.1973,Oehrle.1976,Hornstein.1981,McLaughlin.1983,Mitchell.1985,Barss.1986,Larson.1988,Gropen.1989,Anward.1989,Aoun.1989,Maling.1990,Falk.1990,Jackendoff.1990,Johnson.1991,Hoekstra.1991,Levin.1993,DenDikken.1995,Sprouse.1995,Collins.1995,Holmberg.1998,Allen.1999,Kroch.2000,Maling.2001,Bruening.2001,Oba.2002,Polo.2002,McFadden.2002,Harley.2002,Taylor.2003,Anagnostopoulou.2003,McFadden.2004,Kroch.2004,Postal.2004,Platzack.2005,Oba.2005,Alexiadou.2005,Levinson.2005,Taylor.2006,Bresnan.2007,Gast.2007,Hovav.2008,Bresnan.2009,Kroch.2010,Levin.2010,Bresnan.2010,Davies.2010,Bruening.2010,Bruening.2010b,Haddican.2010,Larson.2010,Myler.2011,Haddican.2011,Haddican.2012,Ormazabal.2012,SowkaPietraszewska.2013,Gerwin.2013,Bruening.2014,Haddican.2014,Sigursson.2014,Biggs.2015,Cuypere.2015,Hallman.2015,Harley.2015}
\subsubsection{Active Data}
\begin{exe}
	\exr{ex:eng-rt} English: I gave the man the book.
	\exr{ex:en-tr} English: I gave the book to the man.
	\exr{ex:dat-shift} Modern American English, Dative Shift:
	\begin{xlist}
		\ex I sent the woman the book.
		\ex I sent the book to the woman.
	\end{xlist}
	\exr{ex:eng-rec-wh} English, Recipients:
		\begin{xlist}
			\ex[ ]{Who did you give the package to?}
			\ex[*]{Where did you give the package to?}
		\end{xlist}
	\exr{ex:eng-goal-wh} English, Goals:
		\begin{xlist}
			\ex[ ]{Who did you send the package to?}
			\ex[ ]{Where did you send the package to?}
		\end{xlist}
	\exr{ex:hallman-rec} English \citep[exx 6 \& 7]{Hallman.2015}:
		\begin{xlist}
			\ex Mary gave John$_{i}$ a puppy$_{k}$ [PRO$_{i}$ to play with e$_{k}$].
			\ex Mary gave a puppy$_{k}$ to John$_{i}$ [PRO$_{i}$ to play with e$_{k}$].
			\ex Mary sent John$_{i}$ a manuscript$_{k}$ [PRO$_{i}$ to read e$_{k}$]
			\ex Mary sent a manuscript$_{k}$ to John$_{i}$ [PRO$_{i}$ to read e$_{k}$]
		\end{xlist}
	\exr{ex:recipient--purpose} English \citep[ex 25]{Hallman.2015}:
	\begin{xlist}
		\ex[*]{Mary gave a puppy to play with to John}
		\ex[ ]{Mary gave a puppy to John to play with}
	\end{xlist}
	\exr{ex:hallman-poc} English \citep[ex 9]{Hallman.2015}:
	\begin{xlist}
		\ex[*]{Mary put the child$_{k}$ on the horse$_{i}$ [PRO$_{i}$ to carry e$_{k}$]}
		\ex[*]{Mary led the horse$_{k}$ to John$_{i}$ [PRO$_{i}$ to feed e$_{k}$]}
		\ex[*]{Mary immersed the cloth$_{k}$ in oil$_{i}$ [PRO$_{i}$ to permeate e$_{k}$]}
		\ex[*]{Mary placed the planting pots$_{k}$ under the tomato vines$_{i}$ [PRO$_{i}$ to grow over e$_{k}$]}
	\end{xlist}
	\exr{ex:hallman-control-2}English \citep[ex 10]{Hallman.2015}:
	\begin{xlist}
		\ex[ ]{Mary put the child on the horse}
		\ex[ ]{Mary led the horse to John}
		\ex[ ]{Mary immersed the cloth in oil}
		\ex[ ]{Mary placed the planting pots under the tomato vines}
	\end{xlist}
	\exr{ex:en-anaphor} English, Anaphor Binding:
		\begin{xlist}
			\ex Recipient--theme: I showed Mary herself (in the mirror).
			\ex Recipient--theme: *I showed herself Mary (in the mirror).
			\ex Theme--recipient: I showed Mary to herself (in the mirror).
			\ex Theme--recipient: *I showed herself to Mary (in the mirror).
		\end{xlist}
	\exr{ex:en-superiority} English, Superiority:
		\begin{xlist}
			\ex Recipient--theme: Who did you give which check?
			\ex Recipient--theme: *Which paycheck did you give who?
			\ex Theme--recipient: Which check did you give to who?
			\ex Theme--recipient: *Who did you give which check to?
		\end{xlist}
	\exr{ex:en-negpol} English, Negative Polarity:
		\begin{xlist}
			\ex Recipient--theme: I showed no one anything.
			\ex Recipient--theme: *I showed anyone nothing.
			\ex Theme--recipient: I showed nothing to any one.
			\ex Theme--recipient: *I showed anything to no one.
		\end{xlist}
	\exr{ex:en-qb} English, Quantifier Binding:
		\begin{xlist}
			\ex Recipient--theme: I gave every worker$_i$'s mother his$_i$ paycheck.
			\ex Recipient--theme: * I gave his$_i$ mother every worker$_i$'s paycheck.
			\ex Theme--recipient: I gave every worker$_i$'s paycheck to his$_i$ mother.
			\ex Theme--recipient: ? I gave his paycheck to every worker$_i$'s mother.
		\end{xlist}
	\exr{ex:en-eachother} English, Each...the other:
		\begin{xlist}
			\ex Recipient--theme: I showed each man the other's friend.
			\ex Recipient--theme: * I showed the other's friend each man.
			\ex Theme--recipient: I showed each man to the other's friend.
			\ex Theme--recipient: ? I showed the other's friend to each man.
		\end{xlist}
	\exr{ex:me-accgen} Modern English, Accusative-to-genitive in nominalisation (Non-recipient):
		\begin{xlist}
			\ex John kissed Mary.
			\ex John's kissing of Mary\ldots
		\end{xlist}
	\exr{ex:me-rec-nominal-gen} Modern English, Accusative-to-genitive in nominalisation (Recipient):
		\begin{xlist}
		\ex[ ]{John gave Mary a book.}
		\ex[*]{John's giving of a book of Mary\ldots}
		\ex[*]{John's giving of Mary\ldots}
		\ex[*]{John's giving of Mary of a book\ldots}
		\end{xlist}
	\exr{ex:en-implicature} Modern English:
	\begin{xlist}
		\ex[\#]{John taught the students French, but they didn't learn French}
		\ex[ ]{John taught French to the students, but they didn't learn French}
	\end{xlist}
	\exr{ex:en-implicature-give} English, `give', \citep[exx 36 \& 37]{Hovav.2008}:
	\begin{xlist}
		\ex[\#]{My aunt gave my brother some money for new skis, but he never got it}
		\ex[\#]{My aunt gave some money to my brother for new skis, but he never got it}
	\end{xlist}
	\exr{ex:en-implicature-offer} English, `offer', \citep[exx 38 \& 39]{Hovav.2008}:
	\begin{xlist}
		\ex Max offered the victims help, but they refused his offer.
		\ex Max offered help to the victims, but they refused his offer.
	\end{xlist}
	\exr{ex:en-nixon-rt} English: Nixon gave Mahler a book.
	\begin{xlist}
		\ex[ ]{Nixon gave Mahler a physical object (namely a book)}
		\ex[ ]{Nixon gave Mahler an idea (that Mahler wrote into a book)}
	\end{xlist}
	\exr{ex:en-nixon-tr} English: Nixon gave a book to Mahler.
	\begin{xlist}
		\ex[ ]{Nixon gave Mahler a physical object (namely a book)}
		\ex[*]{Nixon gave Mahler an idea (that Mahler wrote into a book)}
	\end{xlist}
	\exr{ex:en-creation} English \citep[ex. 2]{Bruening.2010b}:
	\begin{xlist}
	\ex[ ]{The lighting here gives me a headache}
	\ex[*]{The lighting here gives a headache to me}
	\end{xlist}
	\exr{ex:en-idioms} English \citep[ex 3]{Bruening.2010b}:
	\begin{xlist}
		\ex[ ]{The count gives me the creeps}
		\ex[*]{The count gives the creeps to me}
	\end{xlist}

	\exr{ex:me-rec-nominal} Modern English, Recipients in nominalisation:
		\begin{xlist}
			\ex[ ]{John gave Mary a book.}
				\ex[ ]{John's giving of a book to Mary\ldots}
				\ex[ ]{John's giving to Mary\ldots}
				\ex[?]{John's giving to Mary of a book\ldots}
		\end{xlist}
	\exr{ex:en-purpose-donate} English (ex. 48 from \citealt{Hallman.2015}):
	\begin{xlist}
		\ex John donate money$_{j}$ to the church$_{i}$ [PRO$_{i}$ to buy candles with e$_{j}$].
		\ex Mary submitted a draft$_{j}$ to the professor$_{i}$ [PRO$_{i}$ to comment on e$_{j}$].
		\ex Mary returned the books$_{j}$ to John$_{i}$ [PRO$_{i}$ to reshelve e$_{j}$].
		\ex John revealed the plan$_{j}$ to Mary$_{i}$ [PRO$_{i}$ to consider e$_{j}$].
		\ex Mary demonstrated the technique$_{j}$ to John$_{i}$ [PRO$_{i}$ to teach e$_{j}$ to the new assistants].
	\end{xlist}
	\exr{ex:en-donate} Modern English:
	\begin{xlist}
		\ex[*]{John donated him a kidney.}
		\ex[?]{John donated the library books.}
	\end{xlist}

	\exr{ex:nw-brit-P} Northwestern British English:
		\begin{xlist}
		\ex[ ]{John [gave=it] [P=$\emptyset$ Mary]}
		\ex[*]{John [gave] [the book] [P=$\emptyset$ Mary]}
		\end{xlist}

	\exr{ex:liverpool} Liverpool English \citep{Biggs.2015}:
		\begin{xlist}
			\ex Mary gave the teacher the book.
			\ex Mary gave the book the teacher.
			\ex Mary sent the package her nan's.
			\ex I want to go Chessington. (unambiguous goal)
		\end{xlist}

\end{exe}
\subsubsection{Passive Data}
\begin{exe}

	\exr{ex:eng-null-incorp} English: He was P=$\emptyset$-given \sout{he} the ball.
	\exr{ex:eng-pseudopass} English: 
	\begin{xlist}
		\ex[*]{The bed was inslept.}
		\ex[ ]{The bed was slept in.}
	\end{xlist}
	\exr{ex:eng-directtheme} English Dialects:
		\begin{xlist}
			\ex[ ]{The book was given P=$\emptyset$ the man \sout{the book}.}
			\ex[*]{The book was given \sout{the book} P=$\emptyset$ the man \sout{the book}.}
		\end{xlist}

	\exr{ex:amen-pass} Modern American English: *The book was given P=$\emptyset$ John \sout{the book}.

	\exr{ex:amen-thepass} Modern American English: The book was given \sout{the book} P=to John \sout{the book}.
	\exr{ex:en-intervene} English:
	\begin{xlist}
			\ex[ ]{The book was given \sout{the book} P=to the man \sout{the book}.}
			\ex[*]{The book was given \sout{the book} P=$\emptyset$ the man \sout{the book}.}
	\end{xlist}
	\exr{ex:en-clitic} English Dialects (cliticisation): The book was given=me \sout{the book}.
	\exr{ex:endial-prosens} English Dialects:
	\begin{xlist}
		\ex[ ]{The book was given me.}
		\ex[*]{The book was given John.}
	\end{xlist}
	\exr{ex:amen-relpass} Modern American English (Recipient Relatives):
	\begin{xlist}
		\ex[ ]{The man, who was given the book, read.}
		\ex[?]{The man, who the book was given to, read.}
		\ex[??]{The man, who the book was given, read.}
	\end{xlist}
	\exr{ex:amen-rellpass2} Modern American English (Theme Relatives):
	\begin{xlist}
		\ex[?]{The book, which the man was given, was red}
		\ex[ ]{The book, which was given to the man, was red}
		\ex[??]{The book, which was given the man, was red}
	\end{xlist}
\end{exe}
%\bibliography{diss.bib}
