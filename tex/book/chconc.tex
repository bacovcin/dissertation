\chapter{Conclusions and Further Implications}
\section{Conclusions}
This dissertation argued for the dative PP + applicative analysis for recipient ditransitives in Germanic. Recipient dative PPs were distinguished from goal PPs in so far as goal PPs are merged as the complement of V and recipients are merged in the specifier of an applicative head. Data from typology, focus sensitivity, reconstruction effects, and control possibilities (into purpose clauses) were used to support the notion that the recipient starts off in a position outside of the VP, which C-commands the base position of the theme. After linearisation, this generates a base RT word order in the active.

TR (theme--recipient) word orders were derived by VP-internal scrambling of the theme to a second specifier of the applicative phrase. Focus sensitivity evidence from German as well as scope reconstruction effects from German and English support this analysis. The availability of this operation was subject to variation; Modern Icelandic does not have this operation and only has RT (recipient--theme) word orders in the active. 

Dative shift, where the recipient is marked with a preposition in the TR order and unmarked in the RT order, is attributed to allmorphy in the realisation of the dative P head. Evidence from modern Dutch, quantitative historical evidence from Middle English, and data from High German dialects supports the notion that the P-head that marks recipients in dative shift languages is parallel to the synthetic dative case in languages that have synthetic dative case. For dative shift, the overt allomorph (English \textit{to}) is the default realisation, which is blocked by a null allomorph when the P-head is linearly adjacent to the verb.

This linear adjacency property was sensitive to prior copies of the theme (e.g., ``John gave \sout{the book that he loves} *(to) Mary the book that he loves). This data point granted insight into the ordering of morphological processes, namely that checking linear adjacency for allomorphy must occur before (at least some) copies are deleted.

Considering passive data, Swedish provided overt evidence for the idea that P-incorporation licenses recipient passivisation, since only verbs with overt prefixes (argued to be the reflex of incorporated dative P) allow recipient passivisation. Following \cite{Alexiadou.2014}, Dutch and German were also used to provide evidence in the form of auxiliary variation in the availability of recipient passivisation. The difference between auxiliary variation in OV and prefix vs. non-prefixed verbs in VO languages was used to explore the nature of P-incorporation. This sensitivity to object order provided tangential evidence in favour of a roll-up analysis of object--verb ordering. P-incorporation always moved out of the PP into the next highest functional head; for OV, the next highest head is the auxiliary (after rollup), while, for VO, it is the main verb. This analysis was also supported with quantitative historical English data, where nominative recipient passivisation and pseudopassivisation were found to enter the language in parallel.

Theme passivisation, which is a violation of locality without further syntactic operations (since the recipient intervenes between T and the theme in base generated position), provided evidence for a number of distinct licensing operations. VP-internal scrambling solved the locality problem by moving the theme over the recipient. Recipient cliticisation solved the problem by moving the recipient out of the way of the theme.

When the neither the theme nor the recipient moved, languages varied as to how they solved the locality violation. Most languages only allowed DPs to move to subject position (Icelandic exceptionally allowing dative PPs in subject position). Assuming that PPs were not a valid target for movement, languages treated the intervention differently. For some languages (i.e., some English dialects, Icelandic, German and Dutch), T was able to consider multiple arguments, allowing it to see the theme and to directly move the theme directly to spec-TP. In other languages (e.g., modern American English, Danish and Swedish), T could only consider the highest argument, which caused passivisation to fail, requiring one of the previously mentioned licensing techniques.

\section{Implications}
The conclusions discussed above were only able to be supported by combining data from multiple languages. One clear example of this phenomenon was the typological argument for the RT base order (namely that all Germanic languages have the RT order, but Icelandic lacks the TR order). This type of argument is only possible because a wide variety of languages were surveyed.

Another example is the evidence for P-incorporation. Here Swedish, with its surface realisation of incorporated P-heads, provided the clearest evidence in favour of P-incorporation. However, the same complicated data that makes Swedish ideal for studying P-incorporation makes it a less than ideal case study for arguing for the morphological underpinnings of dative shift. Having access to data from a variety of different languages enabled using the clearest supporting evidence for each point being made, which would have been impossible if only data from one language was used.

Also, diachronic data provided independent support for the analysis. While synchronic evidence from scopal ambiguities and control into purpose clauses is strongly suggestive of the dative PP + applicative analysis, the quantitative analysis of how dative shift developed in the Middle and Early Modern English periods provided a distinct type of evidence that the dative PP + applicative analysis must have been true at an earlier stage of English. Since it was true at an earlier stage, and the modern data is still compatible with the analysis, the most parsimonious explanation is to maintain the analysis throughout the history of English.

The history of recipient passivisation in English provided an example of the interplay between the grammar and performance in cases of variation. With ditransitives, the grammar has a number of different mechanisms for generating grammatical passives (recipient passives and a number of distinct types of theme passives). In some cases (as with nominative recipient passivisation), the necessary mechanism may be commonly used outside of ditransitive passivisation (in this case in pseudopassives). However, the availability of the mechanism in the grammar proves insufficient to generate frequent use of the mechanism in production. For most of modern British English, recipient passivisation was grammatical, but strongly dispreferred. This suggests a two-tiered status of operations within the category of grammatical operations: (a) last-resort operations (grammatical, but only used when necessary) and (b) free-use operations (grammatical and not dispreferred).

The stability through the history of English is a sub-case of the stronger point argued for here. I showed that the dative PP + applicative analysis was compatible with synchronic and diachronic data from all of the major Germanic languages. I proposed that this provides support for a subset of the strong UTAH hypothesis \citep{Baker.1988}, namely that \textbf{all} languages share a universal argument structure. In other words, languages cannot vary the base generation positions assigned to arguments. This conclusion resembles the claims about the deep structure of earlier generative traditions \citep{Chomsky.1965,Chomsky.1981}, i.e., that syntax is fundamentally about performing transformations on a universal (possibly non-linguistic) basic structure. The extension of this hypothesis to the full strong UTAH is quite falsifiable, in so far as it predicts that the dative PP + applicative analysis should be able to account for recipient data from all natural languages including those outside Germanic.

A final larger point that this dissertation highlights is the advantage of modularity in approaching linguistic complexity. The Germanic languages showed a large degree of surface variation in the position and surface marking of recipient arguments across active and passive sentences. By distributing the burden of accounting for the surface complexity to the interaction of syntactic and morphological processes, a globally parsimonious account was achieved.

The clearest example of this is the analysis of dative shift. Empirically, dative shift is characterised by an unmarked recipient when adjacent to the verb and a marked recipient elsewhere (with a small number of categoriseable surface exceptions). A purely syntactic approach would need to account for how the syntax is able to identify the linear adjacency of two elements as well as accounting for the difference between marked and unmarked recipients. By using modularity, the syntax of dative shift languages can be made identical to that of non-dative shift languages, with only the addition of the allmorphy operation, which is independently necessary and independently known to be sensitive to linear adjacency. Only because the variation could be attributed to the morphology was the strong UTAH claim explained above possible; languages might have widely varying surface structures that reflect morphological obfuscation of a unified syntactic underpinning.

%\bibliography{diss}
