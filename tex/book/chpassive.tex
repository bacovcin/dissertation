\chapter{Passive Syntax of Recipient Ditransitives}\label{ch:passive}
\section{Introduction}
This chapter analyses how data from the passivisation of recipient ditransitives can be explained by the dative PP + applicative analysis. Passivisation, as a movement operation, is a useful probe in studying the internal structure of clauses. As discussed in Chapter \ref{ch:theoryback}, the assignment of subject properties to arguments shows sensitivity to case and locality issues that reflect on the case and syntactic positions of arguments.

This chapter will start by analysing recipient passivisation. Since (as argued in Chapter \ref{ch:active}) the recipient always receives dative case, which is represented by a PP, full recipient passivisation (with a nominative recipient) requires dative--to--nominative conversion. Building on the analysis of \cite{Alexiadou.2014}, I propose that dative--to--nominative conversion involves incorporation of the P head into a verbal element, which turns the recipient into a bare DP and makes it available for structural case assignment. The dative PP analysis assumes that the difference between inherent/lexical case and structural case is the presence of the PP layer \citep{Bayer.2001}. Evidence for the incorporation analysis will be brought from recipient passives in German, Dutch and Swedish. In the next subsection, I discuss oblique subjects in Icelandic and Faroese and argue for the parameterisation of the validity of PP subjects.

The second section focuses on theme passivisation. I show that there are two mechanisms by which the locality constraint can be violated, namely: (a) restricting subject movement to DPs and allowing T to consider multiple arguments for subject movement and (b) moving either the theme or the recipient so that the theme is the highest argument in an A-position. The first mechanism is a consequence of the P-incorporation analysis for recipient passivisation. If P-incorporation is unavailable, then the recipient is not a valid target for nominative case assignment. If T requires nominative subjects, then only the theme would be a valid subject. I show that some languages allow T to consider multiple arguments and move the theme across the recipient from its base generated position. In other languages, only the highest argument can be considered and either the recipient or theme must move in order for theme passivisation to occur.

\section{Recipient Passivisation}
In this dissertation, recipient passivisation is defined as cases where the recipient is in the higher subject position (i.e., spec-TP). There are two sub-cases of this situation, which will be addressed in turn. The first (dative-to-nominative raising) is a case where the recipient receives nominative case and has all subject properties. The second (oblique subjects) is a case where the subject properties are split with the recipient occupying the higher subject position, but the theme receiving nominative case.

\subsection{Dative-to-Nominative Raising}

The dative PP + applicative analysis claims that all recipients are dative case marked. Therefore, any example of a nominative recipient is an example of dative-to-nominative conversion. As will be shown below, this property can be seen on the surface in a number of Germanic languages (namely Faroese, Halsa Norwegian, and High German). 

Faroese\footnote{Faroese is currently changing from having oblique subjects like Icelandic (discussed below) and having dative-to-nominative raising. The data presented below are from the speakers that have adopted the new grammar with dative-to-nominative raising (see \cite{Eyorsson.2012} for a discussion of this change and survey data attesting to the existence of this sub-population).} and Halsa Norwegian both show the availability of dative-to-nominative conversion, although they do not elucidate the mechanism by which dative-to-nominative conversion occurs. Both languages have a clear morphological distinction between dative and accusative case:

\begin{exe}
	\ex Faroese:\label{ex:far-case}
		\begin{xlist}
			\ex[ ]{\gll Teir góvu \textbf{gentuni} telduna \\
				they gave \textbf{girl-the.DAT} computer-the.ACC \\
			            \trans `They gave the girl the computer.'}
				    \ex[*]{\gll Teir góvu \textbf{gentuna} telduna \\
				they gave \textbf{girl-the.ACC} computer-the.ACC \\
			    \trans `They gave the girl the computer.'}
		\end{xlist}
		\ex Halsa Norwegian:\label{ex:halsa-case}
	\begin{xlist}
		\ex \gll Ho erta \textbf{katt\aa} \\
		she teased \textbf{cat.DEF.ACC} \\
			\trans `She teased the cat.'
			\ex \gll Ho ga \textbf{katt\aa} inn mat \\
			she gave \textbf{cat.DEF.DAT} food \\
			\trans `She gave the cat food.'
	\end{xlist}
\end{exe}

Both languages also allow the dative argument to surface as nominative in the passive. Oblique subjects (of ditransitive passives) are marginal/ungrammatical \citep{Eyorsson.2012}:

\begin{exe}
	\ex Faroese:\label{ex:far-pass}
	\begin{xlist}
		\ex[ ]{\gll Gentan bleiv givin telduna\\
			    girl-the.NOM was given.NOM computer-the.ACC\\
		    	    \trans `The girl was given the computer.'}
		\ex[??]{\gll Gentuni bleiv givin ein telda\\
			    girl-the.DAT was givn.NOM a.NOM computer.NOM\\
		    	    \trans `The girl was given the computer.'}
	\end{xlist}
	\ex Halsa Norwegian:\label{ex:halsa-pass}
	\begin{xlist}
		\ex[ ]{\gll Hainn vart gjevinn ei skei.\\
He.NOM was given a spoon\\
\trans `He was given a spoon.' \cite[ex 50c]{Eyorsson.2012}}
\ex[*]{\gll Hånnå vart gjevinn ei skei.\\
He.DAT was given a spoon\\
\trans `He was given a spoon.' \cite[ex 50c]{Eyorsson.2012}}
	\end{xlist}
\end{exe}

In order to explain how dative-to-nominative conversion occurs, a theory of nominative case assignment needs to be given. As discussed in Chapter \ref{ch:theoryback}, I argue that all arguments marked with non-structural case are actually PPs, and that all and only arguments marked with structural case are bare DPs. Therefore, in order for an element to receive nominative case, it must be a bare DP. While the theme in recipient ditransitives is a DP, the recipient is a PP and thus should be unavailable for nominative case assignment. For it to become available, the PP layer must be removed.

In Chapter \ref{ch:theoryback}, I introduced the operation of P-incorporation as a means of converting PPs into DPs. This operation unites dative-to-nominative conversion with theories of pseudopassivsation, where passivisation of the object of preposition required incorporation of the preposition into the verbal domain \citep{Herslund.1984}. This section argues that both pseudopassivisation and nominative recipient passivisation rely on the same underlying mechanism of P-incorporation, however, the structural/semantic differences between prepositional objects (complements of the main verb) and recipients (specifiers of an applicative phrase) mean that pseudopassivisation and nominative recipient passivisation need not co-occur in the same language (or that the reflex of P-incorporation need not be the same across the two constructions in the same language).

P-incorporation moves the P-head from the specifier of the recipient -- itself in the specifier of the applicative phrase -- and adjoins it to the head of the nearest C-commanding phrase. As argued in the previous chapter, Swedish verbs with prefixes are derived via P-incorporation, since they do not license TR (theme-recipient) orders (\ref{ex:Swedish-complex-act}). After VP-internal scrambling, the theme would C-command the recipient and thus be the nearest C-commanding phrase, making the theme rather than the verb the target of P-incorporation. Since the verb \textit{erbjoda} `offer' is built from P-incorporation, if the dative P does not incorporate, the verb cannot be used (since it cannot be built).
	\begin{exe}
		\exr{ex:Swedish-complex-act} Swedish:
			\begin{xlist}
				\ex[ ]{\gll Han erbjöd Jan ett nytt jobb\\
				he.NOM offered John a new job\\
			\trans `He offered John a new job'}
				\ex[??]{\gll Han erbjöd ett nytt jobb til Jan\\
				he.NOM offered a new job to John\\
			\trans `He offered a new job to John'}
				\ex[*]{\gll Han erbjöd ett nytt jobb Jan\\
				he.NOM offered a new job John\\
			\trans `He offered a new job to John'}
			\end{xlist}
	\end{exe}

	After reviewing some more data, I show that the target site of P-incorporation also has implications for the structure of OV and VO clauses, since OV and VO languages show different reflexes of P-incorporation. Example \ref{ex:VO-Pincorp} shows how P-incorporation in VO clauses can lead to prefixed verbs as in Swedish.

\begin{exe}
	\ex P-incorporation (VO Word Order)\label{ex:VO-Pincorp}\\
			\xymatrix@=2pt{
			& VoiceP\ar@{-}[dl]\ar@{-}[dr]\\
			V+Appl+Voice\ar@{<-}[dd] && ApplP\ar@{-}[dl]\ar@{-}[drr]\\
			& PP_{\text{Recipient}}\ar@{-}[dl]\ar@{-}[dr] & & & \bar{Appl}\ar@{-}[dl]\ar@{-}[dr]\\
			\text{\sout{P}} & & DP_{\text{Recipient}} & \text{\sout{Appl}} & & VP\ar@{-}[dl]\ar@{-}[dr]\\
			&  &  &  & \text{\sout{V}} & & DP_{\text{Theme}}}
\end{exe}


Dutch and High German show how OV languages show different surface properties. Recipient passivisation is not normally available in Dutch or High German, instead the theme must receive nominative case (see below for further discussion of theme passivisation).

\begin{exe}
	\ex High German:\label{ex:hg-normal-pass}
\begin{xlist}
	\ex[ ]{\gll Ich glaube, dass \textbf{den} \textbf{Kindern} das Fahrrad geschenkt worden ist.\\
	I beleive that \textbf{the.DAT.PL} \textbf{children} the.NOM bicycle given become be.3sg\\
\trans `I believe that the children were given the bicycle.'}
\ex[*]{\gll Ich glaube, dass \textbf{die} \textbf{Kindern} das Fahrrad geschenkt worden sind.\\
I beleive that \textbf{the.NOM.PL} \textbf{children} the.ACC bicycle given become be.3pl\\
\trans `I believe that the children were given the bicycle.'}
\end{xlist}
\ex Dutch:\label{ex:dutch-normal-pass}
\begin{xlist}
	\ex[ ]{\gll De boeken \textbf{werden} haar aangeboden.\\
		the books \textbf{became.PL} her given\\
	\trans `The books were given to her.' \citep[ex. 5b]{Broekhuis.1994}}
	\ex[*]{\gll Zij \textbf{werd} de boeken aangeboden.\\
	she.NOM \textbf{became.SG} the books given\\
	\trans `She was given the books.' \citep[ex. 5c]{Broekhuis.1994}}
\end{xlist}
\end{exe}


However, when the passive auxiliary changes from \textit{werden} `become' to \textit{bekommen}/\textit{krijgen} `get', recipient passivisation becomes obligatory (\ref{ex:hg-get-pass} \& \ref{ex:dut-get-pass}). \cite{Alexiadou.2014} argue that the change in auxiliary is the direct reflection of P-incorporation, i.e., that \textit{werden} is the realisation of the passive on its own, while \textit{bekommen}/\textit{krijgen} is the realisation of the passive with the dative P incorporated. For High German, this is a clear case of dative-to-nominative conversion, since dative case is marked on the surface.

\begin{exe}
	\ex High German:\label{ex:hg-get-pass}
\begin{xlist}
	\ex \gll dass der Vater \textbf{der} \textbf{Tochter} ein Buch geschenkt hat\\
	that the.NOM father \textbf{the.DAT} \textbf{daughter} a.ACC book given has\\
	\trans `that the father gave the daughter a book.'
	\ex \gll dass \textbf{die} \textbf{Tochter} von dem Vater ein Buch geschenkt bekommen hat\\
	that \textbf{the.NOM} \textbf{daughter} by the father a.ACC book given got has\\
	\trans `that the daughter got given a book by her father \cite[183]{Draye.1996}.'
\end{xlist}
\ex Dutch:\label{ex:dut-get-pass}
\gll \textbf{Zij} kreeg de boeken (van mij) aangeboden.\\
\textbf{she.NOM} got the books (by me) given\\
\trans `She was given the books (by me).' \citep[ex. 7]{Broekhuis.1994}
\end{exe}

For German and Dutch, there is evidence that the \textit{bekommen}/\textit{kreign} passive is actually a passive construction. This evidence comes from the availability of by-phrases (as seen above) and productivity \citep{Broekhuis.1994}. In Dutch, the construction can be productively used with almost all verbs that assign a recipient or addressee theta role. The only exception is the verb \textit{geben} `give', which \cite{Broekhuis.1994} argue is ruled out on pragmatic grounds, since `get given' is pleonastic for `get'.

As suggested above, another case of overt incorporation can be seen in Danish pseudopassivisation (\ref{ex:dan-pseudopass}). \cite{Herslund.1984} argued that P-incorporation for pseudopassivisation in Danish appears as prefixed verbs rather than P-stranding as in English. 

\begin{exe}
	\ex Danish:\label{ex:dan-pseudopass}
	\begin{xlist}
		\ex[ ]{\gll Revisionen blev \textbf{p\aa begyndt} i maj\\
		revision-the was \textbf{on-begun} in May\\
		\trans `The revision was begun in May'}
		\ex[*]{\gll Revisionen blev \textbf{begyndt} \textbf{p\aa} i maj\\
		revision-the was \textbf{begun} \textbf{on} in May\\
		\trans `The revision was begun in May'}
	\end{xlist}
	\ex English:\label{ex:eng-pseudopass} 
	\begin{xlist}
		\ex[*]{The bed was inslept.}
		\ex[ ]{The bed was slept in.}
\end{xlist}
\end{exe}

Swedish provides evidence that nominative recipient passivisation is derived from P-incorporation, since passivisation possibilities pattern with the lexical split between prefixed and non-prefixed verbs. I showed in Chapter \ref{ch:active} that Swedish shows a split between ditransitive verbs with and without prefixes. There, I suggested that the prefix verbs represented the incorporation of dative P into the verb. This explanation is consonant with the Swedish passivisation data; only verbs with prefixes allow recipient passivisation (see below for theme passivisation strategies in Swedish). Recipient passivisation is not available for non-particle verbs \citep{Lundquist.2006}.\footnote{\cite{Lundquist.2004} shows that there are some exceptional cases where recipient passivisation is available with a verb like \textit{ge} `give', namely ``where the agent has less control over the outcome of the event'' (e.g. ``John was given the opportunity to succeed''). While these examples are marginal, they do not substantially undermine the argument made here. They show that Swedish marginally allows null P-incorporation (as will be proposed for other Germanic languages below) and only prefers (as opposed to requires) overt P-incorporation. Putative cases of reverse type of counter-example (i.e., theme passivisation with prefixed verbs) is addressed later in the chapter.} 

\begin{exe}
	\ex Swedish:\label{ex:swe-part}
	\begin{xlist}
		\ex[ ]{Particle Verb:
		\gll Han erbjöds ett nytt jobb\\
			he.NOM offered.PASS a new job\\
			\trans `He was offered a new job (\citealt{Anward.1989}, \citealt{Lundquist.2006}).'}
		\ex[*]{Non-Particle Verb:
		\gll Pelle gavs ett äpple\\
			Pelle gave.PASS a apple\\
			\trans `Pelle was given an apple (\citealt{Anward.1989}, \citealt{Lundquist.2006}).'}
\end{xlist}	
\end{exe}


Most of the Germanic languages do not show any overt signs of P-incorporation (including Faroese and Halsa Norwegian discussed above). Given the morphological description of dative case realisation discussed in Chapter \ref{ch:active}, this is not surprising. Most of these langauges (e.g., Danish, Standard Norwegian and English) seem to share the distribution of null dative case realisation with English (i.e., the null realisation is restricted to contexts locally adjacent to the verb). When the P-head incorporates, it is maximally adjacent to the verb. Thus, a null realisation is expected.

\begin{exe}
	\ex English: He was P=$\emptyset$-given \sout{he} the ball.\label{ex:eng-null-incorp}
	\ex Standard Norwegian:\label{ex:nor-null-incorp}
	\gll Han vart P=$\emptyset$-gitt \sout{hann} ein medalje\\
	he.NOM was given \sout{he.NOM} a medal\\
	\trans `He was given a medal.'
	\ex Danish:\label{ex:dan-null-incorp}
	\gll Han blev P=$\emptyset$-tilbudt \sout{hann} en stilling\\
	he.NOM was offered \sout{he.NOM} a job\\
	\trans `He was offered a job.'
\end{exe}

This P-incorporation process seems to be sensitive to OV vs VO word order, a generalisation observed in \cite{Sprouse.1995}. In languages like Dutch and German with OV word order, P-incorporation happens with the auxiliary, which is the element to the left of the recipient, and thus recipient passivisation is restricted to cases with a different auxiliary. In VO languages, like Swedish, the verb is the element to the left of the recipient, and thus recipient passivisation is restricted to particle verb cases. In many of the languages, P-incorporation is invisible, since the P element has a null realisation.

The OV/VO split follows from the remnant raising analysis of object linearisation \citep{Biberauer.2004,Biberauer.2005,Wallenberg.2009}. Under these analyses, the VP scrambles to be above VoiceP after the verbal head has already moved into VoiceP via head movement. Under this structure, TP (or AuxP) is the nearest c-commanding head to the recipient and thus the target for P-incorporation. In VO languages, the VP with the recipient inside of it stays low and thus VoiceP is the next highest head, leading to the Swedish case where the verbal prefixes reflect P-incorporation.

\begin{exe}
	\exr{ex:VO-Pincorp} VO:\\
			\xymatrix@=2pt{
			& VoiceP\ar@{-}[dl]\ar@{-}[dr]\\
			V+Appl+Voice\ar@{<-}[dd] && ApplP\ar@{-}[dl]\ar@{-}[drr]\\
			& PP_{\text{Recipient}}\ar@{-}[dl]\ar@{-}[dr] & & & \bar{Appl}\ar@{-}[dl]\ar@{-}[dr]\\
			\sout{P} & & DP_{\text{Recipient}} & \sout{Appl} & & VP\ar@{-}[dl]\ar@{-}[dr]\\
			&  &  &  & \sout{V} & & DP_{\text{Theme}}}
	\ex OV:\\\xymatrix@=2pt{
			& AuxP\ar@{-}[dl]\ar@{-}[drr]\\
			Aux\ar@{<-}[ddd]&&&VoiceP\ar@{-}[dl]\ar@{-}[dr]\\
			&& ApplP\ar@{-}[dl]\ar@{-}[drr]&&\text{V+Appl+Voice}\\
			&PP_{\text{Recipient}}\ar@{-}[dl]\ar@{-}[dr] & & & \bar{Appl}\ar@{-}[dl]\ar@{-}[dr]\\
			\sout{P} & & DP_{\text{Recipient}} & \sout{Appl} & & VP\ar@{-}[dl]\ar@{-}[dr]\\
			& & & & \sout{V} & & DP_{\text{Theme}}}

\end{exe}


In addition to the synchronic/typological discussion above, there are also diachronic reasons to prefer the P-incorporation account. \cite{Falk.1997} and \cite{Allen.1999}, and \cite{Platzack.2005} (following earlier literature) suggest that nominative recipient passivisation is made available by the reanalysis of bare dative recipients as being marked with accusative case (and thus possible targets to raise as nominative subjects). This explanation predicts that nominative recipient passives should become available shortly after the loss of synthetic dative case (since there is no longer any morphological evidence for a dative--accusative distinction). In the discussion of the diachronic data below, I show that in all cases that have been investigated, nominative recipient passivisation only becomes available hundreds of years after the loss of synthetic dative case.

The diachrony of both the loss of synthetic dative case and the availability of nominative recipient passivisation have been examined for English and Swedish. For English, \cite{Allen.1999} shows that the last remnants of synthetic dative case were lost in all English dialects by the middle of the 12th century. However, she carefully shows that the first unambiguous example of nominative recipient passivisation (instead of topicalised dative passives or dative subjects) occurs around 1375, nearly 200 years after dative case has been lost (this is examined in more detail in chapter \ref{ch:diachron}). \cite{Falk.1997} shows that nominative recipient passivisation only becomes available in the end of the 19th century (she does not discuss the split between different verb classes). This is also about 200 years after the loss of synthetic dative case in the 17th century. 

I already suggested that the analysis of nominative recipient passivisation that relies on reanalysis of recipients as being introduced with accusative case in the active has no way to explain why the reanalysis does not occur until almost 200 years after the loss of the morphological forms that would have provided evidence to language learners about the case distinction. In other words, why would language learners keep positing dative case without any surface evidence if an accusative analysis was possible? 

Under the analysis proposed here, the answer to this question is that an accusative (re)analysis is not possible. Recipients are universally marked with dative case (i.e., this is not subject to variation and thus does not need to be learned as part of language acquisition). Nominative recipient passivisation requires the learner to posit an operation of P-incorporation and associate the operation with the dative P (and possibly particular verbs as in the Swedish case). When the dative P incorporates, the recipient becomes a bare DP, which is then available to raise as a nominative subject. The existence of a null realisation of P after the loss of synthetic dative case (i.e., the null allomorph in dative shift) \textbf{licenses} a language learner to posit P-incorporation for datives, since there is no surface evidence about the location of the null allomorph. The learner has no evidence about where the P-head is (since it is silent), so P-incorporation is a possible analysis of the data. However, since the learner is required to posit an independent syntactic operation (P-incorporation), it is not unexpected that there might be a long lag between the development of a situation that licenses the change (i.e., the development of a null allomorph) and learners actually implementing the change (i.e., positing P-incorporation as a valid operation in the language for dative Ps).

As discussed above, the P-incorporation account also explains why Dutch does not allow nominative recipient passivisation with the standard passive auxiliary (namely because as an OV language P-incorporation involves incorporation into the auxiliary triggering a different allomorph of the passive auxiliary). Under the case reanalysis account, it is unclear why Dutch should not have undergone case reanalysis allowing nominative recipient passivisation across--the--board (even with the standard passive auxiliary).

Finally, the existence of nominative recipient passives in languages with synthetic dative case marking (Faroese and Halsa Norwegian) needs to be explained. Case reanalysis cannot account for these languages, since they transparently do not have accusative recipients in the active. The P-incorporation account, however, is compatible with the data. The dative P that triggers the synthetic dative morphology can be incorporated in the passive and maintain its null realisation (since in most cases of synthetic datives the P is null and the case features are realised by concord on other elements in the DP). Positing P-incorporation in these languages should not easily occur, since there is overt evidence that the dative P is still attached to the recipient (in the form of synthetic dative case marking). For both Faroese and Halsa Norwegian, however, spontaneous positing of the operation is unnecessary, since both languages are spoken by populations who are in intense language contact with languages that already have P-incorporation. Halsa Norwegian is spoken in the context of Standard Norwegian. Faroese is under intense contact with Danish \citep{petersen.2010}. In these cases, P-incorporation can plausibly have been borrowed from the contact language.


\subsection{Oblique Subjects}
The previous subsection dealt with cases in which P-incorporation occurred. In that situation, the highest argument (i.e., the recipient) was available both for movement to subject position and nominative case assignment. Most of the rest of this chapter will focus on cases where P-incorporation does not occur. In these situations, the recipient is not available for nominative case assignment. This subsection describes cases where the two subjecthood properties split: the recipient moves to a higher subject position (oblique subject) and the theme receives nominative case (nominative object) and triggers verbal agreement. This split can be encoded in the featural content of T, where the T head that licenses subject properties has the movement and case assignments distinct. See the discussion of theme passivisation below for further discussion of how the assignment of nominative case to the theme proceeds.

\cite{Zaenen.1985} gives the classic presentation of the evidence in Modern Icelandic for oblique subjects. In Icelandic, only subjects can occupy the post-finite verb position:

\begin{exe}
	\ex Icelandic, Topicalization:\label{ex:ice-topic}
\begin{xlist}
	\ex \gll Refinn skaut \textbf{Ólafur} með  þessari byssu.\\
	fox.DEF.ACC shot \textbf{Olaf.NOM} with this shotgun\\
\trans `The fox, Olaf shot with this shotgun \citep[ex. 19a]{Zaenen.1985}.'
\ex[*]{\gll Með  þessari byssu skaut \textbf{refinn} Ólafur.\\
	with this shotgun shot \textbf{fox.DEF.ACC} Olaf.NOM\\
\trans `The fox, Olaf shot with this shotgun \citep[ex. 19b]{Zaenen.1985}.'}
\end{xlist}
\ex Icelandic, Direct Question:\label{ex:ice-dq}
\begin{xlist}
	\ex \gll Hafði \textbf{Sigga} aldrei hjálpað Haraldi?\\
	had \textbf{Sigga.NOM} never helped Harald.DAT\\
\trans `Had Sigga never helped Harald \citep[ex. 20b]{Zaenen.1985}?'
\ex[*]{\gll Hafði \textbf{Haraldi} Sigga aldrei hjálpað?\\
	had \textbf{Harald.DAT} Sigga.NOM never helped\\
\trans `Had Sigga never helped Harald \citep[ex. 20c]{Zaenen.1985}?'}
\end{xlist}
\end{exe}

In cases of ditransitive passives, the dative phrase is capable of filling this position patterning with undisputed subjects:

\begin{exe}
	\ex Icelandic, Ditransitive Topicalization:\label{ex:ice-dittop}
\begin{xlist}
	\ex \gll Um veturinn voru \textbf{konunginum} gefnar amb\'{a}ttir.\\
In winter.the were \textbf{king.the.DAT} given slaves.NOM\\
\trans `In the winter the king was given slaves \citep[ex. 47a]{Zaenen.1985}.'
\end{xlist}
\ex Icelandic, Ditransitive Direct Question:\label{ex:ice-ditdq}
\begin{xlist}
	\ex \gll Voru \textbf{konunginum} gefnar amb\'{a}ttir?\\
were \textbf{king.the.DAT} given slaves.NOM\\
\trans `Was the king given slaves \citep[ex. 48a]{Zaenen.1985}?'
\end{xlist}
\end{exe}

Note that in Icelandic, the theme in clauses with oblique subject receive nominative case. In Faroese, there is interspeaker variation in the grammaticality of oblique subjects, but at least some speakers find sentences with dative subjects and accusative objects grammatical. \cite{Eyorsson.2012} had a number of Faroese speakers give acceptability judgements to passive sentences with dative subjects and accusative objects as in (\ref{ex:faroese-datacc-pass}). He found that 17.7\% of speakers found such sentences grammatical, as opposed to 61.3\% who found it ungrammatical. Since almost 1 in 5 speakers find such sentences grammatical, I propose that they are valid output of a least one version of the grammar of Faroese.

\begin{exe}
	\ex Faroese, DAT-ACC passives:\label{ex:faroese-datacc-pass}
	\gll Gentuni bleiv givið eina teldu.\\
the.girl.DAT was given a.ACC computer.ACC\\
\trans `The girl was given a computer. \cite[ex 45b]{Eyorsson.2012}'
\end{exe}

As explained in Chapter \ref{ch:theoryback}, this difference between Icelandic and Faroese can be captured by parameterising the ability of T to look at multiple arguments. Both languages allow PP subjects, but differ in how many arguments T can consider in assigning subject properties. In Icelandic, T can find the PP recipient raise it to subject position and then keep looking further into the clause to ultimately assign nominative case to the theme. In Faroese, T is only allowed to look at one argument and moves the recipient to subject position; the theme retains accusative case, since T is unable to look past the recipient and assign nominative case to it.

\subsection{More on PP subjects}
The above analysis claimed that oblique subjects represented PPs filling subject position, which at first glance seems to be a very difficult claim to accept. Even Icelandic, the paradigm case of oblique subjects does not allow overt prepositional arguments into subject position: 

\begin{exe}
	\ex Icelandic:\label{ex:ice-ppsbj}
	\begin{xlist}
		\ex[*]{\gll {\'{I} gar} var um þessa konu oftast talað\\
		yesterday was about this woman often talked\\
		\trans `Yesterday, this woman was often talked about'}
		\ex[*]{\gll {\'{I} gar} var \'{i} r\'{u}minu sofið\\
		yesterday was in bed.DEF slept\\
		\trans `Yesterday, the bed was slept in.'}
	\end{xlist}
\end{exe}

The unification of oblique arguments and prepositional phrases allows for a unification of the explanation of why PP subjects and oblique subjects are both so rare crosslinguistically (i.e., because they are the same thing syntactically). Below, I present evidence from Afrikaans that oblique subjects are PPs, since in that language the P-head in oblique subjects is realised overtly. I then conclude this subsection by describing why recipient PPs can become subjects, but most other PPs cannot (e.g., in Icelandic).

Afrikaans provides additional evidence that oblique subjects should be analysed as PPs, by having morphologically clear recipient PPs in subject position in ditransitive passives. According to \cite{Stadler.1996}, Afrikaans has the standard V2 subject position. As discussed above for Icelandic, only subjects are allowed to immediately follow the finite verb in cases where either the sentence is V1 (e.g. yes/no questions) or where there is a topicalised element, unlike in Dutch and German, where the subject position need not be filled. In the passive, the recipient patterns as a subject occurring after the finite verb in both V1 constructions and with a topicalised element, even when it is prepositionally marked:
\begin{exe}
	\ex Afrikaans: \label{ex:af-rec-pass1}
\begin{xlist}
\ex \gll Is aan hom ooit 'n geskenk gegee?\\
Was to him ever a present given.\\
\trans `Was he ever given a present \citep[ex. 49]{Stadler.1996}?'
\ex \gll Gister is aan hom `n klomp geld gegee.\\
Yesterday was to him a {lot of} money given.\\
\trans `Yesterday he was given a lot of money \citep[ex. 50]{Stadler.1996}.'
\end{xlist}
\end{exe}

When the recipient raises, leaving it unmarked (if a full noun phrase) or with nominative case (if a pronoun) is marginal. The preferred construction is for the recipient to be marked with \emph{aan} `to':

\begin{exe}
	\ex Afrikaans: \label{ex:af-rec-pass2}
\begin{xlist}
\ex[?]{\gll hy is `n present gegee.\\
he was a present given\\
\trans `He was given a present \citep[ex. 35]{Stadler.1996}.'}
\ex[ ]{\gll Aan hom is `n present gegee.\\
to him was a present given\\
\trans `He was given a present \citep[ex. 44]{Stadler.1996}.'}
\end{xlist}
\end{exe}

If oblique recipients are just PPs, why are they able to become subjects when other PPs cannot in Icelandic? I propose that the difference comes not from the internal structure of the PPs, but instead from their clausal position. Adjunct PPs can be excluded from raising to subject position (on the assumption that only arguments can become subjects). This prohibition holds with respect to pseudo-passivisation in English, adjunct PPs do not license pseudo-passives \citep{Hornstein.1981,Baker.1988}. However, argument PPs (the kind of PPs that license pseudo-passivisation in English) are also not grammatical subjects in Icelandic.

Under the analysis presented above, this cannot be because PPs cannot be subjects, since I argued that oblique subjects represent PP subjects. I propose that instead of being a property of the PPs that differs, this asymmetry is caused by a difference in the syntactic location of the two arguments. In particular, I argue that the main verb is a syntactic barrier (or phase head), generally prohibiting material in its complement from moving \citep{Chomsky.2001}. Thus most argument PPs cannot raise, because they are too deeply embeded. Pseudo-passivisation is available on the assumption that incorporation is a technique for moving an element past a syntactic barrier (see \citealt{Alexiadou.2013b} and citations therein). 

Under the standard assumption that themes are in the complement of main verbs, this should also rule out passivisation of standard monotransitives. However, this dissertation has already committed to a different location for themes. This follows from the combination of the structure of themes in prepositional object constructions (i.e., in the specifier of the main verb or some higher functional projection) and the requirements of UTAH (i.e., that there be only one location for a given thematic role). Since themes are in a specifier above the main verb, the fact that the main verb is a barrier to movement is irrelevant; the theme is generated beyond the barrier.

\begin{exe}
	\exr{ex:POC} Prepositional Object Construction: \\
\xymatrix@=2pt{
	& VP\ar@{-}[dl]\ar@{-}[dr] \\
 DP_{\text{Theme}} && \bar{V}\ar@{-}[dl]\ar@{-}[dr]\\
 &  V & & (PP_{\text{Goal}})}
\end{exe}

To summarise this section, recipient passivisation arises in two ways. First, P-incorporation licenses nominative recipient passivisation. When the recipient receives nominative case, the theme remains accusative, which shows that accusative case is licensed for themes in ditransitive passivisation. Secondly, there is variation in whether PPs are valid subjects. When PPs are valid subjects, oblique subjects arise. Further variation in the number of arguments T can consider for assigning subject properties determines whether the theme is a nominative or accusative object. Finally, other argument PPs cannot raise to subject position even in languages where PPs are valid subjects, because they are too deeply embedded underneath the finite verb.

\section{Theme Passivisation}
Theme passivisation occurs when the theme is in the higher subject position (i.e., spec-TP). In these case, the theme always receives nominative case (i.e., there are no oblique theme subjects). However, the theme being in subject position is a violation of locality without any intervening operation, since the recipient is always base generated higher than the theme. This section addresses two mechanisms by which the locality violation can be licensed: case sensitivity (a type of relativised minimality) and movement (of either the theme or the recipient).

\subsection{Case Licensed Locality Violation}
This subsection deals with the situation where the recipient's P-head does not P-incorporate, oblique subjects are not licensed, and no movement operation has altered the initial structure. In order for oblique subjects to be prevented, movement to subject position in these cases must be restricted to DPs (i.e., PPs are not valid subjects). The recipient is an intervener between T and the theme. Whether or not the theme can be seen depends on whether or not T can look at multiple arguments and bypass the recipient to find the theme. In cases where T can view multiple arguments, the theme receives nominative case and moves from its base merged positions directly to subject position in the specifier of T. I call this process of moving the theme past the recipient \textbf{direct theme passivisation}.

Evidence for direct theme passivisation comes from a number of different Germanic languages. One piece of evidence that nominative case assignment can target the theme in its base merged position comes from German and Dutch. In both of these languages, there is no requirement that the higher subject position be filled \citep{Besten.1990}. Nominative elements in all clauses can stay in their base merged positions. In the passives of ditransitives with the normal passive auxiliary \textit{werden}, only the theme can receive nominative case (for the behaviour with alternative passive auxiliaries, see above). The nominative theme can be in its base merged position, underneath the recipient, suggesting that the nominative case assignment occurred past the recipient, which was invisible since it was a PP.

\begin{exe}
	\ex High German:\label{ex:hg-insitu-sbj}
\gll Ich glaube, dass den Kindern das Fahrrad geschenkt worden ist.\\
I beleive that the.DAT.PL children the.NOM bicycle granted become be.3sg\\
\trans `I believe that the child was granted the bicycle.'
\ex Dutch:\label{ex:dut-insitu-sbj}
\gll Er werd mij een boek gegeven.\\
There became.3sg me a book given\\
\trans `A book was given to me. \cite[pg 245]{Donaldson.2008}'
\end{exe}

Certain dialects of British English and historical dialects of English also provide evidence for direct theme passivisation. In these dialects, theme passivisation can occur with bare recipients (\ref{ex:baretpeng}). In Chapter \ref{ch:active}, I argued that lower copies of movement are able to intervene for determining the realisation of dative P, namely that they prevent the null allomorph from being realised. Thus, the existence of a null allomorph in theme passive contexts must be due to the theme moving to subject position from its base merged position without an intermediate stage of VP-internal scrambling.

\begin{exe}
	\ex English Dialects: \label{ex:eng-directtheme}
		\begin{xlist}
			\ex[ ]{\label{ex:baretpeng}The book was given P=$\emptyset$ the man \sout{the book}.}
			\ex[*]{\label{ex:badtpeng}The book was given \sout{the book} P=$\emptyset$ the man \sout{the book}.}
		\end{xlist}
\end{exe}

Icelandic also provides an example of direct theme passivisation. As discussed in Chapter \ref{ch:active}, Icelandic lacks VP-internal scrambling. However, theme passivisation is still a robust possibility in Icelandic. Either the theme is moving directly from its base merged position in the passive, or the passive shows evidence of a covert operation (VP-internal scrambling) that can only feed further transformation, but cannot occur on its own. While such operations have been argued for in the literature \citep[119ff]{Richards.2001}, direct theme passivisation gives a simpler analysis of Icelandic clauses, using only operations that are independently necessary.

\begin{exe}
	\ex Icelandic:\label{ex:ice-directthe}
\begin{xlist}
	\ex \gll Um veturinn voru \textbf{amb\'{a}ttin} gefin konunginum \sout{amb\'{a}ttin}.\\
	In winter.the was \textbf{slave-the.NOM} given king.the.DAT \sout{slave-the.NOM}\\
\trans `In the winter the slave was given to the king \citep[ex. 47b]{Zaenen.1985}.'
\ex \gll Var \textbf{amb\'{a}ttin} gefnar konunginum \sout{amb\'{a}ttin}?\\
were \textbf{slave-the.NOM} given king.the.DAT \sout{slave-the.NOM}\\
\trans `Was the slave given to the king \citep[ex. 48b]{Zaenen.1985}?'
\ex \gll \textbf{B\'{o}kin} var gefin J\'{o}ni \sout{B\'{o}kin}\\
\textbf{book-the.NOM} was given John.DAT \sout{book-the.NOM}\\
\trans `The book was given to John \citep{Holmberg.1995,Bardal.2001}.'
\end{xlist}
\end{exe}

However, not all languages have direct theme passivisation. Swedish verbs without particles (e.g., \textit{gav} `give'), Danish and Modern American English all prohibit theme passivisation with bare recipients.

\begin{exe}
	\ex Swedish (verbs without particles):\label{ex:swe-nopart-pass}
	\sn[*]{
	\gll Ett äpple gavs Pelle.\\
	 An apple gave.PASS Pelle.\\
	 \trans `An apple was given to Pelle (\citealt{Anward.1989},\citealt{Lundquist.2006}).'}
	 \ex Danish:\label{ex:dan-pass}
	 \sn[*]{
	 \gll En stilling blev tilbudt ham.\\
A job was offered him.OBL.\\
\trans `A job was offered to him \citep{Falk.1990}.'}
\ex Modern American English: *The book was given P=$\emptyset$ John \sout{the book}.\label{ex:amen-pass}
\end{exe}

These facts can be captured by restricting T to see only the first argument when it searches to assign nominative case and trigger subject raising. When T is restricted in this way, direct theme passivisation is ungrammatical, since the recipient intervenes between T and the theme. The recipient is not a valid target for subjecthood (since P-incorporation has not occurred and PPs are not valid subjects with these types of T). With this variety of T, some movement operation is necessary to allow passivisation in these cases, so that a bare DP argument (i.e., the theme) is the one argument that T is allowed to see.  
\begin{exe}
	\ex Modern American English: The book was given \sout{the book} P=to John \sout{the book}.\label{ex:amen-thepass}
\end{exe}

In summary, assuming that no movement has occurred (i.e., VP-internal scrambling or cliticisation), the recipient intervenes between T and the theme. If PPs are not valid subjects and subject movement is obligatory, then the theme needs to move. Some languages allow T to consider multiple arguments and thus trigger direct theme passivisation, moving the theme past the recipient. For other languages, T only considers the highest argument in an A-position, and the derivation crashes if that argument is not the theme. Icelandic shows that the variation in whether PPs are valid subjects can occur within the same language (i.e., it is a property of T heads not a language wide parameter setting). German and some British dialects show that T can see the theme past the recipient, while modern American English and Danish showed that this ability for T to consider multiple arguments is subject to variation.

\subsection{Movement Licensed Locality Violation}
As already hinted to above, VP-internal scrambling is a straightforward solution to the locality problem. If the theme has moved to be structurally higher than the recipient, then the theme is both available for nominative case assignment and the closest element from a locality standpoint. In English (and other languages with a similar case realisation pattern), this entails that the non-null realisation dative P head be used, since the null allomorph will not be licensed as the copy of the theme will intervene between P and the verb.

\begin{exe}
	\ex English:\label{ex:en-intervene}
	\begin{xlist}
			\ex[ ]{The book was given \sout{the book} P=to the man \sout{the book}.}
			\ex[*]{The book was given \sout{the book} P=$\emptyset$ the man \sout{the book}.}
	\end{xlist}
\end{exe}

VP-internal scrambling solves the locality problem by moving the theme. \cite{Anagnostopoulou.2003} shows that movement of the recipient is also able to obviate locality violations. Germanic languages show two different types of recipient movement. Anagnostopoulou argued that scrambling in Dutch (outside of the VP) is an A-bar operation that makes the recipient invisible for A-movement to subject position (i.e., standard relativised minimality). This type of scrambling can be identified by the placement of the argument to the left of VP-level adverbs (e.g. \textit{waarschijnlijk} `probably').

\begin{exe}
	\ex Dutch:\label{ex:dut-scram}
	\begin{xlist}
		\ex[ ]{\gll dat het boek \textbf{Marie} waarschijnlijk gegeven wordt\\
	that the book \textbf{Mary} probably given was\\
		\trans `that the book was probably given to Mary.'}
		\ex[?*]{\gll dat het boek waarschijnlijk \textbf{Marie} gegeven wordt\\
		that the book probably \textbf{Marie} given was\\
		\trans `that the book was probably given to Mary.'}
	\end{xlist}
\end{exe}

Anagnostopoulou shows that for other languages, e.g., Modern Greek, clitic doubling of the recipient also suffices. For Modern English (and many of the mainland Scandinavian languages), pronoun cliticisation seems to be a sufficient movement operation. Since many of these languages also have direct theme passivisation (see above), only usage data is able to show the existence of a pronoun cliticisation operation. For English dialects in which both direct theme passivisation and cliticisation are available, theme passivisation with bare full noun phrase recipients is rare in corpus data ($\sim$3\%--10\% of all passives). On the other hand, theme passivisation with bare pronominal recipients is common ($\sim$50\%).\footnote{Corpus estimates are drawn from historical data in COHA (1810--2009) \citep{Davies.2010} and the Parsed Corpora of Modern British English (1700--1910) \citep{Kroch.2010}. See the next subsection for a discussion of diachronic patterns and more detail on this construction.} The difference in usage rates suggests that there may be multiple operations at play (namely a rare direct theme passivisation operation and a more common cliticisation operation). Cliticisation of the recipient removes it from further movement and from being an intervener between T and the theme.

\begin{exe}
	\ex English Dialects (cliticisation): The book was given=me \sout{the book}.\label{ex:en-clitic}
\end{exe}

Another piece of evidence for cliticisation is the availability of theme passivisation in languages/dialects for which cliticisation is available, but direct theme passivisation is not. For many modern British English dialects from the Northwest of England (around Manchester and Liverpool), theme passives with bare recipients are only available with pronominal recipients (suggesting that cliticisation is the only available strategy) \citep{Haddican.2010,Myler.2011,Haddican.2012,Biggs.2015}.

\begin{exe}
	\ex English Dialects:\label{ex:endial-prosens}
	\begin{xlist}
		\ex[ ]{The book was given me.}
		\ex[*]{The book was given John.}
	\end{xlist}
\end{exe}

The locality problem in ditransitive passivisation occurs when PPs are not valid subjects, the recipient is a PP and the recipient is the highest argument in an A-position under T. This subsection described operations that removed the final clause of the problem, namely operations that make the recipient no longer the highest argument in an A-position. VP internal scrambling moves the theme above the recipient. Cliticisation moved the recipient to a non-A-position.

\subsection{Bare Recipient Theme Passives and Bare Recipient TR Actives}
This subsection brings additional evidence supporting the existence of direct theme passivisation and cliticisation as methods for generating theme passives in English. A pure locality approach would predict that theme passivisation could only be fed by the TR active word order and that for English bare recipient theme passives would occur only in grammars that had corresponding bare recipients in TR actives. \cite{Haddican.2010} and \cite{Haddican.2011,Haddican.2012} used experimental acceptability ratings to show that this correlation does not hold in the grammar of individual speakers of British English. They found three of the four logically possible grammars attested. 

\begin{exe}
\ex \cite[Table 2]{Haddican.2012}\\
 \begin{tabular}{|c|c|c|}
 \hline
 Grammar & Theme--Goal orders in active sentences & Theme passives\\
 \hline
 1 & * & * \\
 \hline
 2 & Ok & Ok \\
 \hline
 3 & Ok & * \\
 \hline
 4 (unattested) & * & Ok \\
 \hline
 \end{tabular}
\end{exe}% Haddican typology

They concluded that the unattested grammar should be inexpressible and formulated an analysis of British English to account for the ungrammaticality. They only investigated cases with pronominal themes, finding that \textit{it} licensed bare recipients better than \textit{they} as theme subjects. From this they concluded that there was a connection between the pronominal active cases and the passive cases, since a similar pattern vis-a-vis it and them has been found in actives. Since full noun phrase theme subjects occur with bare recipients in reported judgements for some dialects (and occur robustly in corpora as seen below), it seems difficult to maintain this claim, since the same dialects do \textbf{not} allow full themes in bare recipient TR actives. Also, the fourth grammar that was unattested in their investigation of Modern British dialects surfaces in the recent history of American English.

Using the Corpus of Historical American \citep{Davies.2010}, I investigated the loss of both bare recipient TR actives (e.g., ``I gave it John'') and bare recipient theme passives (e.g., ``It was given John'') in the history of American English. I extracted all tokens of the lemma GIVE + \textit{it} in order to examine the rate of bare TR actives. I also extracted all cases of the lemma BE + the passive participle of GIVE in order to study the loss of bare theme passives. In addition, a sample of 50 tokens of OFFER were extracted for each year (25 with a pronoun after the verb and 25 with a following determiner, noun or adjective). All of these tokens were coded by hand for the following features: whether the recipient was a pronoun or full noun phrase, whether the recipient was \textit{to}-marked or bare, and (for passive clauses) whether it was a theme or recipient passive.

Figure \ref{fig:amdata} shows the results from this study with respect to \textit{to}-marking. Bare marked recipients in TR actives were gone by 1940, while bare theme passives survived. After 1940, there are 22 examples of bare TR actives (out of 3098 tokens of TR actives with \textit{it} as the theme), all of which occur either in intentionally archaising contexts (e.g., translations of Norse sagas) or in direct quotations in plays or fiction. The restriction to archaising and quotational environments suggest that there was still an awareness of this use of bare recipient in theme-recipient actives, but that it was no longer a productive part of the grammar of Standard American English. At the same time, among all 2448 theme passive tokens after 1940, 7\% of all tokens for full noun phrase recipients and 39\% of all tokens with pronominal themes are bare. Theme passives with bare recipients were prominent across all genres, but most prevalent in fiction. The prominence of bare recipients in fiction may suggest that theme passives with bare recipients were considered colloquial.

\begin{figure}[ht!]

{\centering \includegraphics[width=\linewidth]{output/images/recpro-to-am} 

}

\caption{LOESS lines for \textit{to} use in Modern American TR actives (with pronominal themes) and theme passives, both with pronominal recipients.\label{fig:amdata}}\label{fig:amtoset-graph}
\end{figure}

In combination with the results from Haddican's studies, this suggests that there is a complete dissociation between bare TR in the active and bare theme passives. All possible combinations of bare vs \textit{to}-marked TR actives and bare vs \textit{to}-marked theme passives are attested in different dialects/time periods. The analysis presented here predicts this dissociation, since the presence of case-based restrictions on locality are neither connected to nor solely dependent on pronoun cliticisation.

In order to investigate whether theme passives with bare recipients were restricted to theme pronouns, I extracted all tokens of theme passives with bare recipients after 1940 for a total of 337 tokens (with both full noun phrase and pronominal recipients. I coded all of the extracted tokens for the status of the theme: theme pronoun, theme noun, or theme empty (for empty categories, mostly subject relative clauses, or where information about the theme was unavailable). I found that theme nouns made up the largest number of tokens 63\%, with theme empty following at 24\%, and finally theme pronouns (almost exclusively \textit{it}) at 13\%. So, \textit{it} predominated among pronouns, probably because it is the most common theme pronoun. However, theme pronouns in general were the least likely to occur with bare recipients, probably because pronouns are rarer than full noun phrases in general in written text. The ample evidence for full noun phrase theme subjects with bare recipients reinforces the fact that bare recipients in active and passive clauses are unrelated.

To summarise, a logical conclusion on seeing theme passives with bare recipients (e.g., ``The book was given John'') would be to conclude that they derived from TR actives with bare recipients (e.g., ``I gave the book John''). In this section, I point out two problems with this conclusion. First, in Early Modern British English (and Early 19th century American English), bare recipients only occur with pronominal themes in the active but occur with full noun phrase themes in the passive. Secondly, mid-20th century American English provides an example of a language with bare recipient theme passives that lacks bare recipient TR actives. In other words, a purely locality based account of ditransitive passivisation (where theme passives always derives from TR actives) is not tenable.

\subsection{Swedish Verbs and Theme Passivisation}
As discussed above, Swedish presents one of the clearest cases for the P-incorporation analysis of dative-to-nominative conversion. In this subsection, I discuss data concerning claims that theme passivisation with bare recipients is also available with prefixed verbs. Since P-incorporation makes the recipient a valid target for subjecthood and blocks VP-internal scrambling, theme passivisation should generally be impossible in these cases, unless the recipient has moved. One type of potential counter-example that can be solved in this way are cases of purported theme passivisation with bare pronominal recipients. Cliticisation of pronominal recipients could explain why pronominal recipients can stay low in these cases.

\begin{exe}
	\ex Swedish:\label{ex:sw-offer-pass}
	\gll Ett nytt jobb erbjöds=honom.\\
A new job offered.PASS=him.OBL.\\
\trans `A new job was offered to him (\citealt{Anward.1989},\citealt{Falk.1990},\citealt{Lundquist.2006}).'
\end{exe}

If this is true, Swedish gives further clarity about the cliticisation process, since theme passivisation with unmarked recipients is \textbf{only} available with particle verbs. This suggests that in Swedish, the cliticisation process is restricted to DP pronouns. Pronouns in a dative PP (i.e., in non particle verbs) are unable to cliticise and thus serve as defective interveners for direct theme passivisation (see previous subsection).

\begin{exe}
	\ex Swedish:\label{ex:sw-give-pass}
	\gll *Ett äpple gavs honom.\\
	 An apple gave.PASS him.\\
	 \trans `An apple was given to him (\citealt{Anward.1989},\citealt{Lundquist.2006}).'
\end{exe}

However, \cite{Lundquist.2004} claims that there are examples of theme passivisation with full noun phrases with prefixed verbs.

\begin{exe}
	\ex Swedish:\label{ex:sw-offer-thepas}
	\gll Jobbet erbjöds mannen med den långa svarta kappan.\\
	job.DEF offered.PASS man.DEF with the long black coat\\
	'The job was offered to the man with the long black coat \citep[ex 26]{Lundquist.2004}.'
\end{exe}

If the prefixed verbs reflect P-incorporation, as I have argued, then direct theme passivisation is not a possible explanation (since the recipient is a valid target for subject movement). Instead, I claim that these are actually cases of recipient passivisation with theme topicalisation. Since Swedish is a V2 language, there is an ambiguity for sentence initial elements between a subject and topic interpretation. \cite{Lundquist.2004} provides examples in which themes occur in unambiguous subject positions (i.e., between an auxiliary and the passive participle) and such examples are degraded.

\begin{exe}
	\ex Swedish:\label{ex:sw-relpass}
	\begin{xlist}
		\ex[ ]{\gll \textbf{DET} \textbf{jobbet} har Kalle tilldelats.\\
		that job.DEF has Kalle assigned.PART.PASS\\
		\trans `THAT job, Kalle has been assigned \citep[ex. 59]{Lundquist.2004}.'}
		\ex[??]{\gll DEN mannen har jobbet tilldelats.\\
		that man.DEF has job.DEF assigned.PART.PASS\\
		\trans `To THAT man, the job has been assigned \citep[ex. 58]{Lundquist.2004}.'}
	\end{xlist}
\end{exe}

Another piece of evidence comes from the distribution of recipient and theme passivisation in corpora. \cite{Lundquist.2004} shows that recipient passivisation is extremely prevalent in modern Swedish (with prefix verbs), while theme passivisation is quite rare. This difference is explained if purported examples of theme passivisation are actually cases of theme topicalisation, which is expected to happen at relatively low rates in a corpus.

One challenge for this view is that there are cases where the recipient seems to not (obligatorily) occur in subject position. Since Swedish generally requires expletives when the subject position is not filled, this analysis would require that null expletives be licensed in theme relative clauses. 

\begin{exe}
	\ex Swedish:\label{ex:sw-relpass2}
	\begin{xlist}
		\ex[ ]{\gll Jobbet som erbjöds \textbf{mannen} var mycket slitsamt.\\
		job.DEF which offered.PASS \textbf{man.DEF} was very tiring\\
		\trans `The job, which was offered to the man, was very tiring \citep[ex. 49]{Lundquist.2004}.'}
		\ex[ ]{\gll Jobbet som \textbf{mannen} erbjöds var mycket slitsamt.\\
		job.DEF which \textbf{man.DEF} offered.PASS was very tiring\\
		\trans `The job, which the man was offered, was very tiring \citep[ex. 50]{Lundquist.2004}.'}
	\end{xlist}
\end{exe}

Interestingly, \cite{Haddican.2015} note that although theme passives with null recipients are generally judged unacceptable in American English, theme relative clauses are often judged much better. Since the relationship between theme relative clauses and bare recipients is replicated across at least two languages, it seems worthy of further research into the relationship between the head of relative clauses and the internal properties of the clause. However, such an investigation of relative clause structure is outside the scope of this dissertation.

\begin{exe}
	\ex Modern American English (Recipient Relatives):\label{ex:amen-relpass}
	\begin{xlist}
		\ex[ ]{The man, who was given the book, read.}
		\ex[?]{The man, who the book was given to, read.}
		\ex[??]{The man, who the book was given, read.}
	\end{xlist}
	\ex Modern American English (Theme Relatives):\label{ex:amen-rellpass2}
	\begin{xlist}
		\ex[?]{The book, which the man was given, was red}
		\ex[ ]{The book, which was given to the man, was red}
		\ex[??]{The book, which was given the man, was red}
	\end{xlist}
\end{exe}

In summary, Swedish has been claimed to allow bare recipient theme passives with prefixed verbs. Since I claim that bare recipient theme passives reflect direct theme movement past a PP recipient and that prefixed verbs in Swedish reflect P-incorporation creating a DP recipient, this data would contradict the claims I am making here. In this subsection, I argued that the Swedish data that had been used to make this claim was actually ambiguous with a theme topicalisation analysis and that data about themes in unambiguous subject position suggests that theme passivisation is not compatible with prefixed verbs in Swedish. Finally, this requires that some clauses in Swedish (especially relative clauses) may not have anything in subject position. Explaining where languages require filled subjects and where they do not is too far from the central point of this dissertation, but I also point out that similar patterns in relative clauses have been noticed to occur in American English.

\section{Conclusions}
This chapter analysed passivisation of recipient ditransitives. P-incorporation converted dative recipients into unmarked DPs, licensing dative-to-nominative conversion. This incorporation was seen on the surface in Dutch, German and Swedish. Oblique subjects were analysed by splitting the movement and case assignment properties of T into different searches (with different domains of application). In addition, surface theme passivisation with nominative themes were shown to arise from a number of possible mechanisms for avoiding locality violation, namely: relativised minimality, VP-internal scrambling, and recipient scrambling/cliticisation. Relativised minimality was argued to result in direct theme passivisation, where the theme moved to subject position directly from its base merged position.

%\bibliography{diss}
