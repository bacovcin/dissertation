This dissertation investigates the interplay of morphology and syntax in generating surface complexity and the universality of argument structure by analysing recipient ditransitives in Germanic. Keeping the semantics constant (recipients) permits direct comparison of the syntactic variation between the Germanic languages. The main claim of the dissertatiion is that all recipients are introduced as dative PPs in the specifier of an applicative phrase. This conclusion supports a strong version of Baker's UTAH hypothesis, namely that there is no variation between natural languages in argument structure and that all surface variation is derived from transformations on a uniform underlying structure.

In addition to arguing for the base generated structure of recipient ditransitives, this dissertation also explores transformations that apply to the base structure and show how these transformations are able to account for the surface variation seen both synchronically and diachronically in Germanic. Morphological variation in the form of allomorphy in the realisation of the dative P head is argued to cause the variation seen in Dative Shift (e.g. ``John gave Mary the book'' vs ``John gave the book to Mary''). In addition to the morphological variation, languages also varied as to the availability of different syntactic transformations.

For active sentence, the main syntactic transformation is VP-internal scrambling, which moves the theme over the recipient to generate theme--recipient word orders (e.g., ``John gave the book to Mary''). Also, pronominal cliticisation can effect the morpholgoical realisation of dative case. In the passive, P-incorporation is argued to license dative-to-nominative recipient subject raising. Theme passivisation is argued to be licensed by a number of different syntactic methods, including relativised minimality with respect to the PP/DP distinction.

The original contribution fo this dissertation is two-fold. First, while none of the components of the base generation analysis (nor the syntactic transformations) is unique to this dissertation, the analysis presented here combines these components in a way that has not been proposed previously. In addition, this is the first time that a substantial review of data from the ditransitives of all the Germanic languages has been brought together in one place.
