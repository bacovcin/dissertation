\documentclass{beamer}
\usetheme{Frankfurt}
\usepackage[T1]{fontenc}
\usepackage[authoryear]{natbib}
\usepackage{gb4e}
\usepackage{dcolumn}
\usepackage{hyperref}
\usepackage[all]{xypic}
\noautomath
\setbeamertemplate{footline}[frame number]

\mode<handout>
{
  \usepackage{pgf}
  \usepackage{pgfpages}

\pgfpagesdeclarelayout{4 on 1 boxed}
{
  \edef\pgfpageoptionheight{\the\paperheight} 
  \edef\pgfpageoptionwidth{\the\paperwidth}
  \edef\pgfpageoptionborder{0pt}
}
{
  \pgfpagesphysicalpageoptions
  {%
    logical pages=4,%
    physical height=\pgfpageoptionheight,%
    physical width=\pgfpageoptionwidth%
  }
  \pgfpageslogicalpageoptions{1}
  {%
    border code=\pgfsetlinewidth{2pt}\pgfstroke,%
    border shrink=\pgfpageoptionborder,%
    resized width=.5\pgfphysicalwidth,%
    resized height=.5\pgfphysicalheight,%
    center=\pgfpoint{.25\pgfphysicalwidth}{.75\pgfphysicalheight}%
  }%
  \pgfpageslogicalpageoptions{2}
  {%
    border code=\pgfsetlinewidth{2pt}\pgfstroke,%
    border shrink=\pgfpageoptionborder,%
    resized width=.5\pgfphysicalwidth,%
    resized height=.5\pgfphysicalheight,%
    center=\pgfpoint{.75\pgfphysicalwidth}{.75\pgfphysicalheight}%
  }%
  \pgfpageslogicalpageoptions{3}
  {%
    border code=\pgfsetlinewidth{2pt}\pgfstroke,%
    border shrink=\pgfpageoptionborder,%
    resized width=.5\pgfphysicalwidth,%
    resized height=.5\pgfphysicalheight,%
    center=\pgfpoint{.25\pgfphysicalwidth}{.25\pgfphysicalheight}%
  }%
  \pgfpageslogicalpageoptions{4}
  {%
    border code=\pgfsetlinewidth{2pt}\pgfstroke,%
    border shrink=\pgfpageoptionborder,%
    resized width=.5\pgfphysicalwidth,%
    resized height=.5\pgfphysicalheight,%
    center=\pgfpoint{.75\pgfphysicalwidth}{.25\pgfphysicalheight}%
  }%
}


  \pgfpagesuselayout{4 on 1 boxed}[letterpaper, border shrink=5mm, landscape]
  \nofiles
  }


\title[]{\textbf{Parameterising Germanic ditransitive variation: A historical-comparative study}}
\author{Hezekiah Akiva Bacovcin}
\institute{University of Pennsylvania - Department of Linguistics}


\begin{document}
\begin{frame}
\titlepage
\end{frame}

\section{Outline}
\begin{frame}
\frametitle{Outline}
\begin{enumerate}
\item Introduction
\item Theoretical Analysis
\item Swedish Data
\item Historical English Data
\item Conclusions and Questions
\end{enumerate}
\end{frame}

\section{Introduction}
\begin{frame}
		\frametitle{Contributions}
		\begin{enumerate}
			\item Defends a strong version of the Uniformity of Theta Assignment Hypothesis
			\item New analysis of recipient ditransitives: \textbf{Recipients are universally introduced as dative PPs in the specifier of an applicative phrase}
			\item Complete syntax framework: Complete set of additional syntactic machinery to explain Germanic recipient ditransitive data
			\item Typological summary: Collection of all of the relevant data on Germanic recipient ditransitives in one place
			\item New historical data: Collection of new data on the history of English from parsed corpora
			\item New historical methods: Applications of new techniques to use historical data in theoretical syntax investigations
		\end{enumerate}
\end{frame}

\section{Theoretical Analysis}
\begin{frame}
	\frametitle{Theory Outline}
	\begin{itemize}
		\item What is a recipient?
		\item What is a dative PP?
		\item Where is the specifier of an applicative phrase?
		\item What syntactic mechanisms do we need to posit for empirical coverage?
		\item Why should we prefer this analysis \textbf{on conceptual grounds}?
	\end{itemize}
\end{frame}
\begin{frame}
	\frametitle{Question}
	\vfill
	\Large
		What is a recipient?
	\vfill
\end{frame}
\subsection{Thematic Roles}
\begin{frame}
	\frametitle{Theta Role Theory}
	\begin{itemize}
		\item Recipient is one of a number of theta roles
		\item Building on \cite{Dowty.1991}:
			\begin{itemize}
				\item Theta roles are morphosyntactic categories (like singular/plural)
				\item Each role is associated with prototypical features
				\item Prototypes \textbf{guide} assignment of theta roles to agruments
				\item Ambiguous cases are determined by local cultural/linguistic norms
			\end{itemize}
	\end{itemize}
\end{frame}

\begin{frame}
	\frametitle{Recipients}
	\begin{itemize}
		\item Recipients are the endpoint of a transfer of possession event (that may or may not involve movement)
		\item GIVE is the prototypical recipient action (an AGENT transfers a THEME to a RECIPIENT)
	\end{itemize}
\end{frame}

\begin{frame}
	\frametitle{Question}
	\vfill
	\Large
		What is a dative PP?
	\vfill
\end{frame}
\subsection{Case Theory}
\begin{frame}
	\frametitle{Structural vs. Non-structural case}
	\begin{itemize}
		\item Long history of a distinction between structural and non-structural case (\citet{Woolford.2006} gives a summary of arguments)
		\item Structural cases alternate, while non-structural cases do not
		\item Dative case (typical for recipients) has usually been regarded as non-structural
	\end{itemize}
\end{frame}

\begin{frame}
	\frametitle{Prepositional theory of non-structural case}
	\begin{itemize}
		\item Non-structural case can be distinguished by the addition of an extra syntactic layer \citep{Bittner.1996,Bayer.2001}
		\item Recent proposals associate this extra layer with prepositional phrases \citep{Asbury.2005,Asbury.2007,Rezac.2008,Caha.2009}
		\item As a non-structural case, the case marking on recipients (often called dative) is actually just the morphological reflex of a preposition concord
	\end{itemize}
\end{frame}

\begin{frame}
	\frametitle{Prepositions, thematic roles, and case}
	\begin{itemize}
		\item The preposition that introduces an argument (or adjunct) is \textbf{syntactically} associated with the thematic role of the argument
		\item All recipients are introduced with a recipient P
		\item Example: [P$_{recipient}$ [DP Recipient]]
	\end{itemize}
\end{frame}

\begin{frame}
	\frametitle{Structural cases}
	\begin{itemize}
		\item Nominative and accusative (structural cases) are features of ungoverned DPs (i.e., not embedded in PPs)
		\item DPs under prepositions do \textbf{not} receive syntactic structural case (although the reflexes of non-strucutral case may be morphologically syncretic with structural cases)
	\end{itemize}
\end{frame}

\begin{frame}
	\frametitle{Question}
	\vfill
	\Large
	Where is the specifier of an applicative phrase?
	\vfill
\end{frame}
\subsection{Argument Structure}
\begin{frame}
	\frametitle{Applicative Analysis}
			\xymatrix@=2pt{
			& ApplP\ar@{-}[dl]\ar@{-}[dr]\\
			PP_{\text{Recipient}} & & \bar{Appl}\ar@{-}[dl]\ar@{-}[dr]\\
			& Appl & & VP\ar@{-}[dl]\ar@{-}[dr]\\
		    & & DP_{\text{Theme}} && \bar{V}\ar@{-}[d]\\
			&&&&V}
\end{frame}

\begin{frame}
	\frametitle{Properties of Analysis}
	\begin{itemize}
		\item Applicatives do \textbf{not} assign thematic roles (prepositions do)
		\item Applicatives only provide a syntactic position for applied arguments to enter the derivation
		\item Completely exchangable with a Larsonian VP-shell analysis \citep{Larson.1988}
	\end{itemize}
\end{frame}

\begin{frame}
	\frametitle{Question}
	\vfill
	\Large
		What syntactic mechanisms do we need to posit for empirical coverage?
	\vfill
\end{frame}
\subsection{Morphosyntactic Operations}
\begin{frame}
	\frametitle{Five morphosyntactic operations}
	\begin{enumerate}
		\item Contextual allomorphy
		\item VP-internal scrambling
		\item Cliticisation
		\item P-incorporation
		\item Locality/intervention effects
	\end{enumerate}
\end{frame}

\begin{frame}
	\frametitle{English Examples}
\begin{exe}
	\ex Active
	\begin{xlist}
		\ex I gave the woman the book.
		\ex I gave the book to the woman.
		\ex I gave it the woman
	\end{xlist}
\end{exe}
\end{frame}

\begin{frame}
	\frametitle{Contextual allomorphy}
	\begin{itemize}
		\item The realization of recipient P is subject to allomorphy
		\item In some languages, this allomorphy is determined contextually by adjacency to particular words
		\item Same mechanism that determines "many dog-s" vs. "many sheep"
	\end{itemize}
\end{frame}

\begin{frame}
	\frametitle{Early Modern English: Analysis}
	\begin{exe}
		\ex Vocabulary Items (14th--18th Centuries):
		\begin{xlist}
			\ex Null Allomorph Item: /$\emptyset$/ $\leftrightarrow$ [dative P] / verb$^{\smallfrown}$\_
			\ex To Item: /tu/ $\leftrightarrow$ [dative P]
		\end{xlist}
	\end{exe}
\end{frame}


\begin{frame}
	\frametitle{English}
	\vfill
	I gave $\emptyset$ the woman the book
	\vfill
\end{frame}

\begin{frame}
	\frametitle{VP-internal scrambling}
	\begin{itemize}
		\item Many languages allow two word orders for recipients and themes (RT vs TR)
		\item RT word order is base generated; TR derived by VP-internal scrambling \citep{Takano.1998,Lenerz.1977}
		\item Theme moves to a higher specifier of the ApplP \citep{McGinnis.1998}
	\end{itemize}
\end{frame}

\begin{frame}
	\frametitle{Scrambling Analysis:}
	\begin{exe}
		\ex
\xymatrix@=2pt{
 & vP\ar@{-}[dl]\ar@{-}[dr]\\
DP\ar@{-}[d] & & \bar{v}\ar@{-}[dl]\ar@{-}[dr]\\
\text{I} & \text{v} & & ApplP\ar@{-}[dl]\ar@{-}[dr]\\
 & & DP\ar@{-}[d] & & \bar{Appl}\ar@{-}[dl]\ar@{-}[dr]\\
 & & \text{the book$_{i}$}\ar@{<-}@(dl,dl)[ddrrr] & PP\ar@{-}[d] & & \bar{Appl}\ar@{-}[dl]\ar@{-}[dr]\\
 & & & \text{to the woman} & Appl & & VP\ar@{-}[dl]\ar@{-}[dr]\\
 & & & & & DP_{i} & & \bar{V}\ar@{-}[d]\\
 & & & & & & & V\ar@{-}[d] \\
 & & & & & & & \text{give}\\}
	\end{exe}
\end{frame}

\begin{frame}
	\frametitle{Cliticisation}
	\begin{itemize}
		\item Cliticisation is head movement of a pronoun into a higher element
		\item Allows two constructions (``violating'' locality):
		\begin{enumerate}
			\item Theme cliticisation: John [gave it] $\emptyset$ him.
			\item Recipient cliticisation: The book was [given $\emptyset$ him].
		\end{enumerate}
	\end{itemize}
\end{frame}

\begin{frame}
	\frametitle{English Examples}
\begin{exe}
	\ex Active
	\begin{xlist}
		\ex The woman was given the book.
		\ex To the woman was given the book.
		\ex The book was given to the woman.
		\ex The book was given the woman.
	\end{xlist}
\end{exe}
\end{frame}

\begin{frame}
	\frametitle{P-incorporation}
	\begin{itemize}
		\item P-incorporation allows dative-to-nominative conversion \citep{Alexiadou.2014}
		\item P-incorporation is a type of head movement
	\end{itemize}
 \end{frame}

 \begin{frame}
	\frametitle{English}
	\vfill
	The woman was [given P=$\emptyset$] the book
	\vfill
\end{frame}

 \begin{frame}
	\frametitle{Head Movement Targeting Condition}
	\vfill
	 	\textbf{Head Movement Targeting Condition:} When head movement is triggered, the head triggering the movement adjoins to the nearest head that asymmetrically c-commands the highest position of the triggering head.
	\vfill
 \end{frame}
 \begin{frame}
	 \frametitle{Incompatibility with VP-internal scrambling}
 \xymatrix@=2pt{
 & vP\ar@{-}[dl]\ar@{-}[dr]\\
	 \text{v}_{\text{passive}} & & ApplP\ar@{-}[dl]\ar@{-}[dr]\\
 & DP\ar@{-}[d] & & \bar{Appl}\ar@{-}[dl]\ar@{-}[dr]\\
 & \text{the book$_{i}$}\ar@{<-}@(dl,dl)[ddrrr] & PP\ar@{-}[d] & & \bar{Appl}\ar@{-}[dl]\ar@{-}[dr]\\
 & & \text{to the woman} & Appl & & VP\ar@{-}[dl]\ar@{-}[dr]\\
 & & & & DP_{i} & & \bar{V}\ar@{-}[d]\\
 & & & & & & V\ar@{-}[d] \\
 & & & & & & \text{give}}
\end{frame}

\begin{frame}
	\frametitle{Passivisation and Properties of T}
	\begin{itemize}
		\item Two crucial properties: Locality and Argument Validity
		\item Assumptions:
		\begin{enumerate}
			\item Only DPs (not PPs) can receive nominative case
			\item Languages vary in whether or not PPs can move to subject position
			\item Language vary in whether they can inspect multiple arguments for movement/case assignment
		\end{enumerate}
	\end{itemize}
\end{frame}

\begin{frame}
	\frametitle{Consequences of T variation}
	\begin{tabular}{cc}
		&	PP Subjects \\
Two Arguments	& DAT recipient subject and NOM theme object \\
One Argument    & ?DAT recipient subject and ACC theme object\\
\hline
	      & No PP Subjects \\
Two Arguments & NOM theme subject and DAT recipient object\\
One Argument  & Ungrammaticality\\
	\end{tabular}
\end{frame}

\begin{frame}
	\frametitle{Question}
	\vfill
	\Large
		Why should we prefer this analysis \textbf{on conceptual grounds}?
	\vfill
\end{frame}

\begin{frame}
	\frametitle{Answer}
	\begin{itemize}
		\item Empirical coverage with smallest number of syntactic tools
		\item Almost all of the machinery is already necessary:
			\begin{itemize}
				\item For other constructions (e.g., prepositions)
				\item For non-English languages (e.g., VP-internal scrambling)
			\end{itemize}
	\end{itemize}
\end{frame}

\section{Swedish}
\subsection{Introduction}
\begin{frame}
	\frametitle{Swedish Introduction}
	\begin{itemize}
		\item Swedish provides best evidence for prepositional analysis of recipient case
		\item Necessary facts about Swedish:
		\begin{itemize}
			\item No remaining synthetic accusative--dative distinction
			\item V2 language
		\end{itemize}
	\end{itemize}
\end{frame}

\begin{frame}
	\frametitle{Swedish Incorporation}
	\begin{itemize}
		\item \textbf{Swedish shows P-incorporation overtly} \citep{Holmberg.1995}
		\item Non-Particle Verbs (e.g., \textit{gav} `give'):
		\begin{itemize}
				\item Allow TR and RT orders
				\item Allow overt prepositional marking on the recipient
				\item Only allow theme passivisation (with prepositional marking on recipient)
		\end{itemize}
	\item Particle Verbs (e.g., \textit{er-bjod} `offer'):
		\begin{itemize}
				\item Only allow RT orders
				\item Never allow prepositional marking on the recipient
				\item Only allow recipient passivisation
		\end{itemize}
	\end{itemize}
\end{frame}
\subsection{Non-Particle Verbs}
\begin{frame}
	\frametitle{Swedish Non-Particle Verbs: Active}
	\begin{exe}
	\ex Swedish:
	\gll Jag gav \textbf{Johan} en bok.\\
		I gave John a book.\\
		\trans `I gave John a book \citep{Holmberg.1995}.'
	\ex Swedish:
	\gll Jag gav en bok \textbf{*(til)} \textbf{Johan}.\\
		I gave a book to John.\\
		\trans `I gave a book to John \citep{Holmberg.1995}.'
	\end{exe}
\end{frame}

\begin{frame}
	\frametitle{Swedish Non-Particle Verbs: Passive}
	\begin{exe}
		\ex[*]{\gll \textbf{Pelle} gavs ett äpple\\
			Pelle gave.PASS a apple\\
			\trans `Pelle was given an apple (\citealt{Anward.1989}, \citealt{Lundquist.2006}).'}
		\ex[ ]{
			\gll \textbf{Ett} \textbf{äpple} gavs *(til) Pelle.\\
			An apple gave.PASS *(to) Pelle.\\
			 \trans `An apple was given to Pelle (\citealt{Anward.1989},\citealt{Lundquist.2006}).'}
	\end{exe}
\end{frame}
\subsection{Active}
\begin{frame}
	\frametitle{Swedish Particle Verbs: Active}
	\begin{exe}
		\ex[ ]{\gll Han erbjöd \textbf{Jan} ett nytt jobb\\
				he.NOM offered John a new job\\
			\trans `He offered John a new job'}
			\ex[??]{\gll Han erbjöd ett nytt jobb \textbf{til} \textbf{Jan}\\
				he.NOM offered a new job to John\\
			\trans `He offered a new job to John'}
			\ex[*]{\gll Han erbjöd ett nytt jobb \textbf{Jan}\\
				he.NOM offered a new job John\\
			\trans `He offered a new job to John'}
\end{exe}
\end{frame}
\subsection{Recipient Passive}
\begin{frame}
	\frametitle{Swedish Particle Verbs: Recipient Passive}
\begin{exe}
		\ex Recipient passive:\\
		\gll \textbf{Han} erbjöds ett nytt jobb\\
			he.NOM offered.PASS a new job\\
			\trans `He was offered a new job (\citealt{Anward.1989}, \citealt{Lundquist.2006}).'
\end{exe}
\end{frame}

\subsection{Theme Passive}
\begin{frame}
	\frametitle{Swedish Particle Verbs: Theme Passive (Take 1)}
	\begin{exe}
	\ex Recipient clitic passive:\\
	\gll \textbf{Ett} \textbf{nytt} \textbf{jobb} erbjöds=honom.\\
		A new job offered.PASS=him.OBL.\\
		\trans `A new job was offered to him (\citealt{Anward.1989},\citealt{Falk.1990},\citealt{Lundquist.2006}).'
	\ex ``Theme passive'': \\
	\gll \textbf{Jobbet} erbjöds mannen med den långa svarta kappan.\\
		job.DEF offered.PASS man.DEF with the long black coat\\
		'The job was offered to the man with the long black coat \citep[ex 26]{Lundquist.2004}.'
\end{exe}
\end{frame}

\begin{frame}
	\frametitle{Problem!!!}
	\begin{itemize}
		\item Theme passive with P-incorporation should be impossible (without recipient cliticization)
		\item But theme passives are reported grammatical with particle verbs in Swedish!!!
		\pause
		\item Solution: In V2 languages, preverbal position is \textbf{not} an unambiguous subject position
	\end{itemize}
\end{frame}

\begin{frame}
	\frametitle{Swedish Particle Verbs: Theme Passive Redo}
	\begin{exe}
	\ex Between auxiliary and participle:
		\begin{xlist}
			\ex[ ]{\gll \textbf{DET} \textbf{jobbet} har Kalle tilldelats.\\
			that job.DEF has Kalle assigned.PART.PASS\\
			\trans `THAT job, Kalle has been assigned \citep[ex. 59]{Lundquist.2004}.'}
			\ex[??]{\gll DEN mannen har \textbf{jobbet} tilldelats.\\
			that man.DEF has job.DEF assigned.PART.PASS\\
			\trans `To THAT man, the job has been assigned \citep[ex. 58]{Lundquist.2004}.'}
		\end{xlist}
\end{exe}
\end{frame}
\subsection{Conclusions}
\begin{frame}
	\frametitle{Cconclusion}
	\vfill
	\Large
	Swedish verbs show P-incorporation overtly
	\vfill
\end{frame}

\section{Historical English}
\subsection{Introduction}
\begin{frame}
	\frametitle{Historical Oultine}
	\begin{enumerate}
		\item Background on Technique
		\item Case \#1: Introduction of P-incorporation
		\item Case \#2: Introduction of recipient ``to''
	\end{enumerate}
\end{frame}

\begin{frame}
	\frametitle{Background}
	\begin{itemize}
		\item Syntactic variation is caused by ``competing grammars'' \citep{Kroch.1989}
		\item The choice between grammatical options can be modelled using logistic regression
		\item Time based regression models captures \textbf{diffusion} of a syntactic option
	\end{itemize}
\end{frame}

\begin{frame}
	\frametitle{Linking Hypothesis}
	\vfill
	\Large
	Shared historical behaviour \textbf{implies} shared syntactic representation
	\vfill
\end{frame}

\begin{frame}
	\frametitle{Technical Details}
	\begin{itemize}
		\item Bayesian models used with weakly informative priors
		\item Results interpretation: Probability distribution over parameter values
		\item All models fit with STAN
	\end{itemize}
\end{frame}

\begin{frame}
	\frametitle{Technical Takeaway}
	\vfill
	Interaction term of zero \textbf{suggests} shared historical behaviour
	\vfill
\end{frame}
\subsection{Case 1: P-incorporation}
\begin{frame}
	\frametitle{Case 1: P-incorporation}
	\vfill
	\Large
	Case \#1: P-incorporation
	\vfill
\end{frame}
\begin{frame}
	\frametitle{Shared Property}
	\begin{itemize}
		\item The existence of P-incorporation permitted two surface constructions:
			\begin{enumerate}
				\item Nominative recipient passives
				\item Pseudopassives
			\end{enumerate}
	\end{itemize}
\end{frame}

\begin{frame}
	\frametitle{Recipient Passive}
	\begin{exe}
		\ex Middle English \citep{Kroch.2000} and Early Modern English \citep{Kroch.2004}
		\begin{xlist}
			\ex\label{ex:obliq-to-rec-eng} to thy holy name be given laude and praise (STOW-E2-P2,581.96)
			\ex\label{ex:obliq-rec-eng} the king Gurthym, that we clepteth Gurmundus, were i-yeve the provinces of Est Anglia and Northumbria (CMPOLYCH-M3,VI,377.2770)
			\ex\label{ex:nom-rec-eng} for the prioress is given a matter to proud in the beginning of her ordinance (CMBENRULE-M3,43.1346)
		\end{xlist}
	\end{exe}
\end{frame}

\begin{frame}
	\frametitle{Pseudopassive}
	\begin{exe}
		\ex Pseudopassive:
		\begin{xlist}
			\ex I slept in the bed.
			\ex The bed was slept in.
		\end{xlist}
	\end{exe}
\end{frame}

\begin{frame}
	\frametitle{Graph}
	\includegraphics[width=\linewidth,height=.8\paperheight]{output/images/recpas-old-pseudo}
\end{frame}

\begin{frame}
	\frametitle{Parameter Estimates}
	\input{output/tables/recpas-old-mcmc.tex}
\end{frame}

\subsection{Case 2: Rise and Fall of `to'}
\begin{frame}
	\frametitle{Case 2: Rise and Fall of `to'}
	\vfill
	\Large
	Case \#2: Rise and Fall of `to'
	\vfill
\end{frame}
\begin{frame}
	\frametitle{Stages of `to'}
	\begin{enumerate}
		\item Old English: Recipient P realised as null 
		\item Early Middle English: Spread of `to' realisation
		\item Early Modern English: Spread of null contextual allomorph
	\end{enumerate}
\end{frame}

\begin{frame}
	\frametitle{Old English: Data}
	\begin{exe}
		\ex Examples using Both Word Orders:
		\begin{xlist}
			\ex \gll and sealde healfne dael (*to) \TH am gesaeligan \TH earfan\\
			and gave half portion.ACC to the.DAT blessed.DAT needy.DAT\\
			\trans ` and gave a half portion to the blessed needy (coaelive.03,+ALS\_[Martin]:69.6009)'
			\ex \gll Man sceal eac syllan (*to) \TH am seocan men husel\\
			one should also give to the.DAT sick.DAT man.DAT eucharist.ACC\\
			\trans `One should also give the sick man eucharist (coaelhom.03,+AHom\_11:177.1583)'
		\end{xlist}
	\end{exe}
\end{frame}

\begin{frame}
	\frametitle{Old English: Analysis}
	\begin{exe}
		\ex Vocabulary Items (6th--11th Centuries):\label{ex:old-eng-vi}
		\begin{xlist}
			\ex Universal Null Item:  /$\emptyset$/ $\leftrightarrow$ [dative P]
		\end{xlist}
	\end{exe}
\end{frame}

\begin{frame}
	\frametitle{Middle English: Data}
	\begin{exe}
		\ex Examples using Both Word Orders:
		\begin{xlist}
		\ex I have given \textbf{Purry} a gown (PASTON,I,232.2716)
		\ex They gave \textbf{to the people} this bread (CMWYCSER-M3,248.452)\label{ex:mideng-rt-to}
		\ex Thou givest thine aught (possessions) \textbf{God} (CMVICES1-M1,37.437)
		\ex Lord, in thy will, thou gave virtue \textbf{to my fairness} (CMEARLPS-M2,32.1360)
	\end{xlist}
	\end{exe}
\end{frame}

\begin{frame}
	\frametitle{Middle English: Analysis}
	\begin{exe}
		\ex Competing Vocabulary Items (11th--14th Centuries):\label{ex:simple-vis-eng}
		\begin{xlist}
			\ex Universal Null Item: /$\emptyset$/ $\leftrightarrow$ [dative P]
			\ex To Item: /tu/ $\leftrightarrow$ [dative P]
		\end{xlist}
	\end{exe}
\end{frame}

\begin{frame}
	\frametitle{English `to' Graph}
	\includegraphics[width=\linewidth,height=.85\paperheight]{output/images/to-use}
\end{frame}

\begin{frame}
	\frametitle{Early Modern English: Analysis}
	\begin{exe}
		\ex Vocabulary Items (14th--18th Centuries):
		\begin{xlist}
			\ex (Null Allomorph Item: /$\emptyset$/ $\leftrightarrow$ [dative P] / verb$^{\smallfrown}$\_)
			\ex To Item: /tu/ $\leftrightarrow$ [dative P]
		\end{xlist}
	\end{exe}
\end{frame}

\begin{frame}
	\frametitle{Quantitative Results}
	\input{'output/tables/to-mcmc.tex'}
\end{frame}

\section{Conclusions}
\subsection{Conclusions}
\begin{frame}
	\frametitle{Conclusions}
	\begin{itemize}
		\item Historical data can provide independent evidence for syntactic theory
		\item Recipients are base-generated as PPs in the specifier of an applicative phrase
		\item Supports a strong version of the Uniformity of Theta Assignment Hypothesis
	\end{itemize}
\end{frame}
\subsection{Thank you}
\begin{frame}
\vfill
\begin{center}
THANK YOU!
\end{center}
\vfill
\end{frame}

\begin{frame}
\vfill
\begin{center}
	QUESTIONS?
\end{center}
\vfill
\end{frame}


\begin{frame}[allowframebreaks]
\frametitle{References}
\nocite{Haddican.2011,Haddican.2012,Gerwin.2013}
\bibliographystyle{tex/book/mcbride}
\bibliography{tex/diss}{}
\end{frame}

\end{document}
